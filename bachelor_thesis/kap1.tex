\chapter{Existující metody a systémy zálohování}

Jak již bylo popsáno v~úvodu práce, dá se k~zálohování přistupovat mnoha způsoby.
Pojďme si zkusit tyto způsoby rozebrat, podívat se na ně z~několika hledisek
efektivity a zhodnotit jejich klady a zápory. Dále se zkusme podívat na
několik existujících zálohovacích řešení.

\section{Jednoduché kopírování}

Při zálohování vždy jde o~to držet si nějakým způsobem kopii nějakých dat. Výběr
dat k~zálohování nyní odložíme stranou a budeme si představovat, že máme nějakou
dobře definovanou množinu souborů, které chceme zálohovat.

\newacronym{FTP}{FTP}{File Transfer Protocol -- jednoduchý síťový protokol pro
přenos souborů}
\newacronym{SSH}{SSH}{Secure shell -- zabezpečený komunikační síťový protokol}
\newacronym{SCP}{SCP}{Secure copy -- zabezpečený protokol pro přenos souborů na
bázi \gls{SSH}}

Nejjednodušším způsobem co do implementace je asi {\bf ruční kopírování} --
uživatel sám čas od času zkopíruje zálohované soubory na nějaké jiné úložiště.
Na tomto úložišti si uživatel může držet buď jen jednu kopii dat (vždy jen
nejnovější verzi), což zabere řádově tolik místa, kolik zabírají zálohovaná data
na zálohovaném stroji, nebo si může držet všechny historické verze. V~takovém
případě ale zálohy zaberou místo $[\hbox{počet záloh}]\times[\hbox{velikost dat}]$,
což je pro časté zálohování velkého objemu dat neudržitelné (objem dat lze sice
snížit vhodnou kompresí, ale to oddálí problém jen o~kus dále).

Jako metodu kopírování lze zvolit mnoho protokolů, od kopírování po místním
souborovém systému (což nebývá moc účinné proti selhání celého počítače nebo
jeho ztrátou), přes kopírování přes síť pomocí \gls{FTP} nebo nad \gls{SSH}
postaveným \gls{SCP}.

Všechny tyto metody kopírování jsou ale \uv{hloupé}, udělají jen přesně to, co
od nich žádáme -- zkopírují zadané soubory na cílové úložiště bez starosti o~to,
jestli již zde stejné soubory nebyly kopírovány dříve. Pokud byly provedeny jen
malé změny, tak přenášíme zbytečně velké objemy dat, které již v~cílovém místě
uložená jsou.

Kdybychom dokázali dostatečně bezpečně poznat, jaká data se nezměnila, mohli
bychom objem přenášených dat výrazně snížit a kopírovat jen rozdílná data.
Když zálohování provádí uživatel sám a ručně, může posloužit jako rozhodovací
veličina v~tom, která data se podle jeho mínění změnila a která je tedy potřeba
zálohovat. To je ale jednak nespolehlivé a může to vést k~chybám, ale hlavně se
tento postup nedá použít při zautomatizování zálohování (což by mělo být jedním
z~hlavních cílů zálohovacích systémů).

\newpage

\section{Kopírování rozdílů}

Když si vezmeme za cíl kopírovat jen ta data, která se oproti poslednímu
zálohovanému stavu změnila, dostáváme se k~otázce, jak dostatečně bezpečně
poznat, co se změnilo.

Vzhledem k~vlastnostem v~současnosti používaných souborových systémů je
nejsnadnější dívat se na změny soubor po souboru a v~případě změny ho přenést
celý. Dá se zavést i drobnější granularita a sledovat data po menších blocích,
než po celých souborech, ale to je již výrazně obtížnější.

\newacronym{UNIX}{UNIX}{Operační systém vzniklý v~roce 1969, který dal vzniknout
celé rodině odvozených systémů -- Linux, BSD, Mac OS X, aj.}

Současné souborové systémy si u~každého souboru pamatují několik parametrů
(vezmeme si za příklad souborové systémy vycházející z~operačního systému
\gls{UNIX}): vlastník a skupina, vlastnická, skupinová a ostatní práva, čas
poslední změny, velikost souboru a několik dalších parametrů.

Pokud uvažujeme soubory s~parametry, je potřeba odlišit změny pouze v~nich
(například změna vlastníka či skupiny) od změn na datech souboru. Parametry
souboru je nutné si přečíst vždy a tak změny na nich poznáme jednoduše, možností
jak poznat změnu na datech souboru je pak vícero:

\newacronym{MD5}{MD5}{Kryptografická hashovací funkce vytvářející z~libovolného vstupu výstup o~velikosti 128 bitů}
\newacronym{SHA1}{SHA1}{Secure Hash Algorithm -- pokročilejší kryptografická hashovací funkce, vytváří otisk o~velikosti 160 bitů}

\begin{itemize}
	\item Kompletní porovnání obsahu souboru po bytech -- zabere čas lineární
	s~velikostí souboru a stejnou přenosovou kapacitu, je ale neomylné.
	\item Kontrolní součet obsahu souboru -- je nutné stále spočítat kontrolní
	součet (třeba pomocí \gls{MD5} nebo \gls{SHA1}) souboru v~lineárním čase
	k~jeho velikosti, ale přenést mezi zálohovaným a zálohovacím strojem je
	potřeba pouze tento kontrolní součet (a porovnat ho s~uloženým). Při volbě
	dobré hashovací funkce se věří, že nalezení {\it kolize} (nalezení více
	různých souborů se stejným hashem) je extrémně nepravděpodovné a tedy
	spolehlivé.
	\item Porovnání jen parametrů (času poslední modifikace a velikosti) --
	je nejrychlejší (jde jen o~několik porovnání v~konstantním čase) a pro
	většinu situací je dostačující.
\end{itemize}

Důvodem, proč je poslední možnost považována za dostačující je ta, že je při
běžném provozu jen velmi málo situací, kdy se změní obsah souboru, ale nezmění
se čas jeho poslední modifikace a jeho velikost. Většinou to vyžaduje přímou
a~vědomou akci, kterou se čas modifikace souboru nastaví zpět na původní
hodnotu.

Situace, ve které je porovnání jen podle velikosti a času modifikace
nedostačující, je již nastíněné odhalování útoků na zálohovaný stroj. Pokud se
již útočníkovi povede získat kontrolu nad strojem, může snadno modifikovat
systémové soubory a potom změnit čas jejich modifikace nazpět. Pokud se před
takovou situací chceme chránit, je vhodné občas nechat zkontrolovat alespoň
kontrolní součty všech souborů oproti verzi uložené v~zálohovacím systému.

V~současnosti jedním z~nejvíce využívaných protokolů a současně i programů
implementující nějakým způsobem kopírování jen rozdílných souborů je {\bf rsync}
(viz \cite{rsync}). Ten potřebuje na jedné straně běžícího klienta a na druhé
straně běžící server. Ty si mezi sebou vyměňují různé parametry a kontrolní
součty souborů a jejich částí. Hlavním cílem protokolu je snížit objem síťové
komunikace na minimum a přenášet jen nezbytné množství dat (přenos navíc probíhá
komprimovaně).

Bohužel primárním cílem rsyncu je {\it synchronizace} dvou složek a není tak
úplně vhodný k~tomu držet si více historických verzí záloh -- to by pravděpodobně
znamenalo mít synchronizovanou složku pro každou historickou verzi a to rsync
neumí dělat efektivním způsobem. Neumí v~nových složkách synchronizovat jen
změny oproti úplně jiné složce, bylo by potřeba si vždy na zálohovacím stroji
zkopírovat celou starší zálohu do nové složky a pak synchronizovat až vůči ní.

Toto je obecně problém většiny synchronizačních protokolů a programů, takže se
většinou nedají samostatně použít k~účinnému zálohování. Je však možné je
využít jako efektivní přenosovou část zálohovacího systému.

\section{Rozdílové zálohování}

Pokud už přenášíme pouze rozdíly, nešel by udělat další krok a místo vždy celých
kopií zálohovaných dat pro každou historickou verzi si ukládat jen tyto rozdíly?
Tím bychom odstranili významný problém nastíněný výše, a to přesněji místo
potřebné k~uložení záloh (což by bez rozdílového zálohování znamenalo ukládat
velký datový objem: $[\hbox{počet záloh}]\times[\hbox{velikost dat}]$).

Pokud se na problém podíváme v~prvním přiblížení, stačí nám si jen na počátku
uložit plnou kopii dat a všechny další historické verze si ukládat jen jako
rozdíly vždy oproti té předchozí. Pro získání konkrétní verze dat nám pak stačí
jen na původní data aplikovat postupně všechny zapamatované rozdíly až do nějaké
verze.

Pro získání rozdílu textových dat (takový rozdíl se typicky nazývá anglickým
výrazem {\it diff} nebo u~binárních dat někdy {\it delta}) je možné použít mnoho
nástrojů porovnávajících zadané textová data řádek po řádku. Produkují pak
takzvaný {\it patch}, který obsahuje řádky, které ubyly, a řádky, které naopak
přibyly.

\newacronym{VCDIFF}{VCDIFF}{Formát a algoritmus definovaný v~RFC 3284 založený
na článku {\it Data Compression Using Long Common String}}

U~binárních dat je situace složitější a není tak jednoduché stanovit, čeho by
se měl binární diff držet a co by mělo být jeho výsledkem. Pro sjednocení mnoha
implementací vznikl formát \gls{VCDIFF}. Ten specifikuje tři různé instrukce:
pro přidání nové sekvence, pro zkopírování úseku ze staré sekvence a pro
opakování určité sekvence dat.

Dá se použít i pro kompresi (i když pro tyto účely existují lepší algoritmy a
formáty), ale hlavně se dá za sebe zřetězit stará a nová verze binárních dat,
ale kóduje se jen úsek odpovídající nové verzi dat. Díky tomu, že se ale může
odkazovat na sekvence v~původní verzi dat, vznikne tak vlastně jen požadovaný
binární diff.

\subsection*{Problémy reálné implementace}

V~reálné implementaci rozdílové zálohování musíme vyřešit ještě několik drobných
problémů. Prvním z~nich je to, že postupně se nabalující rozdíly způsobí, že
při delším provozu zálohovacího systému bude získání nejnovější verze dat trvat
neúměrně dlouho. Bude totiž nutné začít u~první verze a postupně aplikovat
všechny rozdíly -- to u~několika málo verzí ještě nepředstavuje problém, ale
u~několika desítek až stovek verzí to již začne velice zdržovat.

Řešením je omezit maximální hloubku odkazování a pokud by měla být nová verze
v~již moc velké hloubce, uložit ji buď jako kompletní verzi (tedy by se z~ní
stala nová \uv{nultá hladina}), nebo jako diff oproti vhodné verzi v~menší
hloubce. Zde se dá uplatnit heuristika vybírající nejvhodnější předchozí verzi
(například s~nejmenším diffem).

Druhým problémem, před který se musí reálná implementace postavit, je možnost
mazat historické zálohy. Dokud byla každá historická verze nezávislá na
ostatních, tak nás její smazání nic nestálo a nijak neovlivnilo okolní verze.
U~na sebe navazujících rozdílů si takové prosté smazání nemůžeme dovolit,
protože bychom tím znehodnotili všechny verze navazující v~posloupnosti rozdílů
na mazanou verzi.

Pokud tedy chceme mazat historické verze (a to je u~dlouhodobě fungujícího
zálohovacího systému nezbytné), je potřeba implementovat řešení nějakým
způsobem distribuující mazané změny do navazujících verzí, neboli přepočítávat
navazující verze, aby zahrnovaly i mazané změny.

\section{Automatizace}

Výše popsané způsoby se dají snadno zautomatizovat. Pokud nevyžadují uživatelovo
aktivní rozhodování během zálohování, lze naplánovat jejich pravidelné spouštění
v~naplánovaných intervalech.

Pro tuto automatizaci lze využít mnoho různých postupů. Buď může zálohovací
systém stále běžet jako systémová služba (démon) a hlídat si čas plánovaného
zálohování sám, nebo lze využít například v~\gls{UNIX}u dostupný {\it cron} na
naplánování spouštění přednastaveného příkazu.

\section{Existující zálohovací systémy}

TODO... Bacula etc.
