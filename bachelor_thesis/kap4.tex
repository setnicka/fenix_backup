\chapter{Implementace zálohovacího systému FenixBackup}

Zálohovací systém je napsaný v~jazyce \Cpp, konkrétně podle specifikace
\Cpp11. Vyvíjen byl na platformě operačního systému Linux a je optimalizován
pro jeho systém práv a parametrů souborového systému, ale měl by fungovat na
všech hlavních platformách.

\section{Obecné návrhové vzory}

\newacronym{PIMPL}{PIMPL}{Návrhový idiom \uv{private implementation} pokoušející
se skrýt co nejvíce implementace před vnějším světem}

Při implementaci jsem se držel snahy mít v~hlavičkových souborech tříd co možná
nejméně položek nepotřebných jako rozhraní pro komunikaci s~ostatními třídami.
Většina privátních proměnných a funkcí je tedy (podle \gls{PIMPL} idiomu)
zapouzdřena uvnitř vnořených tříd

Důvodem pro tuto snahu je za prvé přehlednost hlavičkových souborů a za druhé
odstranění potřeby překompilování všech souborů, které daný hlavičkový soubor
includují při změně v~privátní metodě.

Dále se v~celém systému hojně používají chytré pointery zavedené v~\Cpp11,
které na mnoha místech zjednodušují zacházení s~instancemi tříd, pokud si na ně
potřebujeme držet odkaz z~více míst zároveň.

Vše bydlí ve jmenném prostoru {\it FenixBackup}.

\section{Přehled tříd a jejich funkce}

V~minulé části jsme si nastínili schéma stromečků reprezentujících jednotlivé
zálohy a držící informace o~souborech, a schéma datových bloků.

Souborové stromečky jsou představovány třídou {\it FileTree} a obsahují většinu
logiky související se zálohováním souborů. V~jednotlivých jeho uzlech pak bydlí
instance třídy {\it FileInfo}, jež si drží informace vždy o~jedné složce,
normálním souboru nebo symlinku. V~případě složky si pamatují pointery na v~ní
obsažené soubory, jinak si pamatují hash obsahu.

Hashe obsahu odkazují na instance třídy {\it FileChunk}, které již drží finální
data. To je vždy \gls{VCDIFF} oproti prázdnému souboru, nebo jinému souboru
reprezentovanému instancí třídy {\it FileChunk}.

\subsection{FileTree}

Hlavní logika jádra zálohovacího systému je soustředěna do třídy {\it FileTree},
jež obsahuje tři metody pro registraci položek do souborového stromu:

\begin{itemize}
	\item\texttt{AddDirectory(rodic, jmeno\_souboru, parametry)}
	\item\texttt{AddFile(rodic, jmeno\_souboru, parametry)}
	\item\texttt{AddSymlink(rodic, jmeno\_souboru, parametry)}
\end{itemize}

Každá z~těchto metod vrací odkaz na nově přidanou položku (který se pak třeba
v~případě složky dá využít pro přidání položek do ní), nebo \texttt{nullptr},
pokud daná věc nemá být uložena ve stromě souborů (a tedy nemá například smysl
procházet podstrom souborů pod touto složkou).

Dále obsahuje třída {\it FileTree} metodu pro vrácení seznamu souborů, které si
zálohovací systém přeje dodat (setříděného podle důležitosti) a několik
pomocných metod pro uložení stromu, vrácení kořenu celého stromu kvůli předání
první vrstvě položek jako rodiče aj.):

\begin{itemize}
	\item\texttt{FinishTree()} $\rightarrow$ vektor odkazů na položky stromu
	\item\texttt{GetRoot()} $\rightarrow$ odkaz na {\it FileInfo}
	reprezentující kořen
\end{itemize}

Nakonec třída obsahuje statické metody pro načtení stromu konkrétního jména
z~úložiště (ta vrací odkaz na strom nebo \texttt{nullptr}) a pro vrácení seznamu
všech stromů v~úložišti.

\subsection{FileInfo}

Hlavním úkolem této třídy je držet si informace o~jednom souboru, složce či
symlinku, o~kterých si ve všech případech drží parametry jako jsou práva, čas
poslední modifikace a podobně. V~případě složky navíc obsahuje vektor obsažených
souborů, v~ostatních případech pak hash odkazující na příslušný datový blok.

Mezi základní metody patří metody pro zpracování obsahu souboru nebo jeho
vrácení. Jako nepovinný parametr mohou dostat odkaz na aktuálně konstruovaný
strom souborů a pomocí něj vyhledávat soubory se stejným hashem v~minulé verzi
stromu:

\begin{itemize}
	\item\texttt{ProcessFileContent(inputstream, strom = nullptr)}
	\item\texttt{GetFileContent(outputstream)} $\rightarrow$
	\texttt{outputstream} (vrácení umožňuje použití v~syntaxi \texttt{<\relax<} streamů)
\end{itemize}

Další důležité metody jsou následující tři sloužící pro správu obsahu složek.
Zbytek veřejných metod jsou pak buď settery nebo gettery na vlastnosti souboru:

\begin{itemize}
	\item\texttt{AddChild(jmeno, odkaz\_na\_instanci\_FileInfo)}
	\item\texttt{GetChild(jmeno)} $\rightarrow$ odkaz na instanci {\it FileInfo}
	\item\texttt{GetChilds()} $\rightarrow$ odkaz na asociativní pole odkazů
	na {\it FileInfo}
\end{itemize}

\subsection{BackupClean}

Tato třída se stará o mazání starých verzí záloh. Obsahuje dvě metody, z nichž
první si načte do paměti všechny souborové stromy, analyzuje je a spočítá
badness (viz kapitola tomu věnovaná) a druhá metoda vždy obstará smazání jednoho
datového bloku a vrátí, kolik dat bylo celkově uvolněno:
\begin{itemize}
	\item\texttt{LoadData()}
	\item\texttt{Clean()} $\rightarrow$ uvolněné místo (v bytech)
\end{itemize}

\subsection{Ostatní třídy}

Třída {\it FileChunk} představuje jednotlivé datové bloky. Každá instance třídy
si záznam, jestli je odvozená jako rozdíl od jiné instance, nebo jestli je
takzvaně na \uv{nulté úrovni} a na ničme nezávisí.

Dále existuje třída představující konfiguraci {\it Config}, která vrací
jednotlivé položky načtené konfigurace a pak řeší dotazy na pravidla pro
jednotlivé zálohované soubory (opět podle načtené konfigurace).

Pro komunikaci s~adaptéry je pak definována abstraktní třída {\it Adapter}
obsahující metody pro proskenování a vytvoření souborového stromu, pro
zpracování obsahu souboru a pro obnovení souboru na původní, nebo na zadané
místo.

A konečně, jako uživatelské rozhraní existuje třída {\it CLI} (zkratka od
{\it command-line-interface}).

\section{Externí knihovny}

Zálohovací systém využívá několik externích knihoven a to pro zpracování
konfiguračních souborů, pro serializaci dat, pro počítání \gls{SHA256} hashů
\gls{VCDIFF} rozdílů souborů a pro práci se souborovým systémem.

Důvody pro použití externích knihoven jsou hlavně dva: Pokud existuje fungující
a efektivní řešení daného problému, není potřeba \uv{znovu vynalézat kolo}, a za
druhé, na mnohá z~těchto řešení mohou být uživatelé zvyklí (například konfigurace)
a použití známých komponent usnadní uživatelům používání.

\subsection*{Konfigurace -- libconfig}

Byla použita mezi mnoha projekty rozšířená a přizpůsobivá knihovna
\texttt{libconfig}\footnote{\url{http://www.hyperrealm.com/libconfig/}}. Důvodem
pro její volbu je velká rozšířenost mezi C a \Cpp projekty a tedy její známost
mezi uživateli. Program potřebuje být slinkován s~knihovnou
\texttt{libconfig++}.

\subsection*{Serializace dat -- Cereal}

Pro serializaci instancí tříd {\it FileTree} (s~navázanými instancemi třídy
{\it FileInfo}) a~{\it FileChunk} se používá serializační knihovna {\it Cereal}%
\footnote{\url{http://uscilab.github.io/cereal/index.html}}, což je projekt
vzniklý cíleně pro \Cpp11 a novější využívající vlastností chytrých
pointerů pro serializaci složitých datových struktur.

Sídlí čistě v~hlavičkových souborech přiložených společně s~projektem, takže
není nutné cokoliv dynamicky linkovat.

\subsection*{Hashe a rozdíly -- sha256.h a open-vcdiff}

Pro počítání \gls{SHA256} je použita část hashovací knihovny, kterou napsal
Stephan Brumme\footnote{\url{http://create.stephan-brumme.com/hash-library/}},
sídlí jen v~hlavičkových souborech, tedy se opět nic dynamicky nelinkuje.

Pro vytváření a zpracování binárních rozdílů souborů je použita\gls{VCDIFF}
implementace \texttt{open-vcdiff}\footnote{\url{https://code.google.com/p/open-vcdiff/}},
která vyžaduje slinkování s~externími knihovnami \texttt{libvcdcom}, \texttt{libvcdenc}
a \texttt{libvcddec}.

\subsection*{Souborový systém -- boost::filesystem}

Na místech interagujících nějakým složitějším způsobem se souborovým systémem
byla použita implementace souborového systému z~\Cpp projektu \texttt{boost}%
\footnote{\url{http://www.boost.org/doc/libs/1_46_0/libs/filesystem/v3/doc/index.htm}}.
Poskytuje na platformě nezávislé rozhraní a při případném použití zálohovacího
systému na jiné platformě, než na které byl vyvinut, by tak neměla nastat žádné
významné problémy. Pro běh musí být program slinkován s~knihovnami
\texttt{libboost\_system} a \texttt{libboost\_filesystem}.

\section{Ukládání dat}

Souborové stromy a datové bloky jsou ukládány na oddělená místa. Každý souborový
strom, stejně jako popis datového bloku, sídlí v~samostatném souboru na disku.
Soubory jsou ukládány v~binárním formátu. Ke každé serializované instanci třídy
{\it FileChunk} je ještě připojen blok dat ve formátu \gls{VCDIFF} popisující
rozdíl oproti prázdnému souboru, nebo oproti nějakému jinému uloženému datovému
bloku.

Formát uložení dat určuje serializační knihovna {\it Cereal} (která podporuje
více druhů serializačních archivů, zálohovací systém využívá binárního formátu
zápisu dat). Formát dat je explicitně verzovaný, což dovoluje do budoucna
přidávat či ubírat položky se zachováním zpětné kompatibility.

Cereal dovoluje pojmenovávat ukládané položky, což se využívá při výstupu
v~lidsky čitelných formátech typu JSON nebo XML (toho bylo využíváno při vývoji),
ale v~binárním formátu jsou za sebe serializovaná jen samotná binární data. Pokud
je serializován chytrý pointer na objektu, je nejdříve uveden čtyřbytový
identifikátor typu objektu a pak buď serializovaná data objektu, nebo čtyřbytové
pořadové číslo již použitého chytrého pointeru na tento typ objektu.

Každému archivu předchází hlavička a čtyřbytové číslo verze, níže následují
seřazené tabulky položek pro obě serializované třídy. S~jejich znalostí je možné
snadno pomoci jakékoliv verze knihovny Cereal získat původní data i bez
zálohovacího systému

\subsection{Formát dat třídy FileTree}

\begin{tabular}{l >{\tt}l}
\bf Položka & \bf datový typ \\
\hline
Název stromu & std::string \\
Čas vytvoření & time\_t \\
Název předchozího stromu & std::string \\
SHA256 hash předchozího stromu & std::string \\
Odkaz na kořen stromu & std::shared\_ptr<FileInfo> \\
\end{tabular}

\subsection{Formát dat třídy FileInfo}

\begin{tabular}{l >{\tt}p{7cm}}
\bf Položka & \bf datový typ \\
\hline
Jméno souboru & std::string \\
Typ souboru & enum\{DIR, FILE, SYMLINK\}\\
Stav verzování & enum\{UNKNOWN, NEW, UNCHANGED, UPDATED\_PARAMS, UPDATED\_FILE, NOT\_UPDATED, DELETED\} \\
Parametry & struct file\_params \\
Index souboru & uint32\_t \\
Index minulé verze souboru & uint32\_t \\
Odkaz na rodiče & std::shared\_ptr<FileInfo> \\
Hash souboru & std::string \\
Asociativní pole synů & std::unordered\_map<std::string, std::shared\_ptr<FileInfo>\relax> \\
\end{tabular}

\subsection{Formát dat struktury file\_params}

\begin{tabular}{l >{\tt}l}
\bf Položka & \bf datový typ \\
\hline
Číslo zařízení & dev\_t \\
Číslo inode & ino\_t \\
Souborová práva & mode\_t \\
UID vlastníka & uid\_t \\
GID skupiny & gid\_t \\
Velikost souboru & size\_t \\
Čas modifikace (sekundy) & timespec.tv\_sec \\
Čas modifikace (nanosekundy) & timespec.tv\_nsec \\
\end{tabular}

\subsection{Formát dat třídy FileChunk}

\begin{tabular}{l >{\tt}l}
\bf Položka & \bf datový typ \\
\hline
Název (hash) & std::string \\
Název (hash) předchůdce & std::string \\
Hloubka zanoření & int \\
Velikost dat & size\_t \\
Závislé instance & std::vector<std::string> \\
\end{tabular}

\section{Postup uložení nové verze souboru}

Prvním krokem je vždy obstarání si stromu souborů, neboli instance třídy
{\it FileTree}. Většinou si vyrobíme aktuální ze současného stavu zálohovaných
dat pomocí vhodného adaptéru, ale je možné vzít i starší instanci třídy
{\it FileTree} a modifikovat tu (je pak ale nutné počítat s~tím, že nebudou
souhlasit kontrolní součty v~navazujících instancích {\it FileTree}).

\newpage

\subsection{Záznam do stromu souborů a hledání minulé verze}

Pokud si strom souborů vyrábíme, tak v~nějakou chvíli dojde k~přidání námi
sledovaného souboru (voláním jedné z~metod třídy {\it FileTree}). Přitom proběhne
následující posloupnost operací:
\begin{enumerate}
	\item Obstarají se pravidla pro tento soubor (od třídy {\it Config})
	a případně se ukončí zpracování, pokud se soubor nemá zálohovat.
	\item Vytvoří se nová instance {\it FileInfo} a přidá se odkaz do rodiče.
	\item Uloží se známé parametry souboru (vlastník, velikost, čas modifikace)
	\item Pokud je znám minulý strom: Systém se pokusí nalézt soubor se
	stejnou cestou (dotazem na svého rodiče, jestli zná svůj starší
	ekvivalent, a z~něj jedním dotazem na jméno souboru). Pokud není nalezen,
	je nastaven stav \texttt{UNKNOWN} a zpracování se ukončí.
	\item Pokud nebyla poslední verze zpracována, i když se změnila (status
	\texttt{UNKNOWN}, \texttt{NOT\_UPDATED} nebo \texttt{DELETED}), nastav
	stejný stav (mimo \texttt{DELETED}, v~tom případě nastav
	\texttt{NOT\_UPDATED}) a skonči.
	\item Pokud je nalezený soubor stejného typu, velikosti, času modifikace
	a parametrů $\rightarrow$ status \texttt{UNCHANGED} a přiřadí se stejný hash
	(odkazující na stejný datový blok).
	\item Pokud je nalezený soubor stejného typu, velikosti, času modifikace,
	ale liší se v~parametrech $\rightarrow$ status \texttt{UPDATED\_STATUS}
	a přiřadí se stejný hash (odkazující na stejný datový blok).
	\item Jinak přiřaď status \texttt{NOT\_UPDATED}.
\end{enumerate}

Pokud soubor skončí ve stavu \texttt{NOT\_UPDATED} nebo \texttt{UNKNOWN}, přidá
se do seznamu souborů k~obstarání a spočítá se pro něj skóre důležitosti, jak
již bylo ukázáno.

\subsection{Zpracování obsahu souboru}

Pokud se zálohovací systém rozhodne, že má zájem o~obsah souboru, tak ho adaptér
nějakým způsobem obstará a poté voláním metody na třídě {\it FileInfo} předá
tomuto záznamu obsah souboru.

Nedjříve se spočítá \gls{SHA256} hash a pokud není známá minulá verze souboru,
porovná se, jestli neexistuje v~minulém stromě soubor se stejným hashem. Pokud
ano, přiřadí se jako minulá verze, k~souboru se uloží hash a skončí se, protože
není nutné obsah dále zpracovávat (takto systém pozná přesuny souborů).

Pokud toto neuspěje, je potřeba obsah souboru uložit. Nedjříve systém prozkoumá,
jestli již nemá uložen soubor se stejným hashem (tedy pokud zachováme důvěru
v~kryptografickou hashovací funkci, tak i se stejným obsahem) a pokud ano, není
nutné nic nového ukládat a použijeme tento uložený datový blok.

Pokud stejný datový blok neexistuje, tak systém vytvoří novou instanci třídy
{\it FileChunk} reprezentující soubor s~tímto hashem a nechá ji zpracovat obsah.
Pokud je znám předek, tak se zkusí vytvořit \gls{VCDIFF} oproti poslední verzi
(získají se data této poslední verze a udělá se rozdíl mezi nimi), jinak se dělá
rozdíl oproti prázdnému souboru.

Pokud by byl při vytváření rozdílu vůči starší verzi překročen limit na maximální
hloubku na sebe navazujících rozdílů, tak se vytvoří rozdíl oproti první známé
verzi souboru (tedy oproti kořeni, který sám vznikl rozdílem oproti prázdnému
souboru), čímž se opět dosáhne nižší hloubky.

Alternativou by bylo vytvářet novou kompletní kopii takového souboru, ale rozdíl
oproti nějaké základní verzi zabere nejvýše tolik místa, kolik by zabral rozdíl
oproti prázdnému souboru. Takže toto řešení je nejméně stejně dobré, jako řešení
s~vyráběním nových záznamů s~o~jedna nižší maximální hloubkou.

\section{Obnova dat}

Při obnově dat se dá zvolit několik strategií. Vždy se zvolí konkrétní záloha
(konkrétní strom), ze kterého nás zajímají daná data, a jádro zálohovacího
systému pak poskytuje podporu pro:
\begin{itemize}
	\item Obnovu jednoho konkrétního souboru nebo celého podstromu
	\item Obnovu do původního místa, do jiného místa na původním (vzdáleném)
	stroji a do jiného místa v~lokálním souborovém systému
	\item Obnovu pouze dat souboru, pouze parametrů souboru, nebo všeho
\end{itemize}

Navíc se dá specifikovat taktika, co se má dělat, pokud v~aktuálním stromu není
soubor dostupný, určuje se podle dalšího parametru předávaného obnovovacím
funkcím. Výchozí chování je to, že v~takovém případě se pokusí systém nalézt
nejnovější zálohovanou verzi souboru a obnoví tu (zde si můžeme všimnout, že
starší verzi má smysl hledat jen u~souborů označených jako
\texttt{NOT\_UPDATED}, u~souborů označených \texttt{UNKNOWN} ne). Alternativním
chováním je obnova pouze těch dat, která jsou odkazovaná aktuálním stromem
(neuložené soubory jsou přeskakovány).

Při obnovování je navíc potřeba dát pozor na hardlinky vedoucí mimo obnovovanou
oblast. V~takovém případě by se mohlo stát, že se obnovením nějakého souboru
přes hardlink projeví změna i na úplně jiném místě, což pravděpodobně není
požadovaný cíl. Zálohovací systém tedy přijal taktiku nejdříve obnovovaný soubor
celý smazat (tím se zruší případné provázání hardlinkem) a pak na stejném místě
vytvořit znovu.

\subsection{Postup získání obsahu souboru}

Obnova jednoho souboru začíná dotazem na jeho typ -- pokud to je složka, tak
jsou všechny potřebné informace uložené již ve stromě souborů a není nutné
získávat jakákoliv data z~datových bloků. Pokud to není složka, tak se přes
třídu {\it FileInfo} dostane žádost o~obnovení dat až na datový blok
reprezentující danou verzi souboru.

Pokud se datový blok (uložený ve formátu \gls{VCDIFF}) neodkazuje na žádný jiný,
je jeho obsah přímo obnoven jako diff od prázdného souboru a vrácen. Pokud se
odkazuje na jiný datový blok, je nejdříve dotazem na tento blok (rekurzivně)
získán jeho obsah, proti kterému je pak aplikováním rozdílu vyroben obsah tohoto
datového bloku a ten je vrácen.

\section{Mazání uložených dat -- uvolňování místa}

Zálohovací systém se může čas od času potkat s~nutností uvolnit nějaké místo.
K~uvolnění místa je nutné zahodit některé informace, které si zálohovací systém
pamatuje, a když už se to provádí, měl by systém nějakým vhodným způsobem zvolit
podle něj nejméně důležitá data a ta zahodit.

Toliko abstraktní představa, teď konkrétněji. Uvolnit místo se dá smazáním
některých datových bloků a úkolem mazání je vybrat ten blok, který bude
\uv{nejméně scházet}. Na co se nesmí zapomenout je to, že na jeden datový blok
se může odkazovat více záznamů o~souborech.

Vybrání bloku ke smazání probíhá tak, že se pro každý záznam o~zálohovaném
souboru spočítá {\it badness} (\uv{míra špatnosti}), která se odvíjí mimo jiné
i od stáří zálohy a jejich nahuštěnosti -- je větší snaha držet novější zálohy
a více do historie může pokrytí řídnout (přesná funkce viz níže).

Z~takto spočítaných badness pro každý záznam o~souboru se pro každý datový
blok vybere ta nejmenší badness ze souborů, které daný blok využívají, čímž se
efektivně ohodnotí bloky podle toho, jak moc jsou významné pro historii. Pak se
mohou bloky postupně od největší badness vybírat a mazat. Přitom je nutné dát
pozor na to, že při smazání nějakého bloku (a tím několik verzí souborů) se
může změnit badness okolním souborům v~historii a ty se musí přepočítat. Dá se
však vypozorovat, že se badness vždy jenom sníží, čehož se dá využít při
praktickém nasazení ke zjednodušení implementace.

\subsection{Postup počítání badness}

Čistící třída {\it BackupClean} obsahuje jednu metodu, jejímž voláním se načte
kompletní seznam všech záloh a obsažených souborů, a druhou metodu sloužící ke
smazání vždy jednoho datového bloku s~největší badness. Postup je tedy takový,
že se jednou načte kompletní seznam a pak se opakovaně volá mazání bloků.

Při načítání se nejprve vytvoří pro každou známou navazující posloupnost souborů
jeden jejich vektor (aby bylo možné snadno nalézt o~jedna novější a o~jedna
starší verzi souboru). Pak se pro každý záznam o~souboru spočítá badness podle
vzorce:

$$B = {[\hbox{stáří zálohy}]\over[\hbox{vzdálenost nejbližší starší zálohy}]}
+{[\hbox{stáří zálohy}]\over[\hbox{vzdálenost nejbližší mladší zálohy}]}$$
$$\hbox{badness} = {B\over[\hbox{hodnota \texttt{history} z konfigurace}]}$$

Tento vzorec vlastně znamená, že čím starší záloha, tím větší volné místo okolo
sebe vyžaduje pro zachování stejně badness, což je přesně to, co jsme chtěli.

Ze spočítaných badness se vybere pro každý datový blok ta nejmenší a pak se
tato hodnota ještě vydělí pro každý datový blok počtem na něj navazujících bloků
(to je heuristika beroucí v~úvahu to, že při vícero navazujících blocích by se
namísto zmenšení mohl celkový datový objem zvětšit, viz dále).

Pak se datové bloky utřídí od největší hodnoty. Po smazání datového bloku se
provede označení všech spojených záznamů o~souborech jako \texttt{DELETED} a
všem okolním záznamům se přepočítá badness. Pak je systém připraven pro mazání
dalšího bloku.

\subsection{Smazání datového bloku}

Datový blok nelze odstranit jen tak jednoduše, protože pokud by od něj byly
odvozeny jiné datové bloky, tak by se tímto znehodnotily. Proto je nutné
nejdříve všechny navazující datové bloky nechat přepočítat, aby se odkazovaly
na předka mazaného bloku, a teprve poté je možné blok odstranit.

Přepočítání proběhne tak, že se provede obnovení dat předka mazaného souboru a následníků mazaného souboru a spočítá se nový \gls{VCDIFF} mezi nimi. Tím se
změny pokrývané tímto blokem rozředí do změn v~dalších blocích.

V~některých případech se může stát, že se tímto krokem celkový objem dat dokonce
zvětší, protože se velký rozdíl pokrývaný tímto souborem odsune do více
navazujících souborů. Proto je součástí výpočtu badness datových bloků také
dělení počtem navazujících bloků
