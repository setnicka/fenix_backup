\chapter{Uživatelská dokumentace}

FenixBackup se v~současnosti ovládá z~příkazové řádky, ale do budoucna není
problém nad samotným jádrem postavit více jinak fungujících rozhraní (třeba
GUI). Pro jeho fungování je nezbytný konfigurační soubor, který popisuje co,
jakým způsobem a kam se má zálohovat.

\section{Konfigurační soubor}

Konfigurační soubor se vztahuje vždy k~jednomu zálohovanému stroji a obsahuje
tři části. První specifikuje úložiště zálohovaných dat, druhá z~nich popisuje
způsob připojení k~zálohovanému stroji -- volba adaptéru a parametry pro něj
(například cesta v~rámci lokálního souborového systému, nebo adresu a port
stroje, na který se připojit) a poslední a nejobsáhlejší nakonec popisuje
zálohovací pravidla pro jednotlivé soubory.

Při spuštění zálohovacího systému je nejdříve načtena celá konfigurace a teprve,
když se jí povede celou zpracovat bez chyby, přistupuje se k~provádění příkazů.

\subsection{Ukázková konfigurace}

Konfigurace pro zápis používá v~mnoha projektech rozšířený zápis poskytovaný
knihovnou {\it libconfig}.%
\footnote{\url{http://www.hyperrealm.com/libconfig/libconfig_manual.html}}

Protože je velmi časté používat společná zálohovací pravidla pro více strojů,
dají se tato společná pravidla sdružit do společných souborů a ty pomocí
direktivy \texttt{@include} v~konfiguraci používat opakovaně na více místech.

Ukázka je připojena níže.

\begin{verbatim}
baseDir: "./backup_dir"
dataSubdir: "data"
treeSubdir" "trees"

adapter: {
  type: "local_filesystem"
  path: "/home/jirka/fenix_backup"
}

paths: (
  { path: "/"
    file_rules: (
      @include "no_exe.config"
    )
  },
  {
    path: "/.git"
    scan: false
  },
  { path: "/test_backup/"
    scan: true
    file_rules: (
      { backup: false },
      { regex: ".*\.fenixtree"
        size_at_least: 10
        size_at_most: 10485760
        backup: true
      }
    )
  }
)

# Obsah souboru no_exe.config:
{ regex: ".*\.exe"
  backup: false
}
\end{verbatim}

První položka \texttt{baseDir} je povinná a popisuje, kam přesně má zálohovací
systém ukládat svá data. Pak lze volitelně modifikovat názvy podsložek na
ukládání dat souborů a souborových stromečků (\texttt{dataSubdir}
a \texttt{treeSubdir}).

Druhá položka \texttt{adapter} specifikuje způsob získávání zálohovaných dat.
V~první řadě je specifikováno, který adaptér se má použít, a zbytek položek je
pak předán tomuto adaptéru jako jeho nastavení.

Poslední části konfigurace popisuje pravidla pro cesty a specifičtější pravidla
určovaná filtry, jejichž fungování jsme rozebrali v~minulé kapitole.

\section{Použití z~příkazové řádky}

FenixBackup je psaný primárně pro spouštění příkazů z~příkazové řádky s~cílem
umožnit snadné plánování akcí třeba pomocí \texttt{cron}u.

Základem je spuštění příkazu \texttt{fenix}, kterému se vždy předá cesta ke
konfiguračnímu souboru a poté příkaz, který chceme provést. Inspirací pro
rozhraní bylo zčásti rozhraní verzovacího systému Git (viz \cite{progit}).

Pomocí tohoto rozhraní lze provádět všechny potřebné akce:
\begin{itemize}
	\item Zobrazit seznam záloh
	\item Zobrazit seznam souborů v záloze
	\item Sledovat historii jednoho konkrétního souboru
	\item Provést zálohu nebo čištění
	\item Obnovit jeden soubor, podstrom nebo všechny soubory z určené zálohy
	buď do původního místa, nebo do specifikované cesty
\end{itemize}

\newpage
\subsection{Popis příkazů}

Přesné použití rozhraní nejlépe popisuje usage-note příkazu \texttt{fenix}:
\begin{verbatim}
Usage: ./fenix <config_file>
And one of these commands:
  show backups			(displays list of all backups)
  show files [<backup>]		(displays list of all files in given backup)
  show history <backup> <path>	(displays known history of given file)
  backup			(run backup)
  restore full <backup>		(run full restore to original path)
  restore full <backup> <path>	(run full restore to given path)
  restore subtree <backup> <subtree_path>
				(restore subtree to original path)
  restore subtree <backup> <subtree_path> <path>
				(restore subtree to given path)
  restore file <backup> <file_path>
				(restore one file to original path)
  restore file <backup> <file_path> <path>
				(restore one file to given path)
  cleanup [<x>]			(run <x> rounds of cleanup, default 1)
\end{verbatim}
