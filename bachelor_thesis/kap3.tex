\chapter{Implementace zálohovacího systému FenixBackup}

Zálohovací systém je napsaný v jazyce C{\tt++}, konkrétně podle specifikace
C{\tt++}11. Vyvíjen byl na platformě operačního systému Linux a je optimalizován
pro jeho systém práv a parametrů souborového systému, ale měl by fungovat na
všech hlavních platformách.

\section{Obecné návrhové vzory}

\newacronym{PIMPL}{PIMPL}{Návrhový idiom \uv{private implementation} pokoušející
se skrýt co nejvíce implementace před vnějším světem}

Při implementaci jsem se držel snahy mít v hlavičkových souborech tříd co možná
nejméně položek nepotřebných jako rozhraní pro komunikaci s ostatními třídami.
Většina privátních proměnných a funkcí je tedy (podle \gls{PIMPL} idiomu)
zapouzdřena uvnitř vnořených tříd

Důvodem pro tuto snahu je za prvé přehlednost hlavičkových souborů a za druhé
odstranění potřeby překompilování všech souborů, které daný hlavičkový soubor
includují při změně v privátní metodě.

Dále se v celém systému hojně používají chytré pointery zavedené v C{\tt ++}11,
které na mnoha místech zjednodušují zacházení s instancemi tříd, pokud si na ně
potřebujeme držet odkaz z více míst zároveň.

Vše bydlí ve jmenném prostoru {\it FenixBackup}.

\section{Přehled tříd a jejich funkce}

V minulé části jsme si nastínili schéma stromečků reprezentujících jednotlivé
zálohy a držící informace o souborech, a schéma datových bloků.

Souborové stromečky jsou představovány třídou {\it FileTree} a obsahují většinu
logiky související se zálohováním souborů. V jednotlivých jeho uzlech pak bydlí
instance třídy {\it FileInfo}, jež si drží informace vždy o jedné složce,
normálním souboru nebo symlinku. V případě složky si pamatují pointery na v ní
obsažené soubory, jinak si pamatují hash obsahu.

Hashe obsahu odkazují na instance třídy {\it FileChunk}, které již drží finální
data. To je vždy \gls{VCDIFF} oproti prázdnému souboru, nebo jinému souboru
reprezentovanému instancí třídy {\it FileChunk}.

\subsection{FileTree}

Hlavní logika jádra zálohovacího systému je soustředěna do třídy {\it FileTree},
jež obsahuje tři metody pro registraci položek do souborového stromu:

\begin{itemize}
	\item\texttt{AddDirectory(rodic, jmeno\_souboru, parametry)}
	\item\texttt{AddFile(rodic, jmeno\_souboru, parametry)}
	\item\texttt{AddSymlink(rodic, jmeno\_souboru, parametry)}
\end{itemize}

Každá z těchto metod vrací odkaz na nově přidanou položku (který se pak třeba
v případě složky dá využít pro přidání položek do ní), nebo \texttt{nullptr},
pokud daná věc nemá být uložena ve stromě souborů (a tedy nemá například smysl
procházet podstrom souborů pod touto složkou).

Dále obsahuje třída {\it FileTree} metodu pro vrácení seznamu souborů, které si
zálohovací systém přeje dodat, a sadu dvou metod pro uložení nového obsahu
položky (volá adaptér pokaždé, když se mu povede získat obsah nějakého souboru)
a pro vrácení obsahu uloženého souboru:

\begin{itemize}
	\item\texttt{FinishTree()} $\rightarrow$ vektor odkazů na položky stromu
	\item\texttt{ProcessFileContent(polozka\_stromu, inputstream)}
	\item\texttt{GetFileContent(polozka\_stromu, outputstream)}
\end{itemize}

Výše vyjmenované metody jsou hlavními komunikačními prostředky jádra
zálohovacího systému. Dále třída {\it FileTree} obsahuje statické metody pro
načtení konkrétního stromu souborů reprezentujícího jednu zálohu z úložiště a
několik pomocných metod (uložení stromu, nalezení položky podle cesty nebo hashe
obsahu, vrácení kořenu celého stromu kvůli předání první vrstvě položek jako
rodiče aj.).

\subsection{FileInfo}

Hlavním úkolem této třídy je držet si informace o jednom souboru, složce či
symlinku. V případě složky obsahuje vektor obsažených souborů, v ostatních
případech obsahuje hash odkazující na příslušný datový blok. Ve všech případech
si drží parametry.

Mezi základní metody patří následující tři, zbytek veřejných metod jsou pak buď
settery nebo gettery na vlastnosti souboru:

\begin{itemize}
	\item\texttt{AddChild(jmeno, odkaz\_na\_instanci\_FileInfo)}
	\item\texttt{GetChild(jmeno)} $\rightarrow$ odkaz na instanci {\it FileInfo}
	\item\texttt{GetChilds()} $\rightarrow$ odkaz na asociativní pole odkazů
	na {\it FileInfo}
\end{itemize}

\subsection{Ostatní třídy}

Třída {\it FileChunk} představuje jednotlivé datové bloky. Každá instance třídy
si záznam, jestli je odvozená jako rozdíl od jiné instance, nebo jestli je
takzvaně na \uv{nulté úrovni} a na ničme nezávisí.

Dále existuje třída představující konfiguraci {\it Config}, která vrací
jednotlivé položky načtené konfigurace a pak řeší dotazy na pravidla pro
jednotlivé zálohované soubory (opět podle načtené konfigurace).

\section{Ukládání dat}

Souborové stromy a datové bloky jsou ukládány na oddělená místa. Každý souborový
strom, stejně jako každý datový blok, sídlí v samostatném souboru na disku.
Soubory jsou ukládány v binárním formátu.

Pro serializaci instancí tříd {\it FileTree} (s navázanými instancemi třídy
{\it FileInfo}) a {\it FileChunk} se používá serializační knihovna {\it Cereal}%
\footnote{\texttt{http://uscilab.github.io/cereal/index.html}}, což je projekt
vzniklý cíleně pro C{\tt ++}11 a novější využívající vlastností chytrých
pointerů pro serializaci složitých datových struktur.

Ke každé serializované instanci třídy {\it FileChunk} je ještě připojen blok
dat ve formátu \gls{VCDIFF} popisující rozdíl oproti prázdnému souboru, nebo
oproti nějakému jinému uloženému datovému bloku.

\subsection{Postup uložení nové verze souboru}

\section{Obnova dat}

TODO (nejdříve smazat a pak až obnovit - hardlinky)

\section{Mazání uložených dat -- uvolňování místa}
