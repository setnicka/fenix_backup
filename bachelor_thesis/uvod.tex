\chapter*{Úvod}
\addcontentsline{toc}{chapter}{Úvod}

Potřeba zálohovat data se pojí s~počítačovým světem prakticky od jeho vzniku
a vychází z~dob ještě před jeho existencí -- z~potřeby udržovat kopie například
důležitých dokumentů a chránit je tak před náhodným nebo cíleným zničením nebo
ztrátou.

V~dřívějších dobách stačilo pořídit fyzickou kopii důležitého dokumentu a ten
uložit ideálně na jiném fyzickém místě, s~nástupem počítačového ukládání dat se
k~tomu však ještě přidal aspekt toho, že data je možná snadno editovat a měnit.

Pokud tedy v~současné době mluvíme o~zálohování, je potřeba toto brát v~potaz
a přizpůsobit tomu strategie zálohování. Nestačí pouze pořídit kopii dat ve
chvíli jejich vzniku, ale je nutné dívat se na data jako na dynamicky se měnící
objekt, u~kterého chceme tyto změny sledovat, a mít možnost vrátit se
k~dřívějšímu stavu, pokud je to zapotřebí.

Pokud si zkusíme rozebrat situace, před kterými by nás zálohovací systém měl
chránit, napadne nás pravděpodobně následující:

\newacronym{RAID}{RAID}{Redundant Array of Inexpensive/Independent Disks -- systém
ukládání dat na více nezávislých disků}

\begin{enumerate}
	\item Závada pevného disku či jiného ukládacího média -- Před tímto nás
	zálohování ochrání možností obnovy poslední zálohované verze dat, ale
	v~případě běžného diskového úložiště se zde nabízí ještě jiná lepší
	alternativa a to použít nějakou variantu \gls{RAID}.
	\item Ztráta, odcizení nebo zničení celého počítače (živelná katastrofa)
	-- Zde nám \gls{RAID} typicky nepomůže, protože jednotlivé disky v~něm
	bývají umístěny na stejném fyzickém místě. Data se dají zachránit
	z~poslední provedené zálohy (zde velmi záleží na politice zálohování,
	jak je záloha aktuální).
	\item Vrácení se do stavu před provedením nějaké akce -- Toto se hodí
	například při neúmyslném smazání některých souborů, nebo při nepovedené
	úpravě systému (instalace nového software aj.). Zde nám naopak techniky
	jako \gls{RAID} nepomohou, protože ty se starají jen o~redundanci
	nejaktuálnější verze uložených dat a nemají historii.
	\item Odhalování útoků na zálohovaný stroj -- Při podezřelé aktivitě na
	zálohovaném stroji lze porovnáním změn v~souborech zjistit, jestli nebyl
	tento stroj napaden (a nebyly například provedeny změny na systémových
	souborech) a v~jakém časovém období se tak případně stalo.
\end{enumerate}

Pro první a druhý případ nám stačí držet si vždy jen poslední verzi souborů.
Pokud ale chceme, aby náš zálohovací systém pokrýval i situace popsané ve třetím
a čtvrtém bodě, je potřeba si nějakým způsobem udržovat více historických verzí
a umět si vyvolat stav dat libovolné z~těchto verzí, případně si mezi libovolnými
dvěma zobrazit rozdíly.

Dalším důležitým aspektem zálohovacího systému diskové místo potřebné
k~zálohování zálohovaného stroje. S~nějakým potřebným místem je nutné počítat, ale
přílišné plýtvání (například způsobené ukládáním celé nové verze souboru při
jeho drobné změně) není na místě. Stejně tak je většinou potřeba \uv{ředit}
staré verze záloh a držet si jen ty významné.

Neméně důležitou věcí je to, že ke všem datům také nepřistupujeme stejně. Některá
data jsou pro nás typicky významnější a je pro nás důležitější zálohovat změny
v~nich prováděné, jiná data tak významná nejsou a není třeba nutné držet si od
těchto dat takové množství historických verzí.

Poslední zásadní věcí je pak pro nás čas provedení zálohy. U~strojů stále
připojených k~rychlé síti to není až tak důležité, ale u~přenosných počítačů,
které jsou k~rychlé síti připojeny jen omezenou dobu nebo u~strojů připojených
po pomalé síti je toto důležité. Zde přichází ke slovu jednak správné sledování
a přenášení jen těch dat, která se změnila, tak prioritizace dat na základě
jejich významu pro nás, jak bylo popsáno v~předchozím odstavci.

Na všechny tyto aspekty jsem se pokoušel při vývoji svého zálohovacího systému
(nazvaného podle bájného mýtického ptáka postávajícího po svém zničení z~popela,
stejně jako to chceme od zálohovaných dat) brát zřetel a zohlednit je. Vznikl
tak zálohovací systém {\it FenixBackup}.
