\chapter*{Závěr}
\addcontentsline{toc}{chapter}{Závěr}

Zálohovací systém FenixBackup se pokusil nabídnout další řešení problému
zálohování dat. V ideálním světe by taková věc vůbec potřeba nebyla, bohužel náš
svět ideální není, ke ztrátám či poškození dat v něm dochází a systémy jako
FenixBackup jsou potřeba.

Častým problémem zálohování je neochota uživatelů strávit čas s jeho
nastavováním, nebo jeho složité použití ve chvíli ztráty dat. Proto jsem se
pokusil při psaní FenixBackup vycházet z principů držet pro uživatele vše co
možná nejpřímočařejší.

Současně jsem se pokusil vyřešit i problém omezeného místa pro zálohy, což
vnímám jako další důvod, proč se někdy se zálohováním vyskytují problémy. Často
uživatelé jednou nakonfigurují zálohování a pak věří tomu, že už bude fungovat
napořád -- bohužel se stává, že zálohovacímu systému dojde místo a uživatelé
tomu nevěnují pozornost. FenixBackup se k tomu staví tak, že má nastavený limit
pro velikost a když je překročen, spustí čištění záloha a maže zálohované verze
souborů s největším spočteným záporným skóre, dokud nevymaže dostatek dat.

Vytvořený zálohovací systém je doufám na počátku dalšího slibného vývoje a zkusím
zde nastínit, jakými dalšími směry by se jeho vývoj mohl ubírat.

\newacronym{SMB}{SMB}{Server Message Block, síťový protokol (nejen) pro přenos
souborů, implementace například projektem Samba}

Jedním směrem vývoje je určitě přidání většího množství adaptérů pro získávání
dat. V prvotní fázi vývoje byl vytvořen jen adaptér pro lokální souborový
systém, ale další adaptéry se nabízejí:
\begin{itemize}
	\item Adaptér připojující vzdálený souborový systém pomocí \gls{SSHFS}
	a dále fungující jako adaptér pro lokální souborový systém.
	\item Adaptér využívající jako protistranu na zálohovaném stroji a jako
	přenosový protokol \texttt{rsync}.
	\item Adaptér připojující se přes protokol \gls{FTP} nebo \gls{SMB} na
	zálohovaný stroj.
	\item Adaptér využívající na zálohovaném stroji běžícího vlastního
	klienta.
\end{itemize}

\newacronym{ACL}{ACL}{Access control list, rozšířený systém práv}

Dalším možným směrem vývoje je modifikovat datový formát tak, aby mohl obsahovat
nepovinné rozšiřující položky -- inspirace například hlavičkami IPv6 paketů,
které obsahují základní společnou hlavičku a pak podle potřeby rozšiřitelné
hlavičky. V těchto nepovinných položkách by mohl sídlit například systém práv
\gls{ACL}, který se u nějakých souborů vyskytuje, ale je zbytečné mít pro něj
vyhrazené pevné datové položky u každého souboru.

Nápad na to použít takový systém ukládání dat přišel bohužel až v pozdější fázi
vývoje systému, kdy již na předělávání současné implementace nebyl dostatek
času, ale je to jeden z cílů, kterým se chci dále věnovat.

Jinou oblastí, ve které se může udělat ještě velký pokrok, je přidat více
inteligence do hledání minulých verzí souborů, nebo do hledání vhodných datových
bloků, na které navázat pomocí rozdílové zálohy. Zde je podle mého ještě velký
prostor, kterým se může systém posunout. Implementace jádra systému je na to
připravená, protože už nyní umožňuje specifikovat, proti kterému datovému bloku
má vznikat rozdílová záloha (i když zatím se volí buď předchozí známá verze
souboru, nebo prázdný soubor).

Poslední oblast, nad kterou aktuálně přemýšlím a která by neměla být příliš
složitá k doimplementaci, je nasazení v prostředí zálohování více podobných
strojů (typicky nějaká firma nebo škola). Dá se vypozorovat, že při zálohování
většího množství podobných strojů je mnoho souborů napříč stroji stejných
(systémové soubory, společná konfigurace aj.). V takovém prostředí by bylo velmiTO
efektivní umět mezi zálohami různých strojů sdílet společná data a zmenšit tak
celkovou velikost všech záloh.

Pokud bychom deaktivovali systém mazání starých záloh, umí to FenixBackup již
v současném stavu -- jednotlivé zálohy budou sdílet společnou datovou složku,
ale budou mít jiné složky, kam ukládají souborové stromy. Protože jsou všechny
informace, jako jsou práva, uloženy v souborových stromech, a protože jsou
datové bloky identifikovány hashem souboru, který zastupují, tak je možné bez
jakýkoliv problémů tyto datové bloky sdílet. Problém nastane jen, kdyby jedna
ze záloh chtěla datové bloky smazat jako nepotřebné -- v tu chvíli by bylo
potřeba sledovat využití datových bloků napříč všemi zálohami ukládajícími do
stejného místa, pro což je nutné přidat podporu (aby zálohy o sobě navzájem
věděly).

FenixBackup je připraven k použití již v současném stavu, ale pevně věřím,
že jeho vývoj bude dále pokračovat a dočká se rozšíření ve výše zmíněných
oblastech a také časem většího nasazení, než jen v řádu jednotek zálohovaných
strojů.
