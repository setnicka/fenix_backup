\chapter{Vlastnosti zálohovacího systému FenixBackup}

Zálohovací systém {\it FenixBackup} dostal své jméno podle mýtického ptáka
fénixe, který dokázal znovu a znovu povstávat znovuzrozený ze svého popela. To
je docela trefná paralela obnově zálohovaných dat, když nás potká nějaká
nešťastná událost.

Nyní se pokusíme představit si hlavní myšlenky, na kterých FenixBackup staví,
a ukázat jeho použití.

\section{Základní přehled}

FenixBackup je souborově orientovaný zálohovací systém schopný si držet více
historických verzí souborů. Každý soubor je v něm navíc ukládán nezávisle a tak
je možné si od určitých souborů držet více historických záloh, než od jiných.

Ukládání jednotlivých verzí funguje na principu rozdílových záloh ve formátu
\gls{VCDIFF} s omezením na maximální hloubku na sebe navázaných delta rozdílů.

Další možností bylo ukládat vždy celé nové verze souborů, sdružovat je k sobě
a komprimovat, jako to dělá třeba verzovací systém Git (stavějící na myšlence,
že podobná data vedle sebe se při komprimaci vhodným kompresním algoritmem
výrazně zmenší), ale zvolil jsem raději cestu přes \gls{VCDIFF}. Při pokusech
s několika většímu binárními soubory vycházely delta rozdíly menší, než když
jsme jednotlivé verze vzal a dohromady zkomprimoval.

Velký důraz byl kladen také na snadnou, přehlednou, ale přitom dostatečně mocnou
schopnost konfigurace, které by měla umožňovat popsat všechny myslitelné
scénáře zálohování.

\newacronym{SSHFS}{SSHFS}{TODO...}

Dalším mým cílem bylo umožnit snadné rozšiřování systému a to jednak ve formě
zajištění zpětné kompatibility dat, tak také ve snadném rozšiřování systému.
Příkladem toho je systém {\it adaptérů} pro přístup k zálohovaným datům:
Existuje společný interface pro komunikaci s adaptéry, ale to, jestli se data
získají z lokálního filesystému, stáhnout přes \gls{SSHFS}, nebo bude probíhat
komunikace přes rsync protokol, by mělo být samotnému jádru zálohovacího
systému jedno.

\section{Konfigurace}

Konfigurační soubor se vztahuje vždy k jednomu zálohovanému stroji a obsahuje
tři části. První specifikuje úložiště zálohovaných dat, druhá z nich popisuje
způsob připojení k zálohovanému stroji -- volba adaptéru a parametry pro něj
(například cesta v rámci lokálního souborového systému, nebo adresu a port
stroje, na který se připojit) a poslední a nejobsáhlejší nakonec popisuje
zálohovací pravidla pro jednotlivé soubory.

\subsection*{Ukázková konfigurace}

Konfigurace pro zápis používá v mnoha projektech rozšířený zápis poskytovaný
knihovnou {\it libconfig}.%
\footnote{http://www.hyperrealm.com/libconfig/libconfig\_manual.html}
Ukázka je připojena níže:

\begin{verbatim}
baseDir: "./backup_dir"
dataSubdir: "data"

adapter: {
  type: "local_filesystem"
  path: "/home/jirka/fenix_backup"
}

paths: (
  { path: "/"
    file_rules: (
      { regex: ".*\.exe"
        backup: false
      }
    )
  },
  { path: "/.git"
    scan: false
  },
  { path: "/test_backup/"
    scan: true
    file_rules: (
      {
        backup: false
      },
      { regex: ".*\.fenixtree"
        path_regex: ""
        size_at_least: 10
        size_at_most: 10485760
        backup: true
      }
    )
  }
)
\end{verbatim}

První položka \texttt{baseDir} je povinná a popisuje, kam přesně má zálohovací
systém ukládat svá data. Pak lze volitelně modifikovat názvy podsložek na
ukládání dat souborů a souborových stromečků (\texttt{dataSubdir}
a \texttt{treeSubdir}).

Druhá položka \texttt{adapter} specifikuje způsob získávání zálohovaných dat.
V první řadě je specifikováno, který adaptér se má použít, a zbytek položek je
pak předán tomuto adaptéru jako jeho nastavení.

Poslední části konfigurace se budeme věnovat v další podkapitole.

\subsection{Zálohovací pravidla pro složky a soubory}

Pravidla je možné specifikovat pro jakoukoliv cestu v zálohovaných datech
(výchozí složka zálohy je označována jen pomocí lomítka: \texttt{/}) a stejnou
cestu je možné uvést i vícekrát -- v takovém případě platí pro tuto cestu
poslední specifikovaná pravidla.

Pokud pro nějakou cestu nejsou specifikována pravidla (v reálném případě to bude
pravděpodobně většina cest), použijí se pro ní nejbližší vyšší pravidla po cestě
ke kořeni. Je možné tak třeba zakázat zálohování určitých složek.

Mimo pravidel odvozených jen od cest existují také pravidla umožňující pokrýt
třeba jen určitý typ souboru. Jsou také vázána na konkrétní cestu (a platí tak
jen v určeném podstromě), ale je možné u nich zavést složitější způsoby
filtrování. Je možné filtrovat podle následujících parametrů:

\begin{itemize}
	\item Regex\footnote{regulární výraz} názvu souboru -- pokud je nevyplněný,
	odpovídá jakýkoliv soubor, pokud je neprázdný, musí celý název souboru
	(jen souboru, bez cesty) odpovídat zadanému regexu.
	\item Regex cesty k souboru -- podobně jako výše, nevyplněnému odpovídá vše,
	vyplněnému jen soubory, které se nacházejí (oproti aktuálnímu adresáři)
	na cestě odpovídající regexu.
	\item Minimální a maximální velikost souboru (v bytech).
\end{itemize}

\noindent Při dotazu na konkrétní soubor se parametry pravidel pak určí takto:
\begin{enumerate}
	\item Vezmou se defaultní hodnoty pravidel.
	\item Vyjde se z kořenové složky a postupně se aplikují všechna
	vyhovující pravidla cestou (zadaná pravidla v tomto uzlu přepíší
	dřívější zapamatované hodnoty).
	\item V každém uzlu cesty, pokud zde existují pravidla pro jednotlivé
	soubory, zkusí se všechny otestovat a aplikovat.
	\item Pokud se dá po zadané cestě sestoupit ještě o úroveň níže, sestoupí
	se a opakuje se znovu od bodu 3.
	\item Nakonec se vrátí finální pravidla vzniklá aplikací všech pravidel
	cestou.
\end{enumerate}

\noindent Poslední důležitou otázkou je, jaké všechna parametry lze nastavovat:
\begin{itemize}
	\item {\tt scan <true|false>}: Platí jen pro složky a říká, jestli vůbec
	tuto složku prozkoumávat. Defaultně {\tt true}.
	\item {\tt backup <true|false>}: Jestli zadaný soubor zálohovat, nebo
	ne. Defaultně {\tt true}.
	\item {\tt history <0-10>}: Parametr pro funkci na ředění historie, čím
	vyšší číslo, tím více verzí držet. Defaultně {\tt 10}.
	\item {\tt priority <0-10>}: Parametr pro prioritizační funkci. Určuje,
	které soubory zálohovat dříve, než ostatní.
\end{itemize}

\section{Adaptéry pro přístup k zálohovaným datům}

\section{Model ukládání dat}

