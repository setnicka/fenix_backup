\chapter{Vlastnosti zálohovacího systému FenixBackup}

Zálohovací systém {\it FenixBackup} dostal své jméno podle mýtického ptáka
fénixe, který dokázal znovu a znovu povstávat znovuzrozený ze svého popela. To
je docela trefná paralela obnově zálohovaných dat, když nás potká nějaká
nešťastná událost.

Nyní se pokusíme představit si hlavní myšlenky, na kterých FenixBackup staví,
a základní principy jeho fungování.

\section{Základní přehled}

FenixBackup je souborově orientovaný zálohovací systém schopný si držet více
historických verzí souborů. Každý soubor je v něm navíc ukládán nezávisle a tak
je možné si od určitých souborů držet více historických záloh, než od jiných.

Ukládání jednotlivých verzí funguje na principu rozdílových záloh ve formátu
\gls{VCDIFF} s omezením na maximální hloubku na sebe navázaných delta rozdílů.

Další možností bylo ukládat vždy celé nové verze souborů, sdružovat je k sobě
a komprimovat, jako to dělá třeba verzovací systém Git (stavějící na myšlence,
že podobná data vedle sebe se při komprimaci vhodným kompresním algoritmem
výrazně zmenší), ale zvolil jsem raději cestu přes \gls{VCDIFF}. Při pokusech
s několika většímu binárními soubory vycházely delta rozdíly menší, než když
jsme jednotlivé verze vzal a dohromady zkomprimoval.

Velký důraz byl kladen také na snadnou, přehlednou, ale přitom dostatečně mocnou
schopnost konfigurace, které by měla umožňovat popsat všechny myslitelné
scénáře zálohování.

\newacronym{SSHFS}{SSHFS}{SSH File System -- připojení vzdáleného souborového
systému přes protokol \texttt{SSH}}

Dalším mým cílem bylo umožnit snadné rozšiřování systému a to jednak ve formě
zajištění zpětné kompatibility dat, tak také ve snadném rozšiřování systému.
Příkladem toho je systém {\it adaptérů} pro přístup k zálohovaným datům:
Existuje společný interface pro komunikaci s adaptéry, ale to, jestli se data
získají z lokálního filesystému, stáhnout přes \gls{SSHFS}, nebo bude probíhat
komunikace přes rsync protokol, by mělo být samotnému jádru zálohovacího
systému jedno.

\section{Pravidla konfigurace}

Zásadní částí dobře fungujícího zálohovacího systému je dobře fungující
a konfigurovatelný systém pravidel zálohování. Zápis pravidel do konfiguračního
souboru je ukázán v kapitole vztahující se k uživatelské dokumentaci, zde se
pokusíme rozebrat obecné fungování pravidel.

\subsection{Zálohovací pravidla určená podle cest}

Pravidla je možné specifikovat pro jakoukoliv cestu v zálohovaných datech
(kořenová složka zálohy je označována pomocí lomítka: \texttt{/}) a stejnou
cestu je možné uvést i vícekrát -- v takovém případě platí pro tuto cestu
poslední specifikovaná pravidla.

Pokud pro nějakou cestu nejsou specifikována pravidla (v reálném případě to bude
pravděpodobně většina cest), použijí se pro ní nejbližší vyšší pravidla po cestě
ke kořeni. Je možné tak třeba zakázat zálohování určitých složek.

\medskip

\noindent{\bf Přehled a význam parametrů, které lze nastavovat:}
\begin{itemize}
	\item {\tt scan <true|false>}: Platí jen pro složky a říká, jestli vůbec
	tuto složku prozkoumávat. Defaultně {\tt true}.
	\item {\tt backup <true|false>}: Jestli zadaný soubor zálohovat, nebo
	ne. Defaultně {\tt true}.
	\item {\tt history <1-10>}: Parametr pro funkci na ředění historie, čím
	vyšší číslo, tím více verzí držet. Defaultně 1.
	\item {\tt priority <1-10>}: Parametr pro prioritizační funkci. Určuje,
	které soubory zálohovat dříve, než ostatní. Čím vyšší číslo, tím
	důležitější. Defaultně 1.
\end{itemize}

\subsection{Zálohovací pravidla podle specifičtějšího výběru}

Mimo pravidel odvozených jen od cest existují také pravidla umožňující pokrýt
třeba jen určitý typ souboru. Jsou také vázána na konkrétní cestu (a platí tak
jen v určeném podstromě), ale je možné u nich zavést složitější způsoby
filtrování.

Pro každou cestu je možné v konfiguraci uvést seznam {\it pravidel pro soubory},
kde každé pravidlo má parametry, které nastavuje, a pak filtry určující, na jaké
soubory se má aplikovat. Je možné filtrovat podle následujících parametrů:

\begin{itemize}
	\item Regex\footnote{regulární výraz} názvu souboru (\texttt{regex}) --
	pokud je nevyplněný či prázdný, odpovídá jakýkoliv soubor, pokud je
	neprázdný, musí celý název souboru (jen souboru, bez cesty) odpovídat
	zadanému regexu.
	\item Regex cesty k souboru (\texttt{path\_regex}) -- podobně jako výše,
	nevyplněnému odpovídá vše, vyplněnému jen soubory, které se nacházejí
	(oproti aktuálnímu adresáři) na cestě odpovídající regexu.
	\item Minimální a maximální velikost souboru v bytech (parametry
	\texttt{size\_at\_least} a \texttt{size\_at\_most}).
\end{itemize}

Jednotlivá pravidla se vždy aplikují v pořadí, v jakém jsou uvedena v konfiguraci,
a funguje zde princip přepisování (tedy pokud je splněna podmínka pravidla, jsou
přepsány všechny parametry zálohování, které pravidlo nastavuje).

\medskip\goodbreak

\noindent Při dotazu na konkrétní soubor se pak parametry pravidel určí takto:
\begin{enumerate}
	\item Vezmou se defaultní hodnoty pravidel a začne se v kořenové složce.
	\item V aktuální složce se nejdříve aplikují pravidla složky (pokud
	existují) a poté se projdou všechny pravidla pro soubory:
	\item Pokud vyhovuje filtr u pravidla pro soubory, aplikuje se.
	\item Pokud se dá po zadané cestě sestoupit ještě o úroveň níže, sestoupí
	se a opakuje se znovu od bodu 2.
\end{enumerate}

Nakonec se vrátí finální pravidla vzniklá aplikací všech pravidel, přes která se
prošlo. Toto řešení nabízí velmi silné možnosti konfigurace, bohužel cacheování
výsledků je zde vzhledem k pravidlům s filtry značně obtížné, pro běžně velké
konfigurační soubory by ale rychlost i tak měla být dostatečná a úzkým hrdlem
zde stále bude rychlost načítání dat z disku.

\section{Model ukládání dat}

FenixBackup představuje každou zálohu (ať už se v rámci ní zálohovaly všechny
soubory, nebo se povedlo zálohovat pouze jediný) jedním souborovým stromem. Ten
obsahuje všechny záznamy o všech souborech, které by se měly dostat do zálohy.

Při svém vzniku obsahuje tento strom jen informace zjistitelné o souborech bez
nutnosti číst jejich obsah (to zajistí příslušný adaptér, viz následující
podkapitola) a postupně se zpracováním obsahu jednotlivých souborů se v něm
objevují i kontrolní součty (hashe) souborů, které odkazují na konkrétní datové
bloky.

\newacronym{SHA256}{SHA256}{Kryptografická hashovací funkce, vytváří otisk o
velikosti 256 bitů}

Každý uložený soubor je jednoznačně identifikován svým \gls{SHA256} hashem
a představován stejnojmenným datovým blokem. Tento datový blok může být buď
úplný, nebo je to jen \gls{VCDIFF} oproti jinému datovému bloku.

Pro fungování zálohovacího systému předpokládáme, že nalezení kolize v
kryptografické hashovací funkci jako je \gls{SHA256} je extrémně
nepravděpodobné (zatím není známý ani cílený útok, který by uměl vyprodukovat
kolizi, a velikost kolizní domény je $2^{256}$). Na stejném předpokladu
o nepravděpodobných kolizích v reálném nasazení staví i verzovací systém Git,
který navíc používá jen kratší hashe generované \gls{SHA1}.

Díky tomu, že jsou datové bloky pojmenovány hashem souboru, který mají
představovat, řeší se velmi efektivním způsobem duplikáty souborů. Ačkoliv třeba
dva stejné soubory mohou mít úplně rozdílnou historii, tak finálně je uložen
v zálohovacím systému jen jeden datový blok, který představuje oba soubory.

Datové bloky a jejich případné rozdílové návaznosti tedy vůbec nemusí
korespondovat s historií jednotlivých souborů uloženou v souborových stromech.
Ale díky tomu, že v souborovém stromu je uložený hash každé jednotlivé verze
souboru, lze všechny historické verze snadno zrekonstruovat.

Záznam o každém jednotlivém souboru se může navíc nacházet v jednom z
následujících stavů:
\begin{itemize}
	\item\texttt{UNKNOWN} -- Nový soubor (bez známé minulé verze), u kterého
	zatím nemáme zálohovaný jeho obsah.
	\item\texttt{NEW} -- Nový soubor (bez známé minulé verze), u kterého již
	máme uložený a zpracovaný jeho obsah.
	\item\texttt{UNCHANGED} -- Soubor se známou minulou verzí, který se
	nezměnil (ani parametry, ani obsahem).
	\item\texttt{UPDATED\_PARAMS} -- Soubor se známou minulou verzí, u kterého
	se změnily pouze parametry, ale obsah ne.
	\item\texttt{NOT\_UPDATED} -- Soubor se známou minulou verzí, u kterého
	se změnil i obsah, a jenž ještě nemáme uložený.
	\item\texttt{UPDATED\_FILE} -- Soubor se známou minulou verzí, u kterého
	se změnil i obsah, a jenž již máme uložený a zpracovaný.
	\item\texttt{DELETED} -- Soubor byl z nějakého důvodu vymazán (ať už na
	přímé přání uživatele, nebo automaticky při uvolňování místa).
\end{itemize}

\section{Adaptéry pro přístup k zálohovaným datům}

Zálohovací systém je psán tak, aby byl nezávislý na metodě přístupu k zálohovaným
datům. Samotnému jádru je jedno, jak data získá, o to se starají takzvané
{\it adaptéry}.

V úloze zálohování se dají jejich úkoly rozdělit do dvou fází. V první fázi
získávají adaptéry informace o souborech a předávají je jádru zálohovacího
systému, ve druhé fázi jim pak jádro zálohovacího systému dodá seznam souborů,
které chce získat, a adaptér nějakým způsobem opatří jejich obsah a dodá je
nazpět jádru.

První fáze začíná tím, že adaptér začne skenovat systém souborů od zadaného
místa, načítá parametry souborů (vlastník, práva, velikost, \dots) a ukládá tyto
záznamy do souborového stromu. Jádro zálohovacího systému všechny ukládané
záznamy zpracovává a může případně operativně rozhodnout, že ho informace v nějakém
podstromě již nezajímají (třeba v případě nastavení \texttt{scan: false} pro
nějakou cestu v konfiguraci). V takovém případě to dá domluveným způsobem vědět
adaptéru a ten by již tento podstrom neměl dále procházet.

Během této fáze se jádro zálohovacího systému pokouší (podle cesty, času poslední
modifikace a velikosti souborů) rozhodnout, které soubory se nezměnily, u kterých
se změnily pouze parametry, a které soubory bude potřeba stáhnout pro uložení
jejich nové verze.

Seznam požadovaných souborů (utříděný podle prioritizační funkce, viz dále) pak
předá nazpět adaptéru a ten by měl zařídit získání obsahu těchto souborů a
předání obsahu jádru zálohovacího systému.

To si spočte hash souborů, zjistí, zdali případně nemá již soubor se stejným
hashem uložený, a pokud ne, uloží nový datový blok (buď jako kompletní datový
blok, nebo jako \gls{VCDIFF} oproti jinému).

\section{Priorita zálohovaných souborů}

Různé soubory mohou mít konfigurací jinak nastavené priority zálohování, čímž
lze zálohovacímu systému sdělit, o které soubory máme větší zájem.

Do rozhodovacího procesu se také promítá to, jakou nejmladší verzi daného
souboru již máme uloženou. Systém je totiž připravený i na to, že ne vždy se
povede provést kompletní zálohu, například u přenosných počítačů připojených k
rychlé síti jen po omezenou dobu. V takovém případě zůstanou některé soubory ve
stromu se stavem \texttt{UNKNOWN} nebo \texttt{NOT\_UPDATED} a další zálohy na
to berou ohled (soubory, jež se nepovedlo zálohovat delší dobu, mají při příštím
zálohování přednost).

Konkrétně si jádro zálohovacího systému přiděluje každému souboru číselné
{\it skóre} odpovídající následujícímu vztahu:

$$[\hbox{čas od poslední zálohované verze}]\times[\hbox{priorita souboru}]$$

Pokud soubor nemá známou minulou verzi, tak se jako čas poslední zálohované
verze bere čas minulého spuštění zálohování -- protože tento nový soubor se mohl
objevit kdykoliv v mezičase mezi touto poslední zálohou a současností.

Tímto postupem by se mělo zajistit zálohování postupně všech souborů i v případě,
že zálohovaný stroj bývá připojen k zálohovacímu systému pravidelně jen po
omezenou dobu.
