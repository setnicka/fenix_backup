%%% Hlavní soubor. Zde se definují základní parametry a odkazuje se na ostatní části. %%%

%% Verze pro jednostranný tisk:
% Okraje: levý 40mm, pravý 25mm, horní a dolní 25mm
% (ale pozor, LaTeX si sám přidává 1in)
\documentclass[12pt,a4paper]{report}
\setlength\textwidth{145mm}
\setlength\textheight{247mm}
\setlength\oddsidemargin{15mm}
\setlength\evensidemargin{15mm}
\setlength\topmargin{0mm}
\setlength\headsep{0mm}
\setlength\headheight{0mm}
% \openright zařídí, aby následující text začínal na pravé straně knihy
\let\openright=\clearpage

%% Pokud tiskneme oboustranně:
% \documentclass[12pt,a4paper,twoside,openright]{report}
% \setlength\textwidth{145mm}
% \setlength\textheight{247mm}
% \setlength\oddsidemargin{15mm}
% \setlength\evensidemargin{0mm}
% \setlength\topmargin{0mm}
% \setlength\headsep{0mm}
% \setlength\headheight{0mm}
% \let\openright=\cleardoublepage

%% Pokud používáte csLaTeX (doporučeno):
%\usepackage{czech}
\usepackage[czech]{babel}

%% Použité kódování znaků: obvykle latin2, cp1250 nebo utf8:
\usepackage[utf8]{inputenc}
%% Ostatní balíčky
\usepackage{graphicx}
\usepackage{amsthm}
\usepackage[footnote,acronym,nomain]{glossaries}
%\setglossarystyle{altlist}
\setglossarystyle{list}

\makeglossaries

%% Balíček hyperref, kterým jdou vyrábět klikací odkazy v PDF,
%% ale hlavně ho používáme k uložení metadat do PDF (včetně obsahu).
\usepackage[pdftex,unicode]{hyperref}   % Musí být za všemi ostatními balíčky
\hypersetup{pdftitle=Zálohovací systém}
\hypersetup{pdfauthor=Jiří Setnička}

%%% Drobné úpravy stylu

% Tato makra přesvědčují mírně ošklivým trikem LaTeX, aby hlavičky kapitol
% sázel příčetněji a nevynechával nad nimi spoustu místa. Směle ignorujte.
\makeatletter
\def\@makechapterhead#1{
  {\parindent \z@ \raggedright \normalfont
   \Huge\bfseries \thechapter. #1
   \par\nobreak
   \vskip 20\p@
}}
\def\@makeschapterhead#1{
  {\parindent \z@ \raggedright \normalfont
   \Huge\bfseries #1
   \par\nobreak
   \vskip 20\p@
}}
\makeatother

% Toto makro definuje kapitolu, která není očíslovaná, ale je uvedena v obsahu.
\def\chapwithtoc#1{
\chapter*{#1}
\addcontentsline{toc}{chapter}{#1}
}

\begin{document}

% Trochu volnější nastavení dělení slov, než je default.
\lefthyphenmin=2
\righthyphenmin=2

%%% Titulní strana práce
\setcounter{page}{-5} %% Hack kvůli generování obsahu a odkazům na stránky, aby se strana 1 neodkazovala sem

\pagestyle{empty}
\begin{center}

\large

Univerzita Karlova v Praze

\medskip

Matematicko-fyzikální fakulta

\vfill

{\bf\Large BAKALÁŘSKÁ PRÁCE}

\vfill

\centerline{\mbox{\includegraphics[width=60mm,type=pdf,ext=.epdf,read=.epdf]{mfflogo}}}

\vfill
\vspace{5mm}

{\LARGE Jiří Setnička}

\vspace{15mm}

% Název práce přesně podle zadání
{\LARGE\bfseries Zálohovací systém}

\vfill

% Název katedry nebo ústavu, kde byla práce oficiálně zadána
% (dle Organizační struktury MFF UK)
Katedra aplikované matematiky

\vfill

\begin{tabular}{rl}

Vedoucí bakalářské práce: & Mgr. Martin Mareš, Ph.D. \\
\noalign{\vspace{2mm}}
Studijní program: & Informatika \\
\noalign{\vspace{2mm}}
Studijní obor: & Obecná informatika \\
\end{tabular}

\vfill

% Zde doplňte rok
Praha 2015

\end{center}

\newpage

%%% Následuje vevázaný list -- kopie podepsaného "Zadání bakalářské práce".
%%% Toto zadání NENÍ součástí elektronické verze práce, nescanovat.

%%% Na tomto místě mohou být napsána případná poděkování (vedoucímu práce,
%%% konzultantovi, tomu, kdo zapůjčil software, literaturu apod.)

\openright

\noindent
Poděkování TODO.

\newpage

%%% Strana s čestným prohlášením k bakalářské práci

\vglue 0pt plus 1fill

\noindent
Prohlašuji, že jsem tuto bakalářskou práci vypracoval samostatně a výhradně
s~použitím citovaných pramenů, literatury a dalších odborných zdrojů.

\medskip\noindent
Beru na~vědomí, že se na moji práci vztahují práva a povinnosti vyplývající
ze zákona č. 121/2000 Sb., autorského zákona v~platném znění, zejména skutečnost,
že Univerzita Karlova v Praze má právo na~uzavření licenční smlouvy o~užití této
práce jako školního díla podle §60 odst. 1 autorského zákona.

\vspace{10mm}

\hbox{\hbox to 0.5\hsize{%
V \hbox to 2cm{\dotfill} dne \hbox to 2cm{\dotfill}
\hss}\hbox to 0.5\hsize{%
%Podpis autora
\hss}}

\vspace{20mm}
\newpage

%%% Povinná informační strana bakalářské práce

\vbox to 0.5\vsize{
\setlength\parindent{0mm}
\setlength\parskip{5mm}

Název práce:
Zálohovací systém

Autor:
Jiří Setnička

Katedra:
Katedra aplikované matematiky

Vedoucí bakalářské práce:
Mgr. Martin Mareš, Ph.D., Katedra aplikované matematiky

Abstrakt: TODO
% abstrakt v rozsahu 80-200 slov; nejedná se však o opis zadání bakalářské práce

Klíčová slova: zálohování
% 3 až 5 klíčových slov

\vss}\nobreak\vbox to 0.49\vsize{
\setlength\parindent{0mm}
\setlength\parskip{5mm}

Title:
A backup system

Author:
Jiří Setnička

Department:
Department of Applied Mathematics
% dle Organizační struktury MFF UK v angličtině

Supervisor:
Mgr. Martin Mareš, Ph.D., Department of Applied Mathematics

Abstract: TODO
% abstrakt v rozsahu 80-200 slov v angličtině; nejedná se však o překlad
% zadání bakalářské práce

Keywords: TODO
% 3 až 5 klíčových slov v angličtině

\vss}

\newpage

%%% Strana s automaticky generovaným obsahem bakalářské práce. U matematických
%%% prací je přípustné, aby seznam tabulek a zkratek, existují-li, byl umístěn
%%% na začátku práce, místo na jejím konci.

\openright
\tableofcontents
\thispagestyle{empty}

\newpage
\setcounter{page}{1}
\pagestyle{plain}

%%% Jednotlivé kapitoly práce jsou pro přehlednost uloženy v samostatných souborech
\chapter*{Úvod}
\addcontentsline{toc}{chapter}{Úvod}

Potřeba zálohovat data se pojí s počítačovým světem prakticky od jeho vzniku
a vychází z dob ještě před jeho existencí -- z potřeby udržovat kopie například
důležitých dokumentů a chránit je tak před náhodným nebo cíleným zničením nebo
ztrátou.

V dřívějších dobách stačilo pořídit fyzickou kopii důležitého dokumentu a ten
uložit ideálně na jiném fyzickém místě, s nástupem počítačového ukládání dat se
k tomu však ještě přidal aspekt toho, že data je možná snadno editovat a měnit.

Pokud tedy v současné době mluvíme o zálohování, je potřeba toto brát v potaz
a přizpůsobit tomu strategie zálohování. Nestačí pouze pořídit kopii dat ve
chvíli jejich vzniku, ale je nutné dívat se na data jako na dynamicky se měnící
objekt, u kterého chceme tyto změny sledovat, a mít možnost vrátit se
k dřívějšímu stavu, pokud je to zapotřebí.

Pokud si zkusíme rozebrat situace, před kterými by nás zálohovací systém měl
chránit, napadne nás pravděpodobně následující:

\newacronym{RAID}{RAID}{Redundant Array of Inexpensive/Independent Disks -- systém
ukládání dat na více nezávislých disků}

\begin{enumerate}
	\item Závada pevného disku či jiného ukládacího média -- Před tímto nás
	zálohování ochrání možností obnovy poslední zálohované verze dat, ale
	v případě běžného diskového úložiště se zde nabízí ještě jiná lepší
	alternativa a to použít nějakou variantu \gls{RAID}.
	\item Ztráta, odcizení nebo zničení celého počítače (živelná katastrofa)
	-- Zde nám \gls{RAID} typicky nepomůže, protože jednotlivé disky v něm
	bývají umístěny na stejném fyzickém místě. Data se dají zachránit
	z poslední provedené zálohy (zde velmi záleží na politice zálohování,
	jak je záloha aktuální).
	\item Vrácení se do stavu před provedením nějaké akce -- Toto se hodí
	například při neúmyslném smazání některých souborů, nebo při nepovedené
	úpravě systému (instalace nového software aj.). Zde nám naopak techniky
	jako \gls{RAID} nepomohou, protože ty se starají jen o redundanci
	nejaktuálnější verze uložených dat a nemají historii.
	\item Odhalování útoků na zálohovaný stroj -- Při podezřelé aktivitě na
	zálohovaném stroji lze porovnáním změn v souborech zjistit, jestli nebyl
	tento stroj napaden (a nebyly například provedeny změny na systémových
	souborech) a v jakém časovém období se tak případně stalo.
\end{enumerate}

Pro první a druhý případ nám stačí držet si vždy jen poslední verzi souborů.
Pokud ale chceme, aby náš zálohovací systém pokrýval i situace popsané ve třetím
a čtvrtém bodě, je potřeba si nějakým způsobem udržovat více historických verzí
a umět si vyvolat stav dat libovolné z těchto verzí, případně si mezi libovolnými
dvěma zobrazit rozdíly.

Dalším důležitým aspektem zálohovacího systému diskové místo potřebné k
zálohování zálohovaného stroje. S nějakým potřebným místem je nutné počítat, ale
přílišné plýtvání (například způsobené ukládáním celé nové verze souboru při
jeho drobné změně) není na místě. Stejně tak je většinou potřeba \uv{ředit}
staré verze záloh a držet si jen ty významné.

Neméně důležitou věcí je to, že ke všem datům také nepřistupujeme stejně. Některá
data jsou pro nás typicky významnější a je pro nás důležitější zálohovat změny
v nich prováděné, jiná data tak významná nejsou a není třeba nutné držet si od
těchto dat takové množství historických verzí.

Poslední zásadní věcí je pak pro nás čas provedení zálohy. U strojů stále
připojených k rychlé síti to není až tak důležité, ale u přenosných počítačů,
které jsou k rychlé síti připojeny jen omezenou dobu nebo u strojů připojených
po pomalé síti je toto důležité. Zde přichází ke slovu jednak správné sledování
a přenášení jen těch dat, která se změnila, tak prioritizace dat na základě
jejich významu pro nás, jak bylo popsáno v předchozím odstavci.

Na všechny tyto aspekty jsem se pokoušel při vývoji svého zálohovacího systému
(nazvaného podle bájného mýtického ptáka postávajícího po svém zničení z popela,
stejně jako to chceme od zálohovaných dat) brát zřetel a zohlednit je. Vznikl
tak zálohovací systém {\it FenixBackup}.

\chapter{Existující metody a systémy zálohování}

Jak již bylo popsáno v úvodu práce, dá se k zálohování přistupovat mnoha způsoby.
Pojďme si zkusit tyto způsoby rozebrat, podívat se na ně z několika hledisek
efektivity a zhodnotit jejich klady a zápory. Dále se zkusme podívat na
několik existujících zálohovacích řešení.

\section{Základní kopírování}

Při zálohování vždy jde o to držet si nějakým způsobem kopii nějakých dat. Výběr
dat k zálohování nyní odložíme stranou a budeme si představovat, že máme nějakou
dobře definovanou množinu souborů, které chceme zálohovat.

\newacronym{FTP}{FTP}{File Transfer Protocol -- jednoduchý síťový protokol pro
přenos souborů}
\newacronym{SSH}{SSH}{Secure shell -- zabezpečený komunikační síťový protokol}
\newacronym{SCP}{SCP}{Secure copy -- zabezpečený protokol pro přenos souborů na
bázi \gls{SSH}}

Nejjednodušším způsobem co do implementace je asi {\bf ruční kopírování} --
uživatel sám čas od času zkopíruje zálohované soubory na nějaké jiné úložiště.
Na tomto úložišti si uživatel může držet buď jen jednu kopii dat (vždy jen
nejnovější verzi), což zabere řádově tolik místa, kolik zabírají zálohovaná data
na zálohovaném stroji, nebo si může držet všechny historické verze. V takovém
případě ale zálohy zaberou místo $[\hbox{počet záloh}]\times[\hbox{velikost dat}]$,
což je pro časté zálohování velkého objemu dat neudržitelné (objem dat lze sice
snížit vhodnou kompresí, ale to oddálí problém jen o kus dále).

Jako metodu kopírování lze zvolit mnoho protokolů, od kopírování po místním
souborovém systému (což nebývá moc účinné proti selhání celého počítače nebo
jeho ztrátou), přes kopírování přes síť pomocí \gls{FTP} nebo nad \gls{SSH}
postaveným \gls{SCP}.

Všechny tyto metody kopírování jsou ale \uv{hloupé}, udělají jen přesně to, co
od nich žádáme -- zkopírují zadané soubory na cílové úložiště bez starosti o to,
jestli již zde stejné soubory nebyly kopírovány dříve. Pokud byly provedeny jen
malé změny, tak přenášíme zbytečně velké objemy dat, které již v cílovém místě
uložená jsou.

Kdybychom dokázali dostatečně bezpečně poznat, jaká data se nezměnila, mohli
bychom objem přenášených dat výrazně snížit a kopírovat jen rozdílná data.
Když zálohování provádí uživatel sám a ručně, může posloužit jako rozhodovací
veličina v tom, která data se podle jeho mínění změnila a která je tedy potřeba
zálohovat. To je ale jednak nespolehlivé a může to vést k chybám, ale hlavně se
tento postup nedá použít při zautomatizování zálohování (což by mělo být jedním
z hlavních cílů zálohovacích systémů).

\newpage

\section{Kopírování rozdílů}

Když si vezmeme za cíl kopírovat jen ta data, která se oproti poslednímu
zálohovanému stavu změnila, dostáváme se k otázce, jak dostatečně bezpečně
poznat, co se změnilo.

Vzhledem k vlastnostem v současnosti používaných souborových systémů je
nejsnadnější dívat se na změny soubor po souboru a v případě změny ho přenést
celý. Dá se zavést i drobnější granularita a sledovat data po menších blocích,
než po celých souborech, ale to je již výrazně obtížnější.

\newacronym{UNIX}{UNIX}{Operační systém vzniklý v roce 1969, který dal vzniknout
celé rodině odvozených systémů -- Linux, BSD, Mac OS X, aj.}

Současné souborové systémy si u každého souboru pamatují několik parametrů
(vezmeme si za příklad souborové systémy vycházející z operačního systému
\gls{UNIX}): vlastník a skupina, vlastnická, skupinová a ostatní práva, čas
poslední změny, velikost souboru a několik dalších parametrů.

Pokud uvažujeme soubory s parametry, je potřeba odlišit změny pouze v nich
(například změna vlastníka či skupiny) od změn na datech souboru. Parametry
souboru je nutné si přečíst vždy a tak změny na nich poznáme jednoduše, možností
jak poznat změnu na datech souboru je pak vícero:

\newacronym{MD5}{MD5}{TODO...}
\newacronym{SHA1}{SHA1}{TODO...}

\begin{itemize}
	\item Kompletní porovnání obsahu souboru po bytech -- zabere čas lineární
	s velikostí souboru a stejnou přenosovou kapacitu, je ale neomylné.
	\item Kontrolní součet obsahu souboru -- je nutné stále spočítat kontrolní
	součet (třeba pomocí \gls{MD5} nebo \gls{SHA1}) souboru v lineárním čase
	k jeho velikosti, ale přenést mezi zálohovaným a zálohovacím strojem je
	potřeba pouze tento kontrolní součet (a porovnat ho s uloženým). Při volbě
	dobré hashovací funkce se věří, že nalezení {\it kolize} (nalezení více
	různých souborů se stejným hashem) je extrémně nepravděpodovné a tedy
	spolehlivé.
	\item Porovnání jen parametrů (času poslední modifikace a velikosti) --
	je nejrychlejší (jde jen o několik porovnání v konstantním čase) a pro
	většinu situací je dostačující.
\end{itemize}

Důvodem, proč je poslední možnost považována za dostačující je ta, že je při
běžném provozu jen velmi málo situací, kdy se změní obsah souboru, ale nezmění
se čas jeho poslední modifikace a jeho velikost. Většinou to vyžaduje přímou
a vědomou akci, kterou se čas modifikace souboru nastaví zpět na původní
hodnotu.

Situace, ve které je porovnání jen podle velikosti a času modifikace
nedostačující, je již nastíněné odhalování útoků na zálohovaný stroj. Pokud se
již útočníkovi povede získat kontrolu nad strojem, může snadno modifikovat
systémové soubory a potom změnit čas jejich modifikace nazpět. Pokud se před
takovou situací chceme chránit, je vhodné občas nechat zkontrolovat alespoň
kontrolní součty všech souborů oproti verzi uložené v zálohovacím systému.

V současnosti jedním z nejvíce využívaných protokolů a současně i programů
implementující nějakým způsobem kopírování jen rozdílných souborů je {\bf rsync}
(viz \cite{rsync}). Ten potřebuje na jedné straně běžícího klienta a na druhé
straně běžící server, kteří si mezi sebou vyměňují různé parametry a kontrolní
součty souborů a jejich částí. Hlavním cílem je snížit objem síťové komunikace
na minimum a přenášet jen nezbytné množství dat (přenos navíc probíhá
komprimovaně).

Bohužel primárním cílem rsyncu je {\it synchronizace} dvou složek a není tak
úplně vhodný k tomu ukládat si více záloh stejných dat do různých složek --
neumí například sáhnout na soubory v jiné složce a použít je, pokud se obsah
těchto souborů nezměnil. Hodí se tedy pro držení jedné kopie nejaktuálnějších
dat, ale pro plnohodnotné zálohování je potřeba sáhnout po něčem mocnějším.

\chapter{Vlastnosti zálohovacího systému FenixBackup}

Zálohovací systém {\it FenixBackup} dostal své jméno podle mýtického ptáka
fénixe, který dokázal znovu a znovu povstávat znovuzrozený ze svého popela. To
je docela trefná paralela obnově zálohovaných dat, když nás potká nějaká
nešťastná událost.

Nyní se pokusíme představit si hlavní myšlenky, na kterých FenixBackup staví,
a základní principy jeho fungování.

\section{Základní přehled}

FenixBackup je souborově orientovaný zálohovací systém schopný si držet více
historických verzí souborů. Každý soubor je v něm navíc ukládán nezávisle a tak
je možné si od určitých souborů držet více historických záloh, než od jiných.

Ukládání jednotlivých verzí funguje na principu rozdílových záloh ve formátu
\gls{VCDIFF} s omezením na maximální hloubku na sebe navázaných delta rozdílů.

Další možností bylo ukládat vždy celé nové verze souborů, sdružovat je k sobě
a komprimovat, jako to dělá třeba verzovací systém Git (stavějící na myšlence,
že podobná data vedle sebe se při komprimaci vhodným kompresním algoritmem
výrazně zmenší), ale zvolil jsem raději cestu přes \gls{VCDIFF}. Při pokusech
s několika většímu binárními soubory vycházely delta rozdíly menší, než když
jsme jednotlivé verze vzal a dohromady zkomprimoval.

Velký důraz byl kladen také na snadnou, přehlednou, ale přitom dostatečně mocnou
schopnost konfigurace, které by měla umožňovat popsat všechny myslitelné
scénáře zálohování.

\newacronym{SSHFS}{SSHFS}{SSH File System -- připojení vzdáleného souborového
systému přes protokol \texttt{SSH}}

Dalším mým cílem bylo umožnit snadné rozšiřování systému a to jednak ve formě
zajištění zpětné kompatibility dat, tak také ve snadném rozšiřování systému.
Příkladem toho je systém {\it adaptérů} pro přístup k zálohovaným datům:
Existuje společný interface pro komunikaci s adaptéry, ale to, jestli se data
získají z lokálního filesystému, stáhnout přes \gls{SSHFS}, nebo bude probíhat
komunikace přes rsync protokol, by mělo být samotnému jádru zálohovacího
systému jedno.

\section{Pravidla konfigurace}

Zásadní částí dobře fungujícího zálohovacího systému je dobře fungující
a konfigurovatelný systém pravidel zálohování. Zápis pravidel do konfiguračního
souboru je ukázán v kapitole vztahující se k uživatelské dokumentaci, zde se
pokusíme rozebrat obecné fungování pravidel.

\subsection{Zálohovací pravidla určená podle cest}

Pravidla je možné specifikovat pro jakoukoliv cestu v zálohovaných datech
(kořenová složka zálohy je označována pomocí lomítka: \texttt{/}) a stejnou
cestu je možné uvést i vícekrát -- v takovém případě platí pro tuto cestu
poslední specifikovaná pravidla.

Pokud pro nějakou cestu nejsou specifikována pravidla (v reálném případě to bude
pravděpodobně většina cest), použijí se pro ní nejbližší vyšší pravidla po cestě
ke kořeni. Je možné tak třeba zakázat zálohování určitých složek.

\medskip

\noindent{\bf Přehled a význam parametrů, které lze nastavovat:}
\begin{itemize}
	\item {\tt scan <true|false>}: Platí jen pro složky a říká, jestli vůbec
	tuto složku prozkoumávat. Defaultně {\tt true}.
	\item {\tt backup <true|false>}: Jestli zadaný soubor zálohovat, nebo
	ne. Defaultně {\tt true}.
	\item {\tt history <1-10>}: Parametr pro funkci na ředění historie, čím
	vyšší číslo, tím více verzí držet. Defaultně 1.
	\item {\tt priority <1-10>}: Parametr pro prioritizační funkci. Určuje,
	které soubory zálohovat dříve, než ostatní. Čím vyšší číslo, tím
	důležitější. Defaultně 1.
\end{itemize}

\subsection{Zálohovací pravidla podle specifičtějšího výběru}

Mimo pravidel odvozených jen od cest existují také pravidla umožňující pokrýt
třeba jen určitý typ souboru. Jsou také vázána na konkrétní cestu (a platí tak
jen v určeném podstromě), ale je možné u nich zavést složitější způsoby
filtrování.

Pro každou cestu je možné v konfiguraci uvést seznam {\it pravidel pro soubory},
kde každé pravidlo má parametry, které nastavuje, a pak filtry určující, na jaké
soubory se má aplikovat. Je možné filtrovat podle následujících parametrů:

\begin{itemize}
	\item Regex\footnote{regulární výraz} názvu souboru (\texttt{regex}) --
	pokud je nevyplněný či prázdný, odpovídá jakýkoliv soubor, pokud je
	neprázdný, musí celý název souboru (jen souboru, bez cesty) odpovídat
	zadanému regexu.
	\item Regex cesty k souboru (\texttt{path\_regex}) -- podobně jako výše,
	nevyplněnému odpovídá vše, vyplněnému jen soubory, které se nacházejí
	(oproti aktuálnímu adresáři) na cestě odpovídající regexu.
	\item Minimální a maximální velikost souboru v bytech (parametry
	\texttt{size\_at\_least} a \texttt{size\_at\_most}).
\end{itemize}

Jednotlivá pravidla se vždy aplikují v pořadí, v jakém jsou uvedena v konfiguraci,
a funguje zde princip přepisování (tedy pokud je splněna podmínka pravidla, jsou
přepsány všechny parametry zálohování, které pravidlo nastavuje).

\medskip\goodbreak

\noindent Při dotazu na konkrétní soubor se pak parametry pravidel určí takto:
\begin{enumerate}
	\item Vezmou se defaultní hodnoty pravidel a začne se v kořenové složce.
	\item V aktuální složce se nejdříve aplikují pravidla složky (pokud
	existují) a poté se projdou všechny pravidla pro soubory:
	\item Pokud vyhovuje filtr u pravidla pro soubory, aplikuje se.
	\item Pokud se dá po zadané cestě sestoupit ještě o úroveň níže, sestoupí
	se a opakuje se znovu od bodu 2.
\end{enumerate}

Nakonec se vrátí finální pravidla vzniklá aplikací všech pravidel, přes která se
prošlo. Toto řešení nabízí velmi silné možnosti konfigurace, bohužel cacheování
výsledků je zde vzhledem k pravidlům s filtry značně obtížné, pro běžně velké
konfigurační soubory by ale rychlost i tak měla být dostatečná a úzkým hrdlem
zde stále bude rychlost načítání dat z disku.

\section{Model ukládání dat}

FenixBackup představuje každou zálohu (ať už se v rámci ní zálohovaly všechny
soubory, nebo se povedlo zálohovat pouze jediný) jedním souborovým stromem. Ten
obsahuje všechny záznamy o všech souborech, které by se měly dostat do zálohy.

Při svém vzniku obsahuje tento strom jen informace zjistitelné o souborech bez
nutnosti číst jejich obsah (to zajistí příslušný adaptér, viz následující
podkapitola) a postupně se zpracováním obsahu jednotlivých souborů se v něm
objevují i kontrolní součty (hashe) souborů, které odkazují na konkrétní datové
bloky.

\newacronym{SHA256}{SHA256}{Kryptografická hashovací funkce, vytváří otisk o
velikosti 256 bitů}

Každý uložený soubor je jednoznačně identifikován svým \gls{SHA256} hashem
a představován stejnojmenným datovým blokem. Tento datový blok může být buď
úplný, nebo je to jen \gls{VCDIFF} oproti jinému datovému bloku.

Pro fungování zálohovacího systému předpokládáme, že nalezení kolize v
kryptografické hashovací funkci jako je \gls{SHA256} je extrémně
nepravděpodobné (zatím není známý ani cílený útok, který by uměl vyprodukovat
kolizi, a velikost kolizní domény je $2^{256}$). Na stejném předpokladu
o nepravděpodobných kolizích v reálném nasazení staví i verzovací systém Git,
který navíc používá jen kratší hashe generované \gls{SHA1}.

Díky tomu, že jsou datové bloky pojmenovány hashem souboru, který mají
představovat, řeší se velmi efektivním způsobem duplikáty souborů. Ačkoliv třeba
dva stejné soubory mohou mít úplně rozdílnou historii, tak finálně je uložen
v zálohovacím systému jen jeden datový blok, který představuje oba soubory.

Datové bloky a jejich případné rozdílové návaznosti tedy vůbec nemusí
korespondovat s historií jednotlivých souborů uloženou v souborových stromech.
Ale díky tomu, že v souborovém stromu je uložený hash každé jednotlivé verze
souboru, lze všechny historické verze snadno zrekonstruovat.

Záznam o každém jednotlivém souboru se může navíc nacházet v jednom z
následujících stavů:
\begin{itemize}
	\item\texttt{UNKNOWN} -- Nový soubor (bez známé minulé verze), u kterého
	zatím nemáme zálohovaný jeho obsah.
	\item\texttt{NEW} -- Nový soubor (bez známé minulé verze), u kterého již
	máme uložený a zpracovaný jeho obsah.
	\item\texttt{UNCHANGED} -- Soubor se známou minulou verzí, který se
	nezměnil (ani parametry, ani obsahem).
	\item\texttt{UPDATED\_PARAMS} -- Soubor se známou minulou verzí, u kterého
	se změnily pouze parametry, ale obsah ne.
	\item\texttt{NOT\_UPDATED} -- Soubor se známou minulou verzí, u kterého
	se změnil i obsah, a jenž ještě nemáme uložený.
	\item\texttt{UPDATED\_FILE} -- Soubor se známou minulou verzí, u kterého
	se změnil i obsah, a jenž již máme uložený a zpracovaný.
	\item\texttt{DELETED} -- Soubor byl z nějakého důvodu vymazán (ať už na
	přímé přání uživatele, nebo automaticky při uvolňování místa).
\end{itemize}

\section{Adaptéry pro přístup k zálohovaným datům}

Zálohovací systém je psán tak, aby byl nezávislý na metodě přístupu k zálohovaným
datům. Samotnému jádru je jedno, jak data získá, o to se starají takzvané
{\it adaptéry}.

V úloze zálohování se dají jejich úkoly rozdělit do dvou fází. V první fázi
získávají adaptéry informace o souborech a předávají je jádru zálohovacího
systému, ve druhé fázi jim pak jádro zálohovacího systému dodá seznam souborů,
které chce získat, a adaptér nějakým způsobem opatří jejich obsah a dodá je
nazpět jádru.

První fáze začíná tím, že adaptér začne skenovat systém souborů od zadaného
místa, načítá parametry souborů (vlastník, práva, velikost, \dots) a ukládá tyto
záznamy do souborového stromu. Jádro zálohovacího systému všechny ukládané
záznamy zpracovává a může případně operativně rozhodnout, že ho informace v nějakém
podstromě již nezajímají (třeba v případě nastavení \texttt{scan: false} pro
nějakou cestu v konfiguraci). V takovém případě to dá domluveným způsobem vědět
adaptéru a ten by již tento podstrom neměl dále procházet.

Během této fáze se jádro zálohovacího systému pokouší (podle cesty, času poslední
modifikace a velikosti souborů) rozhodnout, které soubory se nezměnily, u kterých
se změnily pouze parametry, a které soubory bude potřeba stáhnout pro uložení
jejich nové verze.

Seznam požadovaných souborů (utříděný podle prioritizační funkce, viz dále) pak
předá nazpět adaptéru a ten by měl zařídit získání obsahu těchto souborů a
předání obsahu jádru zálohovacího systému.

To si spočte hash souborů, zjistí, zdali případně nemá již soubor se stejným
hashem uložený, a pokud ne, uloží nový datový blok (buď jako kompletní datový
blok, nebo jako \gls{VCDIFF} oproti jinému).

\section{Priorita zálohovaných souborů}

Různé soubory mohou mít konfigurací jinak nastavené priority zálohování, čímž
lze zálohovacímu systému sdělit, o které soubory máme větší zájem.

Do rozhodovacího procesu se také promítá to, jakou nejmladší verzi daného
souboru již máme uloženou. Systém je totiž připravený i na to, že ne vždy se
povede provést kompletní zálohu, například u přenosných počítačů připojených k
rychlé síti jen po omezenou dobu. V takovém případě zůstanou některé soubory ve
stromu se stavem \texttt{UNKNOWN} nebo \texttt{NOT\_UPDATED} a další zálohy na
to berou ohled (soubory, jež se nepovedlo zálohovat delší dobu, mají při příštím
zálohování přednost).

Konkrétně si jádro zálohovacího systému přiděluje každému souboru číselné
{\it skóre} odpovídající následujícímu vztahu:

$$[\hbox{čas od poslední zálohované verze}]\times[\hbox{priorita souboru}]$$

Pokud soubor nemá známou minulou verzi, tak se jako čas poslední zálohované
verze bere čas minulého spuštění zálohování -- protože tento nový soubor se mohl
objevit kdykoliv v mezičase mezi touto poslední zálohou a současností.

Tímto postupem by se mělo zajistit zálohování postupně všech souborů i v případě,
že zálohovaný stroj bývá připojen k zálohovacímu systému pravidelně jen po
omezenou dobu.

\chapter{Uživatelská dokumentace}

FenixBackup se v~současnosti ovládá z~příkazové řádky, ale do budoucna není
problém nad samotným jádrem postavit více jinak fungujících rozhraní (třeba
GUI). Pro jeho fungování je nezbytný konfigurační soubor, který popisuje co,
jakým způsobem a kam se má zálohovat.

\section{Konfigurační soubor}

Konfigurační soubor se vztahuje vždy k~jednomu zálohovanému stroji a obsahuje
tři části. První specifikuje úložiště zálohovaných dat, druhá z~nich popisuje
způsob připojení k~zálohovanému stroji -- volba adaptéru a parametry pro něj
(například cesta v~rámci lokálního souborového systému, nebo adresu a port
stroje, na který se připojit) a poslední a nejobsáhlejší nakonec popisuje
zálohovací pravidla pro jednotlivé soubory.

Při spuštění zálohovacího systému je nejdříve načtena celá konfigurace a teprve,
když se jí povede celou zpracovat bez chyby, přistupuje se k~provádění příkazů.

\subsection{Ukázková konfigurace}

Konfigurace pro zápis používá v~mnoha projektech rozšířený zápis poskytovaný
knihovnou {\it libconfig}.%
\footnote{\url{http://www.hyperrealm.com/libconfig/libconfig_manual.html}}

Protože je velmi časté používat společná zálohovací pravidla pro více strojů,
dají se tato společná pravidla sdružit do společných souborů a ty pomocí
direktivy \texttt{@include} v~konfiguraci používat opakovaně na více místech.

Ukázka je připojena níže.

\begin{verbatim}
baseDir: "./backup_dir"
dataSubdir: "data"
treeSubdir" "trees"

adapter: {
  type: "local_filesystem"
  path: "/home/jirka/fenix_backup"
}

paths: (
  { path: "/"
    file_rules: (
      @include "no_exe.config"
    )
  },
  {
    path: "/.git"
    scan: false
  },
  { path: "/test_backup/"
    scan: true
    file_rules: (
      { backup: false },
      { regex: ".*\.fenixtree"
        size_at_least: 10
        size_at_most: 10485760
        backup: true
      }
    )
  }
)

# Obsah souboru no_exe.config:
{ regex: ".*\.exe"
  backup: false
}
\end{verbatim}

První položka \texttt{baseDir} je povinná a popisuje, kam přesně má zálohovací
systém ukládat svá data. Pak lze volitelně modifikovat názvy podsložek na
ukládání dat souborů a souborových stromečků (\texttt{dataSubdir}
a \texttt{treeSubdir}).

Druhá položka \texttt{adapter} specifikuje způsob získávání zálohovaných dat.
V~první řadě je specifikováno, který adaptér se má použít, a zbytek položek je
pak předán tomuto adaptéru jako jeho nastavení.

Poslední části konfigurace popisuje pravidla pro cesty a specifičtější pravidla
určovaná filtry, jejichž fungování jsme rozebrali v~minulé kapitole.

\section{Použití z~příkazové řádky}

FenixBackup je psaný primárně pro spouštění příkazů z~příkazové řádky s~cílem
umožnit snadné plánování akcí třeba pomocí \texttt{cron}u.

Základem je spuštění příkazu \texttt{fenix}, kterému se vždy předá cesta ke
konfiguračnímu souboru a poté příkaz, který chceme provést. Inspirací pro
rozhraní bylo zčásti rozhraní verzovacího systému Git (viz \cite{progit}).

Pomocí tohoto rozhraní lze provádět všechny potřebné akce:
\begin{itemize}
	\item Zobrazit seznam záloh
	\item Zobrazit seznam souborů v záloze
	\item Sledovat historii jednoho konkrétního souboru
	\item Provést zálohu nebo čištění
	\item Obnovit jeden soubor, podstrom nebo všechny soubory z určené zálohy
	buď do původního místa, nebo do specifikované cesty
\end{itemize}

\newpage
\subsection{Popis příkazů}

Přesné použití rozhraní nejlépe popisuje usage-note příkazu \texttt{fenix}:
\begin{verbatim}
Usage: ./fenix <config_file>
And one of these commands:
  show backups			(displays list of all backups)
  show files [<backup>]		(displays list of all files in given backup)
  show history <backup> <path>	(displays known history of given file)
  backup			(run backup)
  restore full <backup>		(run full restore to original path)
  restore full <backup> <path>	(run full restore to given path)
  restore subtree <backup> <subtree_path>
				(restore subtree to original path)
  restore subtree <backup> <subtree_path> <path>
				(restore subtree to given path)
  restore file <backup> <file_path>
				(restore one file to original path)
  restore file <backup> <file_path> <path>
				(restore one file to given path)
  cleanup [<x>]			(run <x> rounds of cleanup, default 1)
\end{verbatim}


% Ukázka použití některých konstrukcí LateXu (odkomentujte, chcete-li)
% \include{ukazka}

\chapter*{Závěr}
\addcontentsline{toc}{chapter}{Závěr}

Zálohovací systém FenixBackup se pokusil nabídnout další řešení problému
zálohování dat. V~ideálním světe by taková věc vůbec potřeba nebyla, bohužel náš
svět ideální není, ke ztrátám či poškození dat v~něm dochází a systémy jako
FenixBackup jsou potřeba.

Častým problémem zálohování je neochota uživatelů strávit čas s~jeho
nastavováním, nebo jeho složité použití ve chvíli ztráty dat. Proto jsem se
pokusil při psaní FenixBackup vycházet z~principů držet pro uživatele vše co
možná nejpřímočařejší.

Současně jsem se pokusil vyřešit i problém omezeného místa pro zálohy, což
vnímám jako další důvod, proč se někdy se zálohováním vyskytují problémy. Často
uživatelé jednou nakonfigurují zálohování a pak věří tomu, že už bude fungovat
napořád -- bohužel se stává, že zálohovacímu systému dojde místo a uživatelé
tomu nevěnují pozornost. FenixBackup se k~tomu staví tak, že má nastavený limit
pro velikost a když je překročen, spustí čištění záloha a maže zálohované verze
souborů s~největším spočteným záporným skóre, dokud nevymaže dostatek dat.

Vytvořený zálohovací systém je doufám na počátku dalšího slibného vývoje a~zkusím
zde nastínit, jakými dalšími směry by se jeho vývoj mohl ubírat.

\newacronym{SMB}{SMB}{Server Message Block, síťový protokol (nejen) pro přenos
souborů, implementace například projektem Samba}

Jedním směrem vývoje je určitě přidání většího množství adaptérů pro získávání
dat. V~prvotní fázi vývoje byl vytvořen jen adaptér pro lokální souborový
systém, ale další adaptéry se nabízejí:
\begin{itemize}
	\item Adaptér připojující vzdálený souborový systém pomocí \gls{SSHFS}
	a dále fungující jako adaptér pro lokální souborový systém.
	\item Adaptér využívající jako protistranu na zálohovaném stroji a jako
	přenosový protokol \texttt{rsync}.
	\item Adaptér připojující se přes protokol \gls{FTP} nebo \gls{SMB} na
	zálohovaný stroj.
	\item Adaptér využívající na zálohovaném stroji běžícího vlastního
	klienta.
\end{itemize}

\newacronym{ACL}{ACL}{Access control list, rozšířený systém práv}

Dalším možným směrem vývoje je modifikovat datový formát tak, aby mohl obsahovat
nepovinné rozšiřující položky -- inspirace například hlavičkami IPv6 paketů,
které obsahují základní společnou hlavičku a pak podle potřeby rozšiřitelné
hlavičky. V~těchto nepovinných položkách by mohl sídlit například systém práv
\gls{ACL}, který se u~nějakých souborů vyskytuje, ale je zbytečné mít pro něj
vyhrazené pevné datové položky u~každého souboru.

Nápad na to použít takový systém ukládání dat přišel bohužel až v~pozdější fázi
vývoje systému, kdy již na předělávání současné implementace nebyl dostatek
času, ale je to jeden z~cílů, kterým se chci dále věnovat.

Jinou oblastí, ve které se může udělat ještě velký pokrok, je přidat více
inteligence do hledání minulých verzí souborů, nebo do hledání vhodných datových
bloků, na které navázat pomocí rozdílové zálohy. Zde je podle mého ještě velký
prostor, kterým se může systém posunout. Implementace jádra systému je na to
připravená, protože už nyní umožňuje specifikovat, proti kterému datovému bloku
má vznikat rozdílová záloha (i když zatím se volí buď předchozí známá verze
souboru, nebo prázdný soubor).

Poslední oblast, nad kterou aktuálně přemýšlím a která by neměla být příliš
složitá k~doimplementaci, je nasazení v~prostředí zálohování více podobných
strojů (typicky nějaká firma nebo škola). Dá se vypozorovat, že při zálohování
většího množství podobných strojů je mnoho souborů napříč stroji stejných
(systémové soubory, společná konfigurace aj.). V~takovém prostředí by bylo velmi
efektivní umět mezi zálohami různých strojů sdílet společná data a zmenšit tak
celkovou velikost všech záloh.

Pokud bychom deaktivovali systém mazání starých záloh, umí to FenixBackup již
v~současném stavu -- jednotlivé zálohy budou sdílet společnou datovou složku,
ale budou mít jiné složky, kam ukládají souborové stromy. Protože jsou všechny
informace, jako jsou práva, uloženy v~souborových stromech, a protože jsou
datové bloky identifikovány hashem souboru, který zastupují, tak je možné bez
jakýkoliv problémů tyto datové bloky sdílet. Problém nastane jen, kdyby jedna
ze záloh chtěla datové bloky smazat jako nepotřebné -- v~tu chvíli by bylo
potřeba sledovat využití datových bloků napříč všemi zálohami ukládajícími do
stejného místa, pro což je nutné přidat podporu (aby zálohy o~sobě navzájem
věděly).

FenixBackup je připraven k~použití již v~současném stavu, ale pevně věřím,
že jeho vývoj bude dále pokračovat a dočká se rozšíření ve výše zmíněných
oblastech a také časem většího nasazení, než jen v~řádu jednotek zálohovaných
strojů.


%%% Seznam použité literatury
%%% Seznam použité literatury je zpracován podle platných standardů. Povinnou citační
%%% normou pro bakalářskou práci je ISO 690. Jména časopisů lze uvádět zkráceně, ale jen
%%% v kodifikované podobě. Všechny použité zdroje a prameny musí být řádně citovány.

\def\bibname{Seznam použité literatury}
\begin{thebibliography}{99}
\addcontentsline{toc}{chapter}{\bibname}

\bibitem{progit}
  {\sc Chacon,} Scott.
  \emph{Pro Git}.
  Praha: CZ.NIC, 2009.
  ISBN 978-80-904248-1-4.

\bibitem{rsync}
  {\sc Tridgel,} Andrew et al.
  \emph{Rsync}
  [online]. The Australian National University 1999.
  [Cit. 28.7.2015]
  Dostupné z \url{http://rsync.samba.org/}.

\bibitem{libconfig}
  {\sc Lindner,} Mark.
  \emph{libconfig -- C/C++ Configuration File Library}
  [online].
  [Cit. 28.7.2015]
  Dostupné z \url{http://www.hyperrealm.com/libconfig/}

\bibitem{cereal}
  {\sc Grant,} Shane a {\sc Voorhies,} Randolph.
  \emph{Cereal -- A C++11 library for serialization}
  [online]. University of Southern California.
  [Cit. 28.7.2015]
  Dostupné z \url{http://uscilab.github.io/cereal/}

\bibitem{sha256}
  {\sc Brumme,} Stephan.
  \emph{Portable C++ Hashing Library}
  [online].
  [Cit. 28.7.2015]
  Dostupné z \url{http://create.stephan-brumme.com/hash-library/}

\bibitem{open-vcdiff}
  {\sc Google}.
  \emph{open-vcdiff}
  [online].
  [Cit. 28.7.2015]
  Dostupné z \url{https://code.google.com/p/open-vcdiff/}

\bibitem{boost}
  {\sc Boost community}.
  \emph{Boost Filesystem Library Version 3}
  [online].
  [Cit. 28.7.2015]
  Dostupné z \url{http://www.boost.org/doc/libs/1_46_0/libs/filesystem/v3/doc/index.htm}

\end{thebibliography}


%%% Tabulky v bakalářské práci, existují-li.
\chapwithtoc{Seznam tabulek}

%%% Použité zkratky v bakalářské práci, existují-li, včetně jejich vysvětlení.
% \chapwithtoc{Seznam použitých zkratek}
\advance\glsdescwidth by 2cm
\printglossary[type=\acronymtype,title={Seznam použitých zkratek}]

%%% Přílohy k bakalářské práci, existují-li (různé dodatky jako výpisy programů,
%%% diagramy apod.). Každá příloha musí být alespoň jednou odkazována z vlastního
%%% textu práce. Přílohy se číslují.
\chapwithtoc{Přílohy}

{\bf Příloha 1:} Zdrojové kódy zálohovacího systému FenixBackup (včetně ukázkové
konfigurace a několika pomocných souborů) -- přiloženo k práci a dostupné online
na TODO.

\openright
\end{document}
