%%% Hlavní soubor. Zde se definují základní parametry a odkazuje se na ostatní části. %%%

%% Verze pro jednostranný tisk:
% Okraje: levý 40mm, pravý 25mm, horní a dolní 25mm
% (ale pozor, LaTeX si sám přidává 1in)
\documentclass[12pt,a4paper]{report}
\setlength\textwidth{145mm}
\setlength\textheight{247mm}
\setlength\oddsidemargin{15mm}
\setlength\evensidemargin{15mm}
\setlength\topmargin{0mm}
\setlength\headsep{0mm}
\setlength\headheight{0mm}
% \openright zařídí, aby následující text začínal na pravé straně knihy
\let\openright=\clearpage

%% Pokud tiskneme oboustranně:
% \documentclass[12pt,a4paper,twoside,openright]{report}
% \setlength\textwidth{145mm}
% \setlength\textheight{247mm}
% \setlength\oddsidemargin{15mm}
% \setlength\evensidemargin{0mm}
% \setlength\topmargin{0mm}
% \setlength\headsep{0mm}
% \setlength\headheight{0mm}
% \let\openright=\cleardoublepage

%% Pokud používáte csLaTeX (doporučeno):
%\usepackage{czech}
\usepackage[czech]{babel}
\usepackage[T1]{fontenc}
\usepackage{lmodern}

% Glyphtounicode cause unicode text easily searchable a copyable from PDF
\input{glyphtounicode}
\pdfgentounicode=1

%% Použité kódování znaků: obvykle latin2, cp1250 nebo utf8:
\usepackage[utf8]{inputenc}
%% Ostatní balíčky
\usepackage{graphicx}
\usepackage{amsthm}
\usepackage{array}
\usepackage[footnote,acronym,nomain]{glossaries}
%\setglossarystyle{altlist}
\setglossarystyle{list}

\makeglossaries

%% Balíček hyperref, kterým jdou vyrábět klikací odkazy v PDF,
%% ale hlavně ho používáme k uložení metadat do PDF (včetně obsahu).
%\usepackage{url}
\usepackage[pdftex,unicode]{hyperref}   % Musí být za všemi ostatními balíčky
\hypersetup{pdftitle=Zálohovací systém}
\hypersetup{pdfauthor=Jiří Setnička}

%%% Drobné úpravy stylu

% Tato makra přesvědčují mírně ošklivým trikem LaTeX, aby hlavičky kapitol
% sázel příčetněji a nevynechával nad nimi spoustu místa. Směle ignorujte.
\makeatletter
\def\@makechapterhead#1{
  {\parindent \z@ \raggedright \normalfont
   \Huge\bfseries \thechapter. #1
   \par\nobreak
   \vskip 20\p@
}}
\def\@makeschapterhead#1{
  {\parindent \z@ \raggedright \normalfont
   \Huge\bfseries #1
   \par\nobreak
   \vskip 20\p@
}}
\makeatother

% Toto makro definuje kapitolu, která není očíslovaná, ale je uvedena v obsahu.
\def\chapwithtoc#1{
\chapter*{#1}
\addcontentsline{toc}{chapter}{#1}
}

%%%%%%%%%%%%%%%%%%%%%%
\def\Cpp{C{\tt++}}
%%%%%%%%%%%%%%%%%%%%%%

\begin{document}

% Trochu volnější nastavení dělení slov, než je default.
\lefthyphenmin=2
\righthyphenmin=2

%%% Titulní strana práce
\setcounter{page}{-5} %% Hack kvůli generování obsahu a odkazům na stránky, aby se strana 1 neodkazovala sem

\pagestyle{empty}
\begin{center}

\large

Univerzita Karlova v Praze

\medskip

Matematicko-fyzikální fakulta

\vfill

{\bf\Large BAKALÁŘSKÁ PRÁCE}

\vfill

\centerline{\mbox{\includegraphics[width=60mm,type=pdf,ext=.epdf,read=.epdf]{mfflogo}}}

\vfill
\vspace{5mm}

{\LARGE Jiří Setnička}

\vspace{15mm}

% Název práce přesně podle zadání
{\LARGE\bfseries Zálohovací systém}

\vfill

% Název katedry nebo ústavu, kde byla práce oficiálně zadána
% (dle Organizační struktury MFF UK)
Katedra aplikované matematiky

\vfill

\begin{tabular}{rl}

Vedoucí bakalářské práce: & Mgr. Martin Mareš, Ph.D. \\
\noalign{\vspace{2mm}}
Studijní program: & Informatika \\
\noalign{\vspace{2mm}}
Studijní obor: & Obecná informatika \\
\end{tabular}

\vfill

% Zde doplňte rok
Praha 2015

\end{center}

\newpage

%%% Následuje vevázaný list -- kopie podepsaného "Zadání bakalářské práce".
%%% Toto zadání NENÍ součástí elektronické verze práce, nescanovat.

%%% PODĚKOVÁNÍ:
%%% Na tomto místě mohou být napsána případná poděkování (vedoucímu práce,
%%% konzultantovi, tomu, kdo zapůjčil software, literaturu apod.)
\openright

\noindent
Rád bych poděkoval mému vedoucímu práce Mgr. Martinu Marešovi, Ph.D. za
nadnesení tématu, které mě zaujalo, a za jeho konzultace a rady k vypracování
práce. Těším se na případnou další spolupráci s rozvíjením vzniklého
zálohovacího systému. Současně děkuji také bezpočtu dalších, kteří mi nějakou
radou, myšlenkou, co by systém mohl umět, nebo jinou podporou pomohli práci
dokončit.

\newpage

%%% Strana s čestným prohlášením k bakalářské práci

\vglue 0pt plus 1fill

\noindent
Prohlašuji, že jsem tuto bakalářskou práci vypracoval samostatně a výhradně
s~použitím citovaných pramenů, literatury a dalších odborných zdrojů.

\medskip\noindent
Beru na~vědomí, že se na moji práci vztahují práva a povinnosti vyplývající
ze zákona č. 121/2000 Sb., autorského zákona v~platném znění, zejména skutečnost,
že Univerzita Karlova v Praze má právo na~uzavření licenční smlouvy o~užití této
práce jako školního díla podle §60 odst. 1 autorského zákona.

\vspace{10mm}

\hbox{\hbox to 0.5\hsize{%
V \hbox to 2cm{\dotfill} dne \hbox to 2cm{\dotfill}
\hss}\hbox to 0.5\hsize{%
%Podpis autora
\hss}}

\vspace{20mm}
\newpage

%%% Povinná informační strana bakalářské práce

\vbox to 0.5\vsize{
\setlength\parindent{0mm}
\setlength\parskip{5mm}

{\bf Název práce:}
Zálohovací systém

{\bf Autor:}
Jiří Setnička

{\bf Katedra:}
Katedra aplikované matematiky

{\bf Vedoucí bakalářské práce:}
Mgr. Martin Mareš, Ph.D., Katedra aplikované matematiky

{\bf Abstrakt:}
Přehled existujících přístupů k zálohování dat a zvážení jejich předností
s~ohledem na použití. S uvážením předchozího poté popis souběžně s~touto prací
vzniklého zálohovacího systému FenixBackup jako souborově orientovaného,
schopného si držet více historických verzí a pracujícího s~rozdílovými zálohami.
V závěru je nastíněn možný směr, kterým se může vývoj systému ubírat.
% abstrakt v rozsahu 80-200 slov; nejedná se však o opis zadání bakalářské práce

{\bf Klíčová slova:} zálohování, diff, obnova dat
% 3 až 5 klíčových slov

\vss}\nobreak\vbox to 0.49\vsize{
\setlength\parindent{0mm}
\setlength\parskip{5mm}

{\bf Title:}
A backup system

{\bf Author:}
Jiří Setnička

{\bf Department:}
Department of Applied Mathematics
% dle Organizační struktury MFF UK v angličtině

{\bf Supervisor:}
Mgr. Martin Mareš, Ph.D., Department of Applied Mathematics

{\bf Abstract:}
An overview of existing approaches to data backup and their strengths with
regard to use. After considering the previous describe the resulting backup
system FenixBackup as file-oriented differential
backup system capable holding more historical versions.
In conclusion, outlines the possible directions which the development of
the system could follow.
% abstrakt v rozsahu 80-200 slov v angličtině; nejedná se však o překlad
% zadání bakalářské práce

{\bf Keywords:} backup, diff, data recovery
% 3 až 5 klíčových slov v angličtině

\vss}

\newpage

%%% Strana s automaticky generovaným obsahem bakalářské práce. U matematických
%%% prací je přípustné, aby seznam tabulek a zkratek, existují-li, byl umístěn
%%% na začátku práce, místo na jejím konci.

\openright
\pagestyle{empty}
\tableofcontents
\addtocontents{toc}{\protect\thispagestyle{empty}} % HACK

\newpage
\setcounter{page}{1}
\pagestyle{plain}

%%% Jednotlivé kapitoly práce jsou pro přehlednost uloženy v samostatných souborech
\chapter*{Úvod}
\addcontentsline{toc}{chapter}{Úvod}

Potřeba zálohovat data se pojí s~počítačovým světem prakticky od jeho vzniku
a vychází z~dob ještě před jeho existencí -- z~potřeby udržovat kopie například
důležitých dokumentů a chránit je tak před náhodným nebo cíleným zničením nebo
ztrátou.

V~dřívějších dobách stačilo pořídit fyzickou kopii důležitého dokumentu a ten
uložit ideálně na jiném fyzickém místě, s~nástupem počítačového ukládání dat se
k~tomu však ještě přidal aspekt toho, že data je možná snadno editovat a měnit.

Pokud tedy v~současné době mluvíme o~zálohování, je potřeba toto brát v~potaz
a přizpůsobit tomu strategie zálohování. Nestačí pouze pořídit kopii dat ve
chvíli jejich vzniku, ale je nutné dívat se na data jako na dynamicky se měnící
objekt, u~kterého chceme tyto změny sledovat, a mít možnost vrátit se
k~dřívějšímu stavu, pokud je to zapotřebí.

Pokud si zkusíme rozebrat situace, před kterými by nás zálohovací systém měl
chránit, napadne nás pravděpodobně následující:

\newacronym{RAID}{RAID}{Redundant Array of Inexpensive/Independent Disks -- systém
ukládání dat na více nezávislých disků}

\begin{enumerate}
	\item Závada pevného disku či jiného ukládacího média -- Před tímto nás
	zálohování ochrání možností obnovy poslední zálohované verze dat, ale
	v~případě běžného diskového úložiště se zde nabízí ještě jiná lepší
	alternativa a to použít nějakou variantu \gls{RAID}.
	\item Ztráta, odcizení nebo zničení celého počítače (živelná katastrofa)
	-- Zde nám \gls{RAID} typicky nepomůže, protože jednotlivé disky v~něm
	bývají umístěny na stejném fyzickém místě. Data se dají zachránit
	z~poslední provedené zálohy (zde velmi záleží na politice zálohování,
	jak je záloha aktuální).
	\item Vrácení se do stavu před provedením nějaké akce -- Toto se hodí
	například při neúmyslném smazání některých souborů, nebo při nepovedené
	úpravě systému (instalace nového software aj.). Zde nám naopak techniky
	jako \gls{RAID} nepomohou, protože ty se starají jen o~redundanci
	nejaktuálnější verze uložených dat a nemají historii.
	\item Odhalování útoků na zálohovaný stroj -- Při podezřelé aktivitě na
	zálohovaném stroji lze porovnáním změn v~souborech zjistit, jestli nebyl
	tento stroj napaden (a nebyly například provedeny změny na systémových
	souborech) a v~jakém časovém období se tak případně stalo.
\end{enumerate}

Pro první a druhý případ nám stačí držet si vždy jen poslední verzi souborů.
Pokud ale chceme, aby náš zálohovací systém pokrýval i situace popsané ve třetím
a čtvrtém bodě, je potřeba si nějakým způsobem udržovat více historických verzí
a umět si vyvolat stav dat libovolné z~těchto verzí, případně si mezi libovolnými
dvěma zobrazit rozdíly.

Dalším důležitým aspektem zálohovacího systému diskové místo potřebné
k~zálohování zálohovaného stroje. S~nějakým potřebným místem je nutné počítat, ale
přílišné plýtvání (například způsobené ukládáním celé nové verze souboru při
jeho drobné změně) není na místě. Stejně tak je většinou potřeba \uv{ředit}
staré verze záloh a držet si jen ty významné.

Neméně důležitou věcí je to, že ke všem datům také nepřistupujeme stejně. Některá
data jsou pro nás typicky významnější a je pro nás důležitější zálohovat změny
v~nich prováděné, jiná data tak významná nejsou a není třeba nutné držet si od
těchto dat takové množství historických verzí.

Poslední zásadní věcí je pak pro nás čas provedení zálohy. U~strojů stále
připojených k~rychlé síti to není až tak důležité, ale u~přenosných počítačů,
které jsou k~rychlé síti připojeny jen omezenou dobu nebo u~strojů připojených
po pomalé síti je toto důležité. Zde přichází ke slovu jednak správné sledování
a přenášení jen těch dat, která se změnila, tak prioritizace dat na základě
jejich významu pro nás, jak bylo popsáno v~předchozím odstavci.

Na všechny tyto aspekty jsem se pokoušel při vývoji svého zálohovacího systému
(nazvaného podle bájného mýtického ptáka postávajícího po svém zničení z~popela,
stejně jako to chceme od zálohovaných dat) brát zřetel a zohlednit je. Vznikl
tak zálohovací systém {\it FenixBackup}.


\chapter{Existující metody a systémy zálohování}

Jak již bylo popsáno v úvodu práce, dá se k zálohování přistupovat mnoha způsoby.
Pojďme si zkusit tyto způsoby rozebrat, podívat se na ně z několika hledisek
efektivity a zhodnotit jejich klady a zápory. Dále se zkusme podívat na
několik existujících zálohovacích řešení.

\section{Jednoduché kopírování}

Při zálohování vždy jde o to držet si nějakým způsobem kopii nějakých dat. Výběr
dat k zálohování nyní odložíme stranou a budeme si představovat, že máme nějakou
dobře definovanou množinu souborů, které chceme zálohovat.

\newacronym{FTP}{FTP}{File Transfer Protocol -- jednoduchý síťový protokol pro
přenos souborů}
\newacronym{SSH}{SSH}{Secure shell -- zabezpečený komunikační síťový protokol}
\newacronym{SCP}{SCP}{Secure copy -- zabezpečený protokol pro přenos souborů na
bázi \gls{SSH}}

Nejjednodušším způsobem co do implementace je asi {\bf ruční kopírování} --
uživatel sám čas od času zkopíruje zálohované soubory na nějaké jiné úložiště.
Na tomto úložišti si uživatel může držet buď jen jednu kopii dat (vždy jen
nejnovější verzi), což zabere řádově tolik místa, kolik zabírají zálohovaná data
na zálohovaném stroji, nebo si může držet všechny historické verze. V takovém
případě ale zálohy zaberou místo $[\hbox{počet záloh}]\times[\hbox{velikost dat}]$,
což je pro časté zálohování velkého objemu dat neudržitelné (objem dat lze sice
snížit vhodnou kompresí, ale to oddálí problém jen o kus dále).

Jako metodu kopírování lze zvolit mnoho protokolů, od kopírování po místním
souborovém systému (což nebývá moc účinné proti selhání celého počítače nebo
jeho ztrátou), přes kopírování přes síť pomocí \gls{FTP} nebo nad \gls{SSH}
postaveným \gls{SCP}.

Všechny tyto metody kopírování jsou ale \uv{hloupé}, udělají jen přesně to, co
od nich žádáme -- zkopírují zadané soubory na cílové úložiště bez starosti o to,
jestli již zde stejné soubory nebyly kopírovány dříve. Pokud byly provedeny jen
malé změny, tak přenášíme zbytečně velké objemy dat, které již v cílovém místě
uložená jsou.

Kdybychom dokázali dostatečně bezpečně poznat, jaká data se nezměnila, mohli
bychom objem přenášených dat výrazně snížit a kopírovat jen rozdílná data.
Když zálohování provádí uživatel sám a ručně, může posloužit jako rozhodovací
veličina v tom, která data se podle jeho mínění změnila a která je tedy potřeba
zálohovat. To je ale jednak nespolehlivé a může to vést k chybám, ale hlavně se
tento postup nedá použít při zautomatizování zálohování (což by mělo být jedním
z hlavních cílů zálohovacích systémů).

\newpage

\section{Kopírování rozdílů}

Když si vezmeme za cíl kopírovat jen ta data, která se oproti poslednímu
zálohovanému stavu změnila, dostáváme se k otázce, jak dostatečně bezpečně
poznat, co se změnilo.

Vzhledem k vlastnostem v současnosti používaných souborových systémů je
nejsnadnější dívat se na změny soubor po souboru a v případě změny ho přenést
celý. Dá se zavést i drobnější granularita a sledovat data po menších blocích,
než po celých souborech, ale to je již výrazně obtížnější.

\newacronym{UNIX}{UNIX}{Operační systém vzniklý v roce 1969, který dal vzniknout
celé rodině odvozených systémů -- Linux, BSD, Mac OS X, aj.}

Současné souborové systémy si u každého souboru pamatují několik parametrů
(vezmeme si za příklad souborové systémy vycházející z operačního systému
\gls{UNIX}): vlastník a skupina, vlastnická, skupinová a ostatní práva, čas
poslední změny, velikost souboru a několik dalších parametrů.

Pokud uvažujeme soubory s parametry, je potřeba odlišit změny pouze v nich
(například změna vlastníka či skupiny) od změn na datech souboru. Parametry
souboru je nutné si přečíst vždy a tak změny na nich poznáme jednoduše, možností
jak poznat změnu na datech souboru je pak vícero:

\newacronym{MD5}{MD5}{Kryptografická hashovací funkce vytvářející z libovolného vstupu výstup o velikosti 128 bitů}
\newacronym{SHA1}{SHA1}{Secure Hash Algorithm -- pokročilejší kryptografická hashovací funkce, vytváří otisk o velikosti 160 bitů}

\begin{itemize}
	\item Kompletní porovnání obsahu souboru po bytech -- zabere čas lineární
	s velikostí souboru a stejnou přenosovou kapacitu, je ale neomylné.
	\item Kontrolní součet obsahu souboru -- je nutné stále spočítat kontrolní
	součet (třeba pomocí \gls{MD5} nebo \gls{SHA1}) souboru v lineárním čase
	k jeho velikosti, ale přenést mezi zálohovaným a zálohovacím strojem je
	potřeba pouze tento kontrolní součet (a porovnat ho s uloženým). Při volbě
	dobré hashovací funkce se věří, že nalezení {\it kolize} (nalezení více
	různých souborů se stejným hashem) je extrémně nepravděpodovné a tedy
	spolehlivé.
	\item Porovnání jen parametrů (času poslední modifikace a velikosti) --
	je nejrychlejší (jde jen o několik porovnání v konstantním čase) a pro
	většinu situací je dostačující.
\end{itemize}

Důvodem, proč je poslední možnost považována za dostačující je ta, že je při
běžném provozu jen velmi málo situací, kdy se změní obsah souboru, ale nezmění
se čas jeho poslední modifikace a jeho velikost. Většinou to vyžaduje přímou
a vědomou akci, kterou se čas modifikace souboru nastaví zpět na původní
hodnotu.

Situace, ve které je porovnání jen podle velikosti a času modifikace
nedostačující, je již nastíněné odhalování útoků na zálohovaný stroj. Pokud se
již útočníkovi povede získat kontrolu nad strojem, může snadno modifikovat
systémové soubory a potom změnit čas jejich modifikace nazpět. Pokud se před
takovou situací chceme chránit, je vhodné občas nechat zkontrolovat alespoň
kontrolní součty všech souborů oproti verzi uložené v zálohovacím systému.

V současnosti jedním z nejvíce využívaných protokolů a současně i programů
implementující nějakým způsobem kopírování jen rozdílných souborů je {\bf rsync}
(viz \cite{rsync}). Ten potřebuje na jedné straně běžícího klienta a na druhé
straně běžící server. Ty si mezi sebou vyměňují různé parametry a kontrolní
součty souborů a jejich částí. Hlavním cílem protokolu je snížit objem síťové
komunikace na minimum a přenášet jen nezbytné množství dat (přenos navíc probíhá
komprimovaně).

Bohužel primárním cílem rsyncu je {\it synchronizace} dvou složek a není tak
úplně vhodný k tomu držet si více historických verzí záloh -- to by pravděpodobně
znamenalo mít synchronizovanou složku pro každou historickou verzi a to rsync
neumí dělat efektivním způsobem. Neumí v nových složkách synchronizovat jen
změny oproti úplně jiné složce, bylo by potřeba si vždy na zálohovacím stroji
zkopírovat celou starší zálohu do nové složky a pak synchronizovat až vůči ní.

Toto je obecně problém většiny synchronizačních protokolů a programů, takže se
většinou nedají samostatně použít k účinnému zálohování. Je však možné je
využít jako efektivní přenosovou část zálohovacího systému.

\section{Rozdílové zálohování}

Pokud už přenášíme pouze rozdíly, nešel by udělat další krok a místo vždy celých
kopií zálohovaných dat pro každou historickou verzi si ukládat jen tyto rozdíly?
Tím bychom odstranili významný problém nastíněný výše, a to přesněji místo
potřebné k uložení záloh (což by bez rozdílového zálohování znamenalo ukládat
velký datový objem: $[\hbox{počet záloh}]\times[\hbox{velikost dat}]$).

Pokud se na problém podíváme v prvním přiblížení, stačí nám si jen na počátku
uložit plnou kopii dat a všechny další historické verze si ukládat jen jako
rozdíly vždy oproti té předchozí. Pro získání konkrétní verze dat nám pak stačí
jen na původní data aplikovat postupně všechny zapamatované rozdíly až do nějaké
verze.

Pro získání rozdílu textových dat (takový rozdíl se typicky nazývá anglickým
výrazem {\it diff} nebo u binárních dat někdy {\it delta}) je možné použít mnoho
nástrojů porovnávajících zadané textová data řádek po řádku. Produkují pak
takzvaný {\it patch}, který obsahuje řádky, které ubyly, a řádky, které naopak
přibyly.

\newacronym{VCDIFF}{VCDIFF}{Formát a algoritmus definovaný v RFC 3284 založený
na článku {\it Data Compression Using Long Common String}}

U binárních dat je situace složitější a není tak jednoduché stanovit, čeho by
se měl binární diff držet a co by mělo být jeho výsledkem. Pro sjednocení mnoha
implementací vznikl formát \gls{VCDIFF}. Ten specifikuje tři různé instrukce:
pro přidání nové sekvence, pro zkopírování úseku ze staré sekvence a pro
opakování určité sekvence dat.

Dá se použít i pro kompresi (i když pro tyto účely existují lepší algoritmy a
formáty), ale hlavně se dá za sebe zřetězit stará a nová verze binárních dat,
ale kóduje se jen úsek odpovídající nové verzi dat. Díky tomu, že se ale může
odkazovat na sekvence v původní verzi dat, vznikne tak vlastně jen požadovaný
binární diff.

\subsection*{Problémy reálné implementace}

V reálné implementaci rozdílové zálohování musíme vyřešit ještě několik drobných
problémů. Prvním z nich je to, že postupně se nabalující rozdíly způsobí, že
při delším provozu zálohovacího systému bude získání nejnovější verze dat trvat
neúměrně dlouho. Bude totiž nutné začít u první verze a postupně aplikovat
všechny rozdíly -- to u několika málo verzí ještě nepředstavuje problém, ale
u několika desítek až stovek verzí to již začne velice zdržovat.

Řešením je omezit maximální hloubku odkazování a pokud by měla být nová verze
v již moc velké hloubce, uložit ji buď jako kompletní verzi (tedy by se z ní
stala nová \uv{nultá hladina}), nebo jako diff oproti vhodné verzi v menší
hloubce. Zde se dá uplatnit heuristika vybírající nejvhodnější předchozí verzi
(například s nejmenším diffem).

Druhým problémem, před který se musí reálná implementace postavit, je možnost
mazat historické zálohy. Dokud byla každá historická verze nezávislá na
ostatních, tak nás její smazání nic nestálo a nijak neovlivnilo okolní verze.
U na sebe navazujících rozdílů si takové prosté smazání nemůžeme dovolit,
protože bychom tím znehodnotili všechny verze navazující v posloupnosti rozdílů
na mazanou verzi.

Pokud tedy chceme mazat historické verze (a to je u dlouhodobě fungujícího
zálohovacího systému nezbytné), je potřeba implementovat řešení nějakým
způsobem distribuující mazané změny do navazujících verzí, neboli přepočítávat
navazující verze, aby zahrnovaly i mazané změny.

\section{Automatizace}

Výše popsané způsoby se dají snadno zautomatizovat. Pokud nevyžadují uživatelovo
aktivní rozhodování během zálohování, lze naplánovat jejich pravidelné spouštění
v naplánovaných intervalech.

Pro tuto automatizaci lze využít mnoho různých postupů. Buď může zálohovací
systém stále běžet jako systémová služba (démon) a hlídat si čas plánovaného
zálohování sám, nebo lze využít například v \gls{UNIX}u dostupný {\it cron} na
naplánování spouštění přednastaveného příkazu.

\section{Existující zálohovací systémy}

TODO... Bacule etc.

\chapter{Vlastnosti zálohovacího systému FenixBackup}

Zálohovací systém {\it FenixBackup} dostal své jméno podle mýtického ptáka
fénixe, který dokázal znovu a znovu povstávat znovuzrozený ze svého popela. To
je docela trefná paralela obnově zálohovaných dat, když nás potká nějaká
nešťastná událost.

Nyní se pokusíme představit si hlavní myšlenky, na kterých FenixBackup staví,
a základní principy jeho fungování.

\section{Základní přehled}

FenixBackup je souborově orientovaný zálohovací systém schopný si držet více
historických verzí souborů. Každý soubor je v~něm navíc ukládán nezávisle a tak
je možné si od určitých souborů držet více historických záloh, než od jiných.

Ukládání jednotlivých verzí funguje na principu rozdílových záloh ve formátu
\gls{VCDIFF} s~omezením na maximální hloubku na sebe navázaných delta rozdílů.

Další možností bylo ukládat vždy celé nové verze souborů, sdružovat je k~sobě
a komprimovat, jako to dělá třeba verzovací systém Git (stavějící na myšlence,
že podobná data vedle sebe se při komprimaci vhodným kompresním algoritmem
výrazně zmenší), ale zvolil jsem raději cestu přes \gls{VCDIFF}. Při pokusech
s~několika většímu binárními soubory vycházely delta rozdíly menší, než když
jsme jednotlivé verze vzal a dohromady zkomprimoval.

Velký důraz byl kladen také na snadnou, přehlednou, ale přitom dostatečně mocnou
schopnost konfigurace, které by měla umožňovat popsat všechny myslitelné
scénáře zálohování.

\newacronym{SSHFS}{SSHFS}{SSH File System -- připojení vzdáleného souborového
systému přes protokol \texttt{SSH}}

Dalším mým cílem bylo umožnit snadné rozšiřování systému a to jednak ve formě
zajištění zpětné kompatibility dat, tak také ve snadném rozšiřování systému.
Příkladem toho je systém {\it adaptérů} pro přístup k~zálohovaným datům:
Existuje společný interface pro komunikaci s~adaptéry, ale to, jestli se data
získají z~lokálního filesystému, stáhnout přes \gls{SSHFS}, nebo bude probíhat
komunikace přes rsync protokol, by mělo být samotnému jádru zálohovacího
systému jedno.

\section{Pravidla konfigurace}

Zásadní částí dobře fungujícího zálohovacího systému je dobře fungující
a konfigurovatelný systém pravidel zálohování. Zápis pravidel do konfiguračního
souboru je ukázán v~kapitole vztahující se k~uživatelské dokumentaci, zde se
pokusíme rozebrat obecné fungování pravidel.

\subsection{Zálohovací pravidla určená podle cest}

Pravidla je možné specifikovat pro jakoukoliv cestu v~zálohovaných datech
(kořenová složka zálohy je označována pomocí lomítka: \texttt{/}) a stejnou
cestu je možné uvést i vícekrát -- v~takovém případě platí pro tuto cestu
poslední specifikovaná pravidla.

Pokud pro nějakou cestu nejsou specifikována pravidla (v~reálném případě to bude
pravděpodobně většina cest), použijí se pro ní nejbližší vyšší pravidla po cestě
ke kořeni. Je možné tak třeba zakázat zálohování určitých složek.

\medskip

\noindent{\bf Přehled a význam parametrů, které lze nastavovat:}
\begin{itemize}
	\item {\tt scan <true|false>}: Platí jen pro složky a říká, jestli vůbec
	tuto složku prozkoumávat. Defaultně {\tt true}.
	\item {\tt backup <true|false>}: Jestli zadaný soubor zálohovat, nebo
	ne. Defaultně {\tt true}.
	\item {\tt history <1-10>}: Parametr pro funkci na ředění historie, čím
	vyšší číslo, tím více verzí držet. Defaultně 1.
	\item {\tt priority <1-10>}: Parametr pro prioritizační funkci. Určuje,
	které soubory zálohovat dříve, než ostatní. Čím vyšší číslo, tím
	důležitější. Defaultně 1.
\end{itemize}

\subsection{Zálohovací pravidla podle specifičtějšího výběru}

Mimo pravidel odvozených jen od cest existují také pravidla umožňující pokrýt
třeba jen určitý typ souboru. Jsou také vázána na konkrétní cestu (a platí tak
jen v~určeném podstromě), ale je možné u~nich zavést složitější způsoby
filtrování.

Pro každou cestu je možné v~konfiguraci uvést seznam {\it pravidel pro soubory},
kde každé pravidlo má parametry, které nastavuje, a pak filtry určující, na jaké
soubory se má aplikovat. Je možné filtrovat podle následujících parametrů:

\begin{itemize}
	\item Regex\footnote{regulární výraz} názvu souboru (\texttt{regex}) --
	pokud je nevyplněný či prázdný, odpovídá jakýkoliv soubor, pokud je
	neprázdný, musí celý název souboru (jen souboru, bez cesty) odpovídat
	zadanému regexu.
	\item Regex cesty k~souboru (\texttt{path\_regex}) -- podobně jako výše,
	nevyplněnému odpovídá vše, vyplněnému jen soubory, které se nacházejí
	(oproti aktuálnímu adresáři) na cestě odpovídající regexu.
	\item Minimální a maximální velikost souboru v~bytech (parametry
	\texttt{size\_at\_least} a \texttt{size\_at\_most}).
\end{itemize}

Jednotlivá pravidla se vždy aplikují v~pořadí, v~jakém jsou uvedena v~konfiguraci,
a funguje zde princip přepisování (tedy pokud je splněna podmínka pravidla, jsou
přepsány všechny parametry zálohování, které pravidlo nastavuje).

\medskip\goodbreak

\noindent Při dotazu na konkrétní soubor se pak parametry pravidel určí takto:
\begin{enumerate}
	\item Vezmou se defaultní hodnoty pravidel a začne se v~kořenové složce.
	\item V~aktuální složce se nejdříve aplikují pravidla složky (pokud
	existují) a poté se projdou všechny pravidla pro soubory:
	\item Pokud vyhovuje filtr u~pravidla pro soubory, aplikuje se.
	\item Pokud se dá po zadané cestě sestoupit ještě o~úroveň níže, sestoupí
	se a~opakuje se znovu od bodu~2.
\end{enumerate}

Nakonec se vrátí finální pravidla vzniklá aplikací všech pravidel, přes která se
prošlo. Toto řešení nabízí velmi silné možnosti konfigurace, bohužel cacheování
výsledků je zde vzhledem k~pravidlům s~filtry značně obtížné, pro běžně velké
konfigurační soubory by ale rychlost i tak měla být dostatečná a úzkým hrdlem
zde stále bude rychlost načítání dat z~disku.

\section{Model ukládání dat}

FenixBackup představuje každou zálohu (ať už se v~rámci ní zálohovaly všechny
soubory, nebo se povedlo zálohovat pouze jediný) jedním souborovým stromem. Ten
obsahuje všechny záznamy o~všech souborech, které by se měly dostat do zálohy.

Při svém vzniku obsahuje tento strom jen informace zjistitelné o~souborech bez
nutnosti číst jejich obsah (to zajistí příslušný adaptér, viz následující
podkapitola) a postupně se zpracováním obsahu jednotlivých souborů se v~něm
objevují i kontrolní součty (hashe) souborů, které odkazují na konkrétní datové
bloky.

\newacronym{SHA256}{SHA256}{Kryptografická hashovací funkce, vytváří otisk
o~velikosti 256 bitů}

Každý uložený soubor je jednoznačně identifikován svým \gls{SHA256} hashem
a~představován stejnojmenným datovým blokem. Tento datový blok může být buď
úplný, nebo je to jen \gls{VCDIFF} oproti jinému datovému bloku.

Pro fungování zálohovacího systému předpokládáme, že nalezení kolize
v~kryptografické hashovací funkci jako je \gls{SHA256} je extrémně
nepravděpodobné (zatím není známý ani cílený útok, který by uměl vyprodukovat
kolizi, a velikost kolizní domény je $2^{256}$). Na stejném předpokladu
o~nepravděpodobných kolizích v~reálném nasazení staví i verzovací systém Git,
který navíc používá jen kratší hashe generované \gls{SHA1}.

Díky tomu, že jsou datové bloky pojmenovány hashem souboru, který mají
představovat, řeší se velmi efektivním způsobem duplikáty souborů. Ačkoliv třeba
dva stejné soubory mohou mít úplně rozdílnou historii, tak finálně je uložen
v~zálohovacím systému jen jeden datový blok, který představuje oba soubory.

Datové bloky a jejich případné rozdílové návaznosti tedy vůbec nemusí
korespondovat s~historií jednotlivých souborů uloženou v~souborových stromech.
Ale díky tomu, že v~souborovém stromu je uložený hash každé jednotlivé verze
souboru, lze všechny historické verze snadno zrekonstruovat.

Záznam o~každém jednotlivém souboru se může navíc nacházet v~jednom
z~následujících stavů:
\begin{itemize}
	\item\texttt{UNKNOWN} -- Nový soubor (bez známé minulé verze), u~kterého
	zatím nemáme zálohovaný jeho obsah.
	\item\texttt{NEW} -- Nový soubor (bez známé minulé verze), u~kterého již
	máme uložený a zpracovaný jeho obsah.
	\item\texttt{UNCHANGED} -- Soubor se známou minulou verzí, který se
	nezměnil (ani parametry, ani obsahem).
	\item\texttt{UPDATED\_PARAMS} -- Soubor se známou minulou verzí, u~kterého
	se změnily pouze parametry, ale obsah ne.
	\item\texttt{NOT\_UPDATED} -- Soubor se známou minulou verzí, u~kterého
	se změnil i~obsah, a jenž ještě nemáme uložený.
	\item\texttt{UPDATED\_FILE} -- Soubor se známou minulou verzí, u~kterého
	se změnil i~obsah, a jenž již máme uložený a zpracovaný.
	\item\texttt{DELETED} -- Soubor byl z~nějakého důvodu vymazán (ať už na
	přímé přání uživatele, nebo automaticky při uvolňování místa).
\end{itemize}

\section{Adaptéry pro přístup k~zálohovaným datům}

Zálohovací systém je psán tak, aby byl nezávislý na metodě přístupu k~zálohovaným
datům. Samotnému jádru je jedno, jak data získá, o~to se starají takzvané
{\it adaptéry}.

V~úloze zálohování se dají jejich úkoly rozdělit do dvou fází. V~první fázi
získávají adaptéry informace o~souborech a předávají je jádru zálohovacího
systému, ve druhé fázi jim pak jádro zálohovacího systému dodá seznam souborů,
které chce získat, a adaptér nějakým způsobem opatří jejich obsah a dodá je
nazpět jádru.

První fáze začíná tím, že adaptér začne skenovat systém souborů od zadaného
místa, načítá parametry souborů (vlastník, práva, velikost, \dots) a ukládá tyto
záznamy do souborového stromu. Jádro zálohovacího systému všechny ukládané
záznamy zpracovává a může případně operativně rozhodnout, že ho informace v~nějakém
podstromě již nezajímají (třeba v~případě nastavení \texttt{scan: false} pro
nějakou cestu v~konfiguraci). V~takovém případě to dá domluveným způsobem vědět
adaptéru a ten by již tento podstrom neměl dále procházet.

Během této fáze se jádro zálohovacího systému pokouší (podle cesty, času poslední
modifikace a velikosti souborů) rozhodnout, které soubory se nezměnily, u~kterých
se změnily pouze parametry, a které soubory bude potřeba stáhnout pro uložení
jejich nové verze.

Seznam požadovaných souborů (utříděný podle prioritizační funkce, viz dále) pak
předá nazpět adaptéru a ten by měl zařídit získání obsahu těchto souborů
a~předání obsahu jádru zálohovacího systému.

To si spočte hash souborů, zjistí, zdali případně nemá již soubor se stejným
hashem uložený, a pokud ne, uloží nový datový blok (buď jako kompletní datový
blok, nebo jako \gls{VCDIFF} oproti jinému).

\section{Priorita zálohovaných souborů}

Různé soubory mohou mít konfigurací jinak nastavené priority zálohování, čímž
lze zálohovacímu systému sdělit, o~které soubory máme větší zájem.

Do rozhodovacího procesu se také promítá to, jakou nejmladší verzi daného
souboru již máme uloženou. Systém je totiž připravený i na to, že ne vždy se
povede provést kompletní zálohu, například u~přenosných počítačů připojených
k~rychlé síti jen po omezenou dobu. V~takovém případě zůstanou některé soubory ve
stromu se stavem \texttt{UNKNOWN} nebo \texttt{NOT\_UPDATED} a další zálohy na
to berou ohled (soubory, jež se nepovedlo zálohovat delší dobu, mají při příštím
zálohování přednost).

Konkrétně si jádro zálohovacího systému přiděluje každému souboru číselné
{\it skóre} odpovídající následujícímu vztahu:

$$[\hbox{čas od poslední zálohované verze}]\times[\hbox{priorita souboru}]$$

Pokud soubor nemá známou minulou verzi, tak se jako čas poslední zálohované
verze bere čas minulého spuštění zálohování -- protože tento nový soubor se mohl
objevit kdykoliv v~mezičase mezi touto poslední zálohou a současností.

Tímto postupem by se mělo zajistit zálohování postupně všech souborů i v~případě,
že zálohovaný stroj bývá připojen k~zálohovacímu systému pravidelně jen po
omezenou dobu.

\chapter{Implementace zálohovacího systému FenixBackup}

Zálohovací systém je napsaný v jazyce C{\tt++}, konkrétně podle specifikace
C{\tt++}11. Vyvíjen byl na platformě operačního systému Linux a je optimalizován
pro jeho systém práv a parametrů souborového systému, ale měl by fungovat na
všech hlavních platformách.

\section{Obecné návrhové vzory}

\newacronym{PIMPL}{PIMPL}{Návrhový idiom \uv{private implementation} pokoušející
se skrýt co nejvíce implementace před vnějším světem}

Při implementaci jsem se držel snahy mít v hlavičkových souborech tříd co možná
nejméně položek nepotřebných jako rozhraní pro komunikaci s ostatními třídami.
Většina privátních proměnných a funkcí je tedy (podle \gls{PIMPL} idiomu)
zapouzdřena uvnitř vnořených tříd

Důvodem pro tuto snahu je za prvé přehlednost hlavičkových souborů a za druhé
odstranění potřeby překompilování všech souborů, které daný hlavičkový soubor
includují při změně v privátní metodě.

Dále se v celém systému hojně používají chytré pointery zavedené v C{\tt ++}11,
které na mnoha místech zjednodušují zacházení s instancemi tříd, pokud si na ně
potřebujeme držet odkaz z více míst zároveň.

Vše bydlí ve jmenném prostoru {\it FenixBackup}.

\section{Přehled tříd a jejich funkce}

V minulé části jsme si nastínili schéma stromečků reprezentujících jednotlivé
zálohy a držící informace o souborech, a schéma datových bloků.

Souborové stromečky jsou představovány třídou {\it FileTree} a obsahují většinu
logiky související se zálohováním souborů. V jednotlivých jeho uzlech pak bydlí
instance třídy {\it FileInfo}, jež si drží informace vždy o jedné složce,
normálním souboru nebo symlinku. V případě složky si pamatují pointery na v ní
obsažené soubory, jinak si pamatují hash obsahu.

Hashe obsahu odkazují na instance třídy {\it FileChunk}, které již drží finální
data. To je vždy \gls{VCDIFF} oproti prázdnému souboru, nebo jinému souboru
reprezentovanému instancí třídy {\it FileChunk}.

\subsection{FileTree}

Hlavní logika jádra zálohovacího systému je soustředěna do třídy {\it FileTree},
jež obsahuje tři metody pro registraci položek do souborového stromu:

\begin{itemize}
	\item\texttt{AddDirectory(rodic, jmeno\_souboru, parametry)}
	\item\texttt{AddFile(rodic, jmeno\_souboru, parametry)}
	\item\texttt{AddSymlink(rodic, jmeno\_souboru, parametry)}
\end{itemize}

Každá z těchto metod vrací odkaz na nově přidanou položku (který se pak třeba
v případě složky dá využít pro přidání položek do ní), nebo \texttt{nullptr},
pokud daná věc nemá být uložena ve stromě souborů (a tedy nemá například smysl
procházet podstrom souborů pod touto složkou).

Dále obsahuje třída {\it FileTree} metodu pro vrácení seznamu souborů, které si
zálohovací systém přeje dodat, a sadu dvou metod pro uložení nového obsahu
položky (volá adaptér pokaždé, když se mu povede získat obsah nějakého souboru)
a pro vrácení obsahu uloženého souboru:

\begin{itemize}
	\item\texttt{FinishTree()} $\rightarrow$ vektor odkazů na položky stromu
	\item\texttt{ProcessFileContent(polozka\_stromu, inputstream)}
	\item\texttt{GetFileContent(polozka\_stromu, outputstream)}
\end{itemize}

Výše vyjmenované metody jsou hlavními komunikačními prostředky jádra
zálohovacího systému. Dále třída {\it FileTree} obsahuje statické metody pro
načtení konkrétního stromu souborů reprezentujícího jednu zálohu z úložiště a
několik pomocných metod (uložení stromu, nalezení položky podle cesty nebo hashe
obsahu, vrácení kořenu celého stromu kvůli předání první vrstvě položek jako
rodiče aj.).

\subsection{FileInfo}

Hlavním úkolem této třídy je držet si informace o jednom souboru, složce či
symlinku. V případě složky obsahuje vektor obsažených souborů, v ostatních
případech obsahuje hash odkazující na příslušný datový blok. Ve všech případech
si drží parametry.

Mezi základní metody patří následující tři, zbytek veřejných metod jsou pak buď
settery nebo gettery na vlastnosti souboru:

\begin{itemize}
	\item\texttt{AddChild(jmeno, odkaz\_na\_instanci\_FileInfo)}
	\item\texttt{GetChild(jmeno)} $\rightarrow$ odkaz na instanci {\it FileInfo}
	\item\texttt{GetChilds()} $\rightarrow$ odkaz na asociativní pole odkazů
	na {\it FileInfo}
\end{itemize}

\subsection{Ostatní třídy}

Třída {\it FileChunk} představuje jednotlivé datové bloky. Každá instance třídy
si záznam, jestli je odvozená jako rozdíl od jiné instance, nebo jestli je
takzvaně na \uv{nulté úrovni} a na ničme nezávisí.

Dále existuje třída představující konfiguraci {\it Config}, která vrací
jednotlivé položky načtené konfigurace a pak řeší dotazy na pravidla pro
jednotlivé zálohované soubory (opět podle načtené konfigurace).

\section{Ukládání dat}

Souborové stromy a datové bloky jsou ukládány na oddělená místa. Každý souborový
strom, stejně jako každý datový blok, sídlí v samostatném souboru na disku.
Soubory jsou ukládány v binárním formátu.

Pro serializaci instancí tříd {\it FileTree} (s navázanými instancemi třídy
{\it FileInfo}) a {\it FileChunk} se používá serializační knihovna {\it Cereal}%
\footnote{\texttt{http://uscilab.github.io/cereal/index.html}}, což je projekt
vzniklý cíleně pro C{\tt ++}11 a novější využívající vlastností chytrých
pointerů pro serializaci složitých datových struktur.

Ke každé serializované instanci třídy {\it FileChunk} je ještě připojen blok
dat ve formátu \gls{VCDIFF} popisující rozdíl oproti prázdnému souboru, nebo
oproti nějakému jinému uloženému datovému bloku.

\subsection{Postup uložení nové verze souboru}

\section{Obnova dat}

TODO (nejdříve smazat a pak až obnovit - hardlinky)

\section{Mazání uložených dat -- uvolňování místa}

\chapter{Implementace zálohovacího systému FenixBackup}

Zálohovací systém je napsaný v~jazyce \Cpp, konkrétně podle specifikace
\Cpp11. Vyvíjen byl na platformě operačního systému Linux a je optimalizován
pro jeho systém práv a parametrů souborového systému, ale měl by fungovat na
všech hlavních platformách.

\section{Obecné návrhové vzory}

\newacronym{PIMPL}{PIMPL}{Návrhový idiom \uv{private implementation} pokoušející
se skrýt co nejvíce implementace před vnějším světem}

Při implementaci jsem se držel snahy mít v~hlavičkových souborech tříd co možná
nejméně položek nepotřebných jako rozhraní pro komunikaci s~ostatními třídami.
Většina privátních proměnných a funkcí je tedy (podle \gls{PIMPL} idiomu)
zapouzdřena uvnitř vnořených tříd

Důvodem pro tuto snahu je za prvé přehlednost hlavičkových souborů a za druhé
odstranění potřeby překompilování všech souborů, které daný hlavičkový soubor
includují při změně v~privátní metodě.

Dále se v~celém systému hojně používají chytré pointery zavedené v~\Cpp11,
které na mnoha místech zjednodušují zacházení s~instancemi tříd, pokud si na ně
potřebujeme držet odkaz z~více míst zároveň.

Vše bydlí ve jmenném prostoru {\it FenixBackup}.

\section{Přehled tříd a jejich funkce}

V~minulé části jsme si nastínili schéma stromečků reprezentujících jednotlivé
zálohy a držící informace o~souborech, a schéma datových bloků.

Souborové stromečky jsou představovány třídou {\it FileTree} a obsahují většinu
logiky související se zálohováním souborů. V~jednotlivých jeho uzlech pak bydlí
instance třídy {\it FileInfo}, jež si drží informace vždy o~jedné složce,
normálním souboru nebo symlinku. V~případě složky si pamatují pointery na v~ní
obsažené soubory, jinak si pamatují hash obsahu.

Hashe obsahu odkazují na instance třídy {\it FileChunk}, které již drží finální
data. To je vždy \gls{VCDIFF} oproti prázdnému souboru, nebo jinému souboru
reprezentovanému instancí třídy {\it FileChunk}.

\subsection{FileTree}

Hlavní logika jádra zálohovacího systému je soustředěna do třídy {\it FileTree},
jež obsahuje tři metody pro registraci položek do souborového stromu:

\begin{itemize}
	\item\texttt{AddDirectory(rodic, jmeno\_souboru, parametry)}
	\item\texttt{AddFile(rodic, jmeno\_souboru, parametry)}
	\item\texttt{AddSymlink(rodic, jmeno\_souboru, parametry)}
\end{itemize}

Každá z~těchto metod vrací odkaz na nově přidanou položku (který se pak třeba
v~případě složky dá využít pro přidání položek do ní), nebo \texttt{nullptr},
pokud daná věc nemá být uložena ve stromě souborů (a tedy nemá například smysl
procházet podstrom souborů pod touto složkou).

Dále obsahuje třída {\it FileTree} metodu pro vrácení seznamu souborů, které si
zálohovací systém přeje dodat (setříděného podle důležitosti) a několik
pomocných metod pro uložení stromu, vrácení kořenu celého stromu kvůli předání
první vrstvě položek jako rodiče aj.):

\begin{itemize}
	\item\texttt{FinishTree()} $\rightarrow$ vektor odkazů na položky stromu
	\item\texttt{GetRoot()} $\rightarrow$ odkaz na {\it FileInfo}
	reprezentující kořen
\end{itemize}

Nakonec třída obsahuje statické metody pro načtení stromu konkrétního jména
z~úložiště (ta vrací odkaz na strom nebo \texttt{nullptr}) a pro vrácení seznamu
všech stromů v~úložišti.

\subsection{FileInfo}

Hlavním úkolem této třídy je držet si informace o~jednom souboru, složce či
symlinku, o~kterých si ve všech případech drží parametry jako jsou práva, čas
poslední modifikace a podobně. V~případě složky navíc obsahuje vektor obsažených
souborů, v~ostatních případech pak hash odkazující na příslušný datový blok.

Mezi základní metody patří metody pro zpracování obsahu souboru nebo jeho
vrácení. Jako nepovinný parametr mohou dostat odkaz na aktuálně konstruovaný
strom souborů a pomocí něj vyhledávat soubory se stejným hashem v~minulé verzi
stromu:

\begin{itemize}
	\item\texttt{ProcessFileContent(inputstream, strom = nullptr)}
	\item\texttt{GetFileContent(outputstream)} $\rightarrow$
	\texttt{outputstream} (vrácení umožňuje použití v~syntaxi \texttt{<\relax<} streamů)
\end{itemize}

Další důležité metody jsou následující tři sloužící pro správu obsahu složek.
Zbytek veřejných metod jsou pak buď settery nebo gettery na vlastnosti souboru:

\begin{itemize}
	\item\texttt{AddChild(jmeno, odkaz\_na\_instanci\_FileInfo)}
	\item\texttt{GetChild(jmeno)} $\rightarrow$ odkaz na instanci {\it FileInfo}
	\item\texttt{GetChilds()} $\rightarrow$ odkaz na asociativní pole odkazů
	na {\it FileInfo}
\end{itemize}

\subsection{Ostatní třídy}

Třída {\it FileChunk} představuje jednotlivé datové bloky. Každá instance třídy
si záznam, jestli je odvozená jako rozdíl od jiné instance, nebo jestli je
takzvaně na \uv{nulté úrovni} a na ničme nezávisí.

Dále existuje třída představující konfiguraci {\it Config}, která vrací
jednotlivé položky načtené konfigurace a pak řeší dotazy na pravidla pro
jednotlivé zálohované soubory (opět podle načtené konfigurace).

Pro komunikaci s~adaptéry je pak definována abstraktní třída {\it Adapter}
obsahující metody pro proskenování a vytvoření souborového stromu, pro
zpracování obsahu souboru a pro obnovení souboru na původní, nebo na zadané
místo.

\section{Externí knihovny}

Zálohovací systém využívá několik externích knihoven a to pro zpracování
konfiguračních souborů, pro serializaci dat, pro počítání \gls{SHA256} hashů
\gls{VCDIFF} rozdílů souborů a pro práci se souborovým systémem.

Důvody pro použití externích knihoven jsou hlavně dva: Pokud existuje fungující
a efektivní řešení daného problému, není potřeba \uv{znovu vynalézat kolo}, a za
druhé, na mnohá z~těchto řešení mohou být uživatelé zvyklí (například konfigurace)
a použití známých komponent usnadní uživatelům používání.

\subsection*{Konfigurace -- libconfig}

Byla použita mezi mnoha projekty rozšířená a přizpůsobivá knihovna
\texttt{libconfig}\footnote{\url{http://www.hyperrealm.com/libconfig/}}. Důvodem
pro její volbu je velká rozšířenost mezi C a \Cpp projekty a tedy její známost
mezi uživateli. Program potřebuje být slinkován s~knihovnou
\texttt{libconfig++}.

\subsection*{Serializace dat -- Cereal}

Pro serializaci instancí tříd {\it FileTree} (s~navázanými instancemi třídy
{\it FileInfo}) a~{\it FileChunk} se používá serializační knihovna {\it Cereal}%
\footnote{\url{http://uscilab.github.io/cereal/index.html}}, což je projekt
vzniklý cíleně pro \Cpp11 a novější využívající vlastností chytrých
pointerů pro serializaci složitých datových struktur.

Sídlí čistě v~hlavičkových souborech přiložených společně s~projektem, takže
není nutné cokoliv dynamicky linkovat.

\subsection*{Hashe a rozdíly -- sha256.h a open-vcdiff}

Pro počítání \gls{SHA256} je použita část hashovací knihovny, kterou napsal
Stephan Brumme\footnote{\url{http://create.stephan-brumme.com/hash-library/}},
sídlí jen v~hlavičkových souborech, tedy se opět nic dynamicky nelinkuje.

Pro vytváření a zpracování binárních rozdílů souborů je použita\gls{VCDIFF}
implementace \texttt{open-vcdiff}\footnote{\url{https://code.google.com/p/open-vcdiff/}},
která vyžaduje slinkování s~externími knihovnami \texttt{libvcdcom}, \texttt{libvcdenc}
a \texttt{libvcddec}.

\subsection*{Souborový systém -- boost::filesystem}

Na místech interagujících nějakým složitějším způsobem se souborovým systémem
byla použita implementace souborového systému z~\Cpp projektu \texttt{boost}%
\footnote{\url{http://www.boost.org/doc/libs/1_46_0/libs/filesystem/v3/doc/index.htm}}.
Poskytuje na platformě nezávislé rozhraní a při případném použití zálohovacího
systému na jiné platformě, než na které byl vyvinut, by tak neměla nastat žádné
významné problémy. Pro běh musí být program slinkován s~knihovnami
\texttt{libboost\_system} a \texttt{libboost\_filesystem}.

\section{Ukládání dat}

Souborové stromy a datové bloky jsou ukládány na oddělená místa. Každý souborový
strom, stejně jako popis datového bloku, sídlí v~samostatném souboru na disku.
Soubory jsou ukládány v~binárním formátu. Ke každé serializované instanci třídy
{\it FileChunk} je ještě připojen blok dat ve formátu \gls{VCDIFF} popisující
rozdíl oproti prázdnému souboru, nebo oproti nějakému jinému uloženému datovému
bloku.

Formát uložení dat určuje serializační knihovna {\it Cereal} (která podporuje
více druhů serializačních archivů, zálohovací systém využívá binárního formátu
zápisu dat). Formát dat je explicitně verzovaný, což dovoluje do budoucna
přidávat či ubírat položky se zachováním zpětné kompatibility.

Cereal dovoluje pojmenovávat ukládané položky, což se využívá při výstupu
v~lidsky čitelných formátech typu JSON nebo XML (toho bylo využíváno při vývoji),
ale v~binárním formátu jsou za sebe serializovaná jen samotná binární data. Pokud
je serializován chytrý pointer na objektu, je nejdříve uveden čtyřbytový
identifikátor typu objektu a pak buď serializovaná data objektu, nebo čtyřbytové
pořadové číslo již použitého chytrého pointeru na tento typ objektu.

Každému archivu předchází hlavička a čtyřbytové číslo verze, níže následují
seřazené tabulky položek pro obě serializované třídy. S~jejich znalostí je možné
snadno pomoci jakékoliv verze knihovny Cereal získat původní data i bez
zálohovacího systému

\subsection{Formát dat třídy FileTree}

\begin{tabular}{l >{\tt}l}
\bf Položka & \bf datový typ \\
\hline
Název stromu & std::string \\
Čas vytvoření & time\_t \\
Název předchozího stromu & std::string \\
SHA256 hash předchozího stromu & std::string \\
Odkaz na kořen stromu & std::shared\_ptr<FileInfo> \\
\end{tabular}

\subsection{Formát dat třídy FileInfo}

\begin{tabular}{l >{\tt}p{7cm}}
\bf Položka & \bf datový typ \\
\hline
Jméno souboru & std::string \\
Typ souboru & enum\{DIR, FILE, SYMLINK\}\\
Stav verzování & enum\{UNKNOWN, NEW, UNCHANGED, UPDATED\_PARAMS, UPDATED\_FILE, NOT\_UPDATED, DELETED\} \\
Parametry & struct file\_params \\
Index souboru & uint32\_t \\
Index minulé verze souboru & uint32\_t \\
Odkaz na rodiče & std::shared\_ptr<FileInfo> \\
Hash souboru & std::string \\
Asociativní pole synů & std::unordered\_map<std::string, std::shared\_ptr<FileInfo>\relax> \\
\end{tabular}

\subsection{Formát dat struktury file\_params}

\begin{tabular}{l >{\tt}l}
\bf Položka & \bf datový typ \\
\hline
Číslo zařízení & dev\_t \\
Číslo inode & ino\_t \\
Souborová práva & mode\_t \\
UID vlastníka & uid\_t \\
GID skupiny & gid\_t \\
Velikost souboru & size\_t \\
Čas modifikace (sekundy) & timespec.tv\_sec \\
Čas modifikace (nanosekundy) & timespec.tv\_nsec \\
\end{tabular}

\subsection{Formát dat třídy FileChunk}

\begin{tabular}{l >{\tt}l}
\bf Položka & \bf datový typ \\
\hline
Název (hash) & std::string \\
Název (hash) předchůdce & std::string \\
Hloubka zanoření & int \\
Velikost dat & size\_t \\
Závislé instance & std::vector<std::string> \\
\end{tabular}

\section{Postup uložení nové verze souboru}

Prvním krokem je vždy obstarání si stromu souborů, neboli instance třídy
{\it FileTree}. Většinou si vyrobíme aktuální ze současného stavu zálohovaných
dat pomocí vhodného adaptéru, ale je možné vzít i starší instanci třídy
{\it FileTree} a modifikovat tu (je pak ale nutné počítat s~tím, že nebudou
souhlasit kontrolní součty v~navazujících instancích {\it FileTree}).

\newpage

\subsection{Záznam do stromu souborů a hledání minulé verze}

Pokud si strom souborů vyrábíme, tak v~nějakou chvíli dojde k~přidání námi
sledovaného souboru (voláním jedné z~metod třídy {\it FileTree}). Přitom proběhne
následující posloupnost operací:
\begin{enumerate}
	\item Obstarají se pravidla pro tento soubor (od třídy {\it Config})
	a případně se ukončí zpracování, pokud se soubor nemá zálohovat.
	\item Vytvoří se nová instance {\it FileInfo} a přidá se odkaz do rodiče.
	\item Uloží se známé parametry souboru (vlastník, velikost, čas modifikace)
	\item Pokud je znám minulý strom: Systém se pokusí nalézt soubor se
	stejnou cestou (dotazem na svého rodiče, jestli zná svůj starší
	ekvivalent, a z~něj jedním dotazem na jméno souboru). Pokud není nalezen,
	je nastaven stav \texttt{UNKNOWN} a zpracování se ukončí.
	\item Pokud nebyla poslední verze zpracována, i když se změnila (status
	\texttt{UNKNOWN}, \texttt{NOT\_UPDATED} nebo \texttt{DELETED}), nastav
	stejný stav (mimo \texttt{DELETED}, v~tom případě nastav
	\texttt{NOT\_UPDATED}) a skonči.
	\item Pokud je nalezený soubor stejného typu, velikosti, času modifikace
	a parametrů $\rightarrow$ status \texttt{UNCHANGED} a přiřadí se stejný hash
	(odkazující na stejný datový blok).
	\item Pokud je nalezený soubor stejného typu, velikosti, času modifikace,
	ale liší se v~parametrech $\rightarrow$ status \texttt{UPDATED\_STATUS}
	a přiřadí se stejný hash (odkazující na stejný datový blok).
	\item Jinak přiřaď status \texttt{NOT\_UPDATED}.
\end{enumerate}

Pokud soubor skončí ve stavu \texttt{NOT\_UPDATED} nebo \texttt{UNKNOWN}, přidá
se do seznamu souborů k~obstarání a spočítá se pro něj skóre důležitosti, jak
již bylo ukázáno.

\subsection{Zpracování obsahu souboru}

Pokud se zálohovací systém rozhodne, že má zájem o~obsah souboru, tak ho adaptér
nějakým způsobem obstará a poté voláním metody na třídě {\it FileInfo} předá
tomuto záznamu obsah souboru.

Nedjříve se spočítá \gls{SHA256} hash a pokud není známá minulá verze souboru,
porovná se, jestli neexistuje v~minulém stromě soubor se stejným hashem. Pokud
ano, přiřadí se jako minulá verze, k~souboru se uloží hash a skončí se, protože
není nutné obsah dále zpracovávat (takto systém pozná přesuny souborů).

Pokud toto neuspěje, je potřeba obsah souboru uložit. Nedjříve systém prozkoumá,
jestli již nemá uložen soubor se stejným hashem (tedy pokud zachováme důvěru
v~kryptografickou hashovací funkci, tak i se stejným obsahem) a pokud ano, není
nutné nic nového ukládat a použijeme tento uložený datový blok.

Pokud stejný datový blok neexistuje, tak systém vytvoří novou instanci třídy
{\it FileChunk} reprezentující soubor s~tímto hashem a nechá ji zpracovat obsah.
Pokud je znám předek, tak se zkusí vytvořit \gls{VCDIFF} oproti poslední verzi
(získají se data této poslední verze a udělá se rozdíl mezi nimi), jinak se dělá
rozdíl oproti prázdnému souboru.

Pokud by byl při vytváření rozdílu vůči starší verzi překročen limit na maximální
hloubku na sebe navazujících rozdílů, tak se vytvoří rozdíl oproti první známé
verzi souboru (tedy oproti kořeni, který sám vznikl rozdílem oproti prázdnému
souboru), čímž se opět dosáhne nižší hloubky.

Alternativou by bylo vytvářet novou kompletní kopii takového souboru, ale rozdíl
oproti nějaké základní verzi zabere nejvýše tolik místa, kolik by zabral rozdíl
oproti prázdnému souboru. Takže toto řešení je nejméně stejně dobré, jako řešení
s~vyráběním nových záznamů s~o~jedna nižší maximální hloubkou.

\section{Obnova dat}

Při obnově dat se dá zvolit několik strategií. Vždy se zvolí konkrétní záloha
(konkrétní strom), ze kterého nás zajímají daná data, a jádro zálohovacího
systému pak poskytuje podporu pro:
\begin{itemize}
	\item Obnovu jednoho konkrétního souboru nebo celého podstromu
	\item Obnovu do původního místa, do jiného místa na původním (vzdáleném)
	stroji a do jiného místa v~lokálním souborovém systému
	\item Obnovu pouze dat souboru, pouze parametrů souboru, nebo všeho
\end{itemize}

Navíc se dá specifikovat taktika, co se má dělat, pokud v~aktuálním stromu není
soubor dostupný, určuje se podle dalšího parametru předávaného obnovovacím
funkcím. Výchozí chování je to, že v~takovém případě se pokusí systém nalézt
nejnovější zálohovanou verzi souboru a obnoví tu (zde si můžeme všimnout, že
starší verzi má smysl hledat jen u~souborů označených jako
\texttt{NOT\_UPDATED}, u~souborů označených \texttt{UNKNOWN} ne). Alterantivním
chováním je obnova pouze těch dat, která jsou odkazovaná aktuálním stromem
(neuložené soubory jsou přeskakovány).

Při obnovování je navíc potřeba dát pozor na hardlinky vedoucí mimo obnovovanou
oblast. V~takovém případě by se mohlo stát, že se obnovením nějakého souboru
přes hardlink projeví změna i na úplně jiném místě, což pravděpodobně není
požadovaný cíl. Zálohovací systém tedy přijal taktiku nejdříve obnovovaný soubor
celý smazat (tím se zruší případné provázání hardlinkem) a pak na stejném místě
vytvořit znovu.

\subsection{Postup získání obsahu souboru}

Obnova jednoho souboru začíná dotazem na jeho typ -- pokud to je složka, tak
jsou všechny potřebné informace uložené již ve stromě souborů a není nutné
získávat jakákoliv data z~datových bloků. Pokud to není složka, tak se přes
třídu {\it FileInfo} dostane žádost o~obnovení dat až na datový blok
reprezentující danou verzi souboru.

Pokud se datový blok (uložený ve formátu \gls{VCDIFF}) neodkazuje na žádný jiný,
je jeho obsah přímo obnoven jako diff od prázdného souboru a vrácen. Pokud se
odkazuje na jiný datový blok, je nejdříve dotazem na tento blok (rekurzivně)
získán jeho obsah, proti kterému je pak aplikováním rozdílu vyroben obsah tohoto
datového bloku a ten je vrácen.

\section{Mazání uložených dat -- uvolňování místa}

Zálohovací systém se může čas od času potkat s~nutností uvolnit nějaké místo.
K~uvolnění místa je nutné zahodit některé informace, které si zálohovací systém
pamatuje, a když už se to provádí, měl by systém nějakým vhodným způsobem zvolit
podle něj nejméně důležitá data a ta zahodit.

Toliko abstraktní představa, teď konkrétněji. Uvolnit místo se dá smazáním
některých datových bloků a úkolem mazání je vybrat ten blok, který bude
\uv{nejméně scházet}. Na co se nesmí zapomenout je to, že na jeden datový blok
se může odkazovat více záznamů o~souborech.

Vybrání bloku ke smazání probíhá tak, že se pro každý záznam o~zálohovaném
souboru spočítá {\it baddness} (\uv{míra špatnosti}), která se odvíjí mimo jiné
i od stáří zálohy a jejich nahuštěnosti -- je větší snaha držet novější zálohy
a více do historie může pokrytí řídnout (přesná funkce viz níže).

Z~takto spočítaných baddness pro každý záznam o~souboru se pro každý datový
blok vybere ta nejmenší baddness ze souborů, které daný blok využívají, čímž se
efektivně ohodnotí bloky podle toho, jak moc jsou významné pro historii. Pak se
mohou bloky postupně od největší baddness vybírat a mazat. Přitom je nutné dát
pozor na to, že při smazání nějakého bloku (a tím několik verzí souborů) se
může změnit baddness okolním souborům v~historii a ty se musí přepočítat. Dá se
však vypozorovat, že se baddness vždy jenom sníží, čehož se dá využít při
praktickém nasazení ke zjednodušení implementace.

\subsection{Postup počítání baddness}

Čistící třída {\it BackupClean} obsahuje jednu metodu, jejímž voláním se načte
kompletní seznam všech záloh a obsažených souborů, a druhou metodu sloužící ke
smazání vždy jednoho datového bloku s~největší baddness. Postup je tedy takový,
že se jednou načte kompletní seznam a pak se opakovaně volá mazání bloků.

Při načítání se nejprve vytvoří pro každou známou navazující posloupnost souborů
jeden jejich vektor (aby bylo možné snadno nalézt o~jedna novější a o~jedna
starší verzi souboru). Pak se pro každý záznam o~souboru spočítá baddness podle
tohoto vzorce:

$$TODO\over[\hbox{hodnota \texttt{history} z konfigurace}]$$

Ze spočítaných baddness se vybere pro každý datový blok ta nejmenší a pak se
tato hodnota ještě vydělí pro každý datový blok počtem na něj navazujících bloků
(to je heuristika beroucí v~úvahu to, že při vícero navazujících blocích by se
namísto zmenšení mohl celkový datový objem zvětšit, viz dále).

Pak se datové bloky utřídí od největší hodnoty. Po smazání datového bloku se
provede označení všech spojených záznamů o~souborech jako \texttt{DELETED} a
všem okolním záznamům se přepočítá baddness. Pak je systém připraven pro mazání
dalšího bloku.

\subsection{Smazání datového bloku}

Datový blok nelze odstranit jen tak jednoduše, protože pokud by od něj byly
odvozeny jiné datové bloky, tak by se tímto znehodnotily. Proto je nutné
nejdříve všechny navazující datové bloky nechat přepočítat, aby se odkazovaly
na předka mazaného bloku, a teprve poté je možné blok odstranit.

Přepočítání proběhne tak, že se provede obnovení dat předka mazaného souboru a následníků mazaného souboru a spočítá se nový \gls{VCDIFF} mezi nimi. Tím se
změny pokrývané tímto blokem rozředí do změn v~dalších blocích.

V~některých případech se může stát, že se tímto krokem celkový objem dat dokonce
zvětší, protože se velký rozdíl pokrývaný tímto souborem odsune do více
navazujících souborů. Proto je součástí výpočtu baddness datových bloků také
dělení počtem navazujících bloků


\chapter*{Závěr}
\addcontentsline{toc}{chapter}{Závěr}

Zálohovací systém FenixBackup se pokusil nabídnout další řešení problému
zálohování dat. V~ideálním světe by taková věc vůbec potřeba nebyla, bohužel náš
svět ideální není, ke ztrátám či poškození dat v~něm dochází a systémy jako
FenixBackup jsou potřeba.

Častým problémem zálohování je neochota uživatelů strávit čas s~jeho
nastavováním, nebo jeho složité použití ve chvíli ztráty dat. Proto jsem se
pokusil při psaní FenixBackup vycházet z~principů držet pro uživatele vše co
možná nejpřímočařejší.

Současně jsem se pokusil vyřešit i problém omezeného místa pro zálohy, což
vnímám jako další důvod, proč se někdy se zálohováním vyskytují problémy. Často
uživatelé jednou nakonfigurují zálohování a pak věří tomu, že už bude fungovat
napořád -- bohužel se stává, že zálohovacímu systému dojde místo a uživatelé
tomu nevěnují pozornost. FenixBackup se k~tomu staví tak, že má nastavený limit
pro velikost a když je překročen, spustí čištění záloha a maže zálohované verze
souborů s~největším spočteným záporným skóre, dokud nevymaže dostatek dat.

Vytvořený zálohovací systém je doufám na počátku dalšího slibného vývoje a~zkusím
zde nastínit, jakými dalšími směry by se jeho vývoj mohl ubírat.

\newacronym{SMB}{SMB}{Server Message Block, síťový protokol (nejen) pro přenos
souborů, implementace například projektem Samba}

Jedním směrem vývoje je určitě přidání většího množství adaptérů pro získávání
dat. V~prvotní fázi vývoje byl vytvořen jen adaptér pro lokální souborový
systém, ale další adaptéry se nabízejí:
\begin{itemize}
	\item Adaptér připojující vzdálený souborový systém pomocí \gls{SSHFS}
	a dále fungující jako adaptér pro lokální souborový systém.
	\item Adaptér využívající jako protistranu na zálohovaném stroji a jako
	přenosový protokol \texttt{rsync}.
	\item Adaptér připojující se přes protokol \gls{FTP} nebo \gls{SMB} na
	zálohovaný stroj.
	\item Adaptér využívající na zálohovaném stroji běžícího vlastního
	klienta.
\end{itemize}

\newacronym{ACL}{ACL}{Access control list, rozšířený systém práv}

Dalším možným směrem vývoje je modifikovat datový formát tak, aby mohl obsahovat
nepovinné rozšiřující položky -- inspirace například hlavičkami IPv6 paketů,
které obsahují základní společnou hlavičku a pak podle potřeby rozšiřitelné
hlavičky. V~těchto nepovinných položkách by mohl sídlit například systém práv
\gls{ACL}, který se u~nějakých souborů vyskytuje, ale je zbytečné mít pro něj
vyhrazené pevné datové položky u~každého souboru.

Nápad na to použít takový systém ukládání dat přišel bohužel až v~pozdější fázi
vývoje systému, kdy již na předělávání současné implementace nebyl dostatek
času, ale je to jeden z~cílů, kterým se chci dále věnovat.

Jinou oblastí, ve které se může udělat ještě velký pokrok, je přidat více
inteligence do hledání minulých verzí souborů, nebo do hledání vhodných datových
bloků, na které navázat pomocí rozdílové zálohy. Zde je podle mého ještě velký
prostor, kterým se může systém posunout. Implementace jádra systému je na to
připravená, protože už nyní umožňuje specifikovat, proti kterému datovému bloku
má vznikat rozdílová záloha (i když zatím se volí buď předchozí známá verze
souboru, nebo prázdný soubor).

Poslední oblast, nad kterou aktuálně přemýšlím a která by neměla být příliš
složitá k~doimplementaci, je nasazení v~prostředí zálohování více podobných
strojů (typicky nějaká firma nebo škola). Dá se vypozorovat, že při zálohování
většího množství podobných strojů je mnoho souborů napříč stroji stejných
(systémové soubory, společná konfigurace aj.). V~takovém prostředí by bylo velmiTO
efektivní umět mezi zálohami různých strojů sdílet společná data a zmenšit tak
celkovou velikost všech záloh.

Pokud bychom deaktivovali systém mazání starých záloh, umí to FenixBackup již
v~současném stavu -- jednotlivé zálohy budou sdílet společnou datovou složku,
ale budou mít jiné složky, kam ukládají souborové stromy. Protože jsou všechny
informace, jako jsou práva, uloženy v~souborových stromech, a protože jsou
datové bloky identifikovány hashem souboru, který zastupují, tak je možné bez
jakýkoliv problémů tyto datové bloky sdílet. Problém nastane jen, kdyby jedna
ze záloh chtěla datové bloky smazat jako nepotřebné -- v~tu chvíli by bylo
potřeba sledovat využití datových bloků napříč všemi zálohami ukládajícími do
stejného místa, pro což je nutné přidat podporu (aby zálohy o~sobě navzájem
věděly).

FenixBackup je připraven k~použití již v~současném stavu, ale pevně věřím,
že jeho vývoj bude dále pokračovat a dočká se rozšíření ve výše zmíněných
oblastech a také časem většího nasazení, než jen v~řádu jednotek zálohovaných
strojů.


%%% Seznam použité literatury
%%% Seznam použité literatury je zpracován podle platných standardů. Povinnou citační
%%% normou pro bakalářskou práci je ISO 690. Jména časopisů lze uvádět zkráceně, ale jen
%%% v kodifikované podobě. Všechny použité zdroje a prameny musí být řádně citovány.

\def\bibname{Seznam použité literatury}
\begin{thebibliography}{99}
\addcontentsline{toc}{chapter}{\bibname}

\bibitem{progit}
  {\sc Chacon,} Scott.
  \emph{Pro Git}.
  Praha: CZ.NIC, 2009.
  ISBN 978-80-904248-1-4.

\bibitem{rsync}
  {\sc Tridgel,} Andrew et al.
  \emph{Rsync}
  [online]. The Australian National University 1999.
  [Cit. 28.7.2015]
  Dostupné z \url{http://rsync.samba.org/}.

\bibitem{libconfig}
  {\sc Lindner,} Mark.
  \emph{libconfig -- C/C++ Configuration File Library}
  [online].
  [Cit. 28.7.2015]
  Dostupné z \url{http://www.hyperrealm.com/libconfig/}

\bibitem{cereal}
  {\sc Grant,} Shane a {\sc Voorhies,} Randolph.
  \emph{Cereal -- A C++11 library for serialization}
  [online]. University of Southern California.
  [Cit. 28.7.2015]
  Dostupné z \url{http://uscilab.github.io/cereal/}

\bibitem{sha256}
  {\sc Brumme,} Stephan.
  \emph{Portable C++ Hashing Library}
  [online].
  [Cit. 28.7.2015]
  Dostupné z \url{http://create.stephan-brumme.com/hash-library/}

\bibitem{open-vcdiff}
  {\sc Google}.
  \emph{open-vcdiff}
  [online].
  [Cit. 28.7.2015]
  Dostupné z \url{https://code.google.com/p/open-vcdiff/}

\bibitem{boost}
  {\sc Boost community}.
  \emph{Boost Filesystem Library Version 3}
  [online].
  [Cit. 28.7.2015]
  Dostupné z \url{http://www.boost.org/doc/libs/1_46_0/libs/filesystem/v3/doc/index.htm}

\end{thebibliography}


%%% Tabulky v bakalářské práci, existují-li.
% \chapwithtoc{Seznam tabulek}

%%% Použité zkratky v bakalářské práci, existují-li, včetně jejich vysvětlení.
\addcontentsline{toc}{chapter}{Seznam použitých zkratek}
\advance\glsdescwidth by 2cm
\printglossary[type=\acronymtype,title={Seznam použitých zkratek}]

%%% Přílohy k bakalářské práci, existují-li (různé dodatky jako výpisy programů,
%%% diagramy apod.). Každá příloha musí být alespoň jednou odkazována z vlastního
%%% textu práce. Přílohy se číslují.
\chapwithtoc{Přílohy}

{\bf Příloha 1:} Zdrojové kódy zálohovacího systému FenixBackup (včetně ukázkové
konfigurace a několika pomocných souborů).

\openright
\end{document}
