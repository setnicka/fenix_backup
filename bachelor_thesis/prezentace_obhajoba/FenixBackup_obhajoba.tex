\documentclass{beamer}				% Základní vzhled pro prezentaci (s přechody)
%\documentclass[trans]{beamer}			% Zrušení všech přechodů
%\documentclass[handout]{beamer}		% Tisk handoutů

% Pro tisk handoutů
\usepackage{pgfpages}
\mode<handout>{
	\usetheme{default}
	\setbeamercolor{background canvas}{bg=black!5}
	\pgfpagesuselayout{4 on 1}[letterpaper,landscape,border shrink=2.5mm]
}

\beamertemplatenavigationsymbolsempty % Zrušení navigačních klikacích ikonek

\usepackage[utf8]{inputenc}			%%% UTF8 kódování češtiny


\usepackage[czech]{babel}
%\usetheme{AnnArbor}
\usetheme{Warsaw}
\usecolortheme{wolverine}
\usefonttheme{serif}
\usepackage{lscape}
\usepackage{graphicx}
\usepackage{multirow}

\setbeamersize{text margin left=1em,text margin right=1em}
\setbeamertemplate{itemize items}[circle]

% Nastavíme hyperref tak, aby si nestěžoval na odkazy uvnitř dokumentu obsahující české znaky
\hypersetup{unicode}

% Úprava barevnosti záhlaví a zápatí
% šedivočerná -> méně odpoutávají pozornost od obsahu slidu
\setbeamercolor{section in head/foot}{fg=white,bg=gray!40!black}
\setbeamercolor{subsection in head/foot}{fg=black,bg=gray!30}

% Deklarujeme si \mysectionpage (podobne jako partpage)
% (prefixujeme my, protoze \sectionpage v nekterych verzich LaTeXu způsobuje kolizi
\setbeamerfont{section title}{parent=title}
\setbeamercolor{section title}{parent=titlelike}
\defbeamertemplate*{my section page}{default}[1][]
{
	\centering
	\begin{beamercolorbox}[sep=8pt,center,rounded=true,shadow=true,#1]{section title}
		\usebeamerfont{section title}\insertsection\par
	\end{beamercolorbox}
	%\addtocounter{framenumber}{-1}
}
\newcommand*{\mysectionpage}{\usebeamertemplate*{my section page}}

\title[Zálohovací systém]{Zálohovací systém}
\subtitle{Obhajoba bakalářské práce}
\author{Jiří Setnička}
\institute[KAM]{Katedra aplikované matematiky}
\date{7. září 2015}

\begin{document}

\def\O{{\cal O}}

\begin{frame}[plain]
\vfill
\begin{figure}
	\includegraphics[width=100mm,type=pdf,ext=.epdf,read=.epdf]{mff_logo_new-crop}
\end{figure}
\titlepage
\end{frame}

\section{Zálohování obecně}
\subsection{Popis problému}

\begin{frame}{Problém zálohování}
Z pohledu systému:\pause
\begin{itemize}
	\item Držení více historických verzí
	\pause\item Objem zálohy -- co ukládat
	\pause\item Objem zálohy -- ředění historie
\end{itemize}

\bigskip
Z pohledu uživatele:\pause
\begin{itemize}
	\item Konfigurovatelnost
	\pause\item Rychlost -- objem přenášených dat a čas zpracování
	\pause\item Snadná a rychlá obnova dat
\end{itemize}
\end{frame}

\subsection{Zvolené řešení}

\begin{frame}{Zvolené řešení v systému FenixBackup}

{\bf Problém:} Objem zálohy
\begin{itemize}
	\item Jen změněné nebo nové soubory
	\pause\item Rozdílové zálohy souborů (VCDIFF)
	\pause\item Duplikáty
	\pause\item Automatický výběr záloh ke smazání
\end{itemize}

\bigskip\pause

{\bf Problém:} Rychlost
\begin{itemize}
	\item Proskenování souborového stromu
	\pause\item Identifikace nezměněných souborů
\end{itemize}

\bigskip\pause

{\bf Problém:} Univerzálnost
\begin{itemize}
	\item Adaptéry pro připojování
\end{itemize}

\end{frame}

\section{Implementace}
\subsection{Reprezentace dat}

\begin{frame}{Reprezentace dat -- souborové stromy}
Každá záloha = samostatně uložený {\bf souborový strom}

\pause\bigskip

Údaje o každém souboru:
\begin{itemize}
	\item File/directory/symlink
	\pause\item Metadata (vlastník, práva, čas poslední modifikace, \dots)
	\pause\item Verzovací status
	\item Odkaz na datový blok
\end{itemize}
\end{frame}

\begin{frame}{Reprezentace dat -- datové bloky}
\begin{itemize}
	\item Nezávislé, určeny jen obsahem -- žádná metadata
	\pause\item Identifikace SHA256 hashem
	\pause\item VCDIFF (minulá verze/prázdný soubor)
\end{itemize}
\end{frame}

\subsection{Konfigurace}

\begin{frame}[fragile]
\frametitle{Konfigurace}
\begin{verbatim}
paths: (
  { path: "/.git"
    scan: false
  },
  { path: "/test_backup/"
    file_rules: (
      { backup: false },
      { regex: ".*\.fenixtree"
        path_regex: ""
        size_at_least: 10
        size_at_most: 10485760
        backup: true
        priority: 10
      })
  })
\end{verbatim}
\end{frame}

\section{Další vývoj}
\subsection{Možná vylepšení}

\begin{frame}{Další vylepšení}
\begin{itemize}
	\item Ukládání dat -- nepovinné rozšiřující hlavičky (ACL, \dots)
	\pause\item Další adaptéry -- rsync, Samba (stroje s Windows), \dots
	\pause\item Sdílení datových bloků mezi zálohami více strojů
\end{itemize}

\end{frame}

\subsection{Poděkování}
\begin{frame}{}

\centerline{Za vedení práce, poskytované rady a konzultace děkuji svému}
\centerline{vedoucímu Mgr. Martinu Marešovi, Ph.D.}

\bigskip\bigskip

\centerline{Za poskytnuté připomínky děkuji svému oponentovi}
\centerline{Mgr. Pavlu Veselému}

\pause
\bigskip\hrule
\vfill
{\Large\centerline{Děkuji za pozornost}}
\vfill
\end{frame}

\subsection{Reakce na připomínky}
\begin{frame}{Velikost adresářového stromu}
\pause
\begin{tabular}{l | r }
\hline\hline
Parametry (práva, vlastník, velikost, \dots) & 56B \\
Pomocné hodnoty v uzlu & 20B \\
Název souboru (string) & $\sim$30B \\
U souborů (string) & 32B \\
U složek -- odkazy na soubory & započítány souborům \\
Cache cesty (string): & $\sim$120B\\
\hline\hline
\bf Velikost jednoho uzlu stromu: & \bf $\sim$256B
\end{tabular}

\bigskip\pause

Počet souborů (\texttt{find / | wc -l}):
\begin{itemize}
	\item Notebook s Linuxem: $468\,000$
	\item Notebook s Win10: $538\,000$
\end{itemize}

\bigskip\pause

{\bf Vrchní odhad:} $1\,000\,000 \cdot 256\hbox{B} = 244\hbox{MB}$

\end{frame}

\end{document}
